% ./chapters/chapter-2/semiconductor-detectors/pulse-formation/neutrons.tex
% ============================================================================ %
\documentclass[../../../../main.tex]{subfiles}
\begin{document}%
% ============================================================================ %
    \subsubsection{Neutron Interactions and Detection}%
    \label{sec:chapter-2:semiconductor-detectors:pulse-formation:neutrons}%
    % ======================================================================== %
    Neutrons, like \glspl{gamma-ray}, are indirectly ionizing radiation, requiring a nuclear reaction to generate directly ionizing secondary particles.
    Following the neutron interaction mechanisms described in \Xrefsection{chapter-2:radiation-detection:neutron-interactions}, an incident neutron will pass straight through the material until it encounters a nucleus. 
    In a \gls{lise} detector, the neutron must find a \isotope[6]{Li} nucleus, shown in \Xreffigure*{semiconductor-detection-1}, to induce the desirable \gls{li6-neutron-alpha} reaction, described in \Xrefsection{chapter-2:radiation-detection:neutron-interactions:lithium-6}.
    The reaction yields two daughter particles, a \gls{triton-particle} and an \gls{alpha-particle}, traveling in opposite directions through the sensor.
    Following the previous description of charged particle interactions, these secondary particles will ionize atoms through the bulk.
    Along the ion trajectory, electron-hole pairs will be generated, shown in \Xreffigure*{semiconductor-detection-2}.
    An analysis described by Klein relates the kinetic energy of an ionizing radiation particle to the anticipated number of resultant electron-hole pairs produced in a target material \cite{Klein_1968}.
    The quantum yield \Xvariable{Q} describing the number of electron-hole pairs per charged particle is proportional to the energy lost by the ionizing radiation particle \Xvariable{W} and the radiation-ionization energy \Xvariable{\epsilon} shown in \Xrefequation{charge-pair-production-1}.
    % ======================================================================== %
    \Xequationfile{charge-pair-production-1}%
    % ======================================================================== %
    For semiconductors with a bandgap \Xvariable{E_{g} \gtrapprox \SI{2}{\electronvolt}} the radiation-ionization energy can be approximated using the bandgap \Xvariable{\epsilon \approx 3 E_{g}}[].
    Assuming an ionizing radiation particle stops within the bulk sensor, depositing its full kinetic energy, the total number of electron-hole pairs produced by the incident particle can be found from \Xrefequation{charge-pair-production-2}.
    % ======================================================================== %
    \Xequationfile{charge-pair-production-2}%
    % ======================================================================== %
    \citeauthor*{Klein_1968} noted the specific type of radiation does not directly impact this estimation, relying solely on the incident particle's kinetic energy.
    Given the reaction \gls{q-value} of \SI{4.780}{\mega\electronvolt} and a semiconductor band gap of \Xvariable{E_{g}=\SI{2.85}{\electronvolt}} an incident neutron will produce roughly \num{5.6E5} electron-hole pairs \cite{Tupitsyn_2012}.
    These charge carriers are driven to the metal electrodes by the internal electric field generated from the bias voltage, as seen in \Xreffigure*{semiconductor-detection-3}.
    As the charge carriers approach the electrode of opposite polarity, an image charge is created on the thin metal film, generating the electrical pulse indicative of the neutron capture event.
    This charge pulse is collected by the coupling capacitor as it enters the electronic readout circuit.
% ============================================================================ %
\end{document}%