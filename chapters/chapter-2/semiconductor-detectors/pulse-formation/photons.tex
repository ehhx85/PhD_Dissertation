% ./chapters/chapter-2/semiconductor-detectors/pulse-formation/photons.tex
% ============================================================================ %
\documentclass[../../../../main.tex]{subfiles}%
\begin{document}%
% ============================================================================ %
    \subsubsection{Photon Interactions}%
    \label{sec:chapter-2:semiconductor-detectors:pulse-formation:photons}%
    % ======================================================================== %
    In photon sensitive detectors, the incident photon will lose energy as it traverses the material through a series of ionizing interactions.
    At higher photon energies (\Xmath{E_{\gamma} > \SI{1.022}{\mega\electronvolt}}) \glspl{gamma-ray} may interact with nuclei to convert their energy into an electron-positron pair, known as pair production.
    The anti-matter positron eventually combines with an electron in the bulk, annihilating to produce two new gamma photons of \SI{0.511}{\mega\electronvolt} each.
    \Glsplname{gamma-ray} ranging from \SIrange{0.1}{10}{\mega\electronvolt} are deenergized through Compton scattering.
    The incident photon is absorbed by the nucleus, with some of the energy used to eject an atomic electron, and the remainder of the energy imparted to a lower energy gamma ray emission \cite{book:Tsoulfanidis_2010}.
    At energies below \SI{50}{\kilo\electronvolt}, both \glspl{x-ray} and \glspl{gamma-ray} are subject to the photoelectric effect.
    This mechanism is similar to Compton scattering, resulting in an electron emission, however the photon is completely absorbed \cite{book:Turner_1995}.
% ============================================================================ %
\end{document}%