% ./chapters/chapter-2/semiconductor-detectors/pulse-formation/charged-particles.tex
% ============================================================================ %
\documentclass[../../../../main.tex]{subfiles}%
\begin{document}%
% ============================================================================ %
    \subsubsection{Charged Particle Interactions}%
    \label{sec:chapter-2:semiconductor-detectors:pulse-formation:charged-particles}%
    % ======================================================================== %
    Interactions involving charged particle radiation generally interact with the orbital electrons of a target atom, instead of the nucleus.
    Light charged particles include \glspl{beta-plus-particle}, \glspl{beta-minus-particle}, \glspl{delta-ray}, and \glspl{epsilon-ray}, all possessing the rest mass of an electron.
    Heavy charge particles include species with a mass of a proton or greater.
    While this definition includes all heave ions, typically the larger variations are found in extraterrestrial or nuclear reactor environments.
    In neutron detection, smaller nuclei such as protons and \glspl{alpha-particle} are the most common forms of heavy charged particles.
    % ======================================================================== %
    \par%
    % ======================================================================== %
    Heavier charged particles may be slowed through Coulombic interactions with the electrons in the target material, generally carrying a positive electronic charge.
    The charged particles exert an electromagnetic force on the outermost orbital electrons, potentially stripping them from the atom.
    If the atom is not ionized, it may still be excited, potentially leading to other emissions \cite{book:Das_2003}.
    These particles will follow a linear trajectory, releasing a majority of their energy close to the end of travel as described by their Bragg curve.
    To determine the range of \glspl{triton-particle} and \glspl{alpha-particle} in \gls{lise}, the \gls{srim} software package was used to simulate 500 ions of each type traveling through the sensor, shown in \Xreffigure*{ion-travel} \cite{Ziegler_2010}.
    The simulated trajectory cloud offers a statistical average for the range of each ion, evaluating to \SI{39.4}{\micro\meter} for a \gls{triton-particle} and \SI{7.44}{\micro\meter} for an \gls{alpha-particle} \cite{Lukosi_2016}.
    Light particles experience the following loss mechanisms: elastic scattering, inelastic scattering, bremsstrahlung emission (braking radiation), and positron annihilation. 
    These particles, typically negative, scatter off orbital electrons, potentially ejecting them as \glspl{delta-ray} \cite{book:Turner_1995}.
% ============================================================================ %
\end{document}%