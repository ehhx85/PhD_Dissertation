% ./chapters/chapter-2/semiconductor-detectors/charge-collection.tex
% ============================================================================ %
\documentclass[../../../main.tex]{subfiles}
\begin{document}%
% ============================================================================ %
    \subsection{Charge Collection}%
    \label{sec:chapter-2:semiconductor-detectors:charge-collection}%
    % ======================================================================== %
    For a single set of parallel contacts, the Shockley-Ramo Theorem can be used to describe the induced charge \Xvariable{dQ} given a sensor material thickness \Xvariable{W} the number of electron-hole pairs generated by the nuclear interaction \Xvariable{N_0} and the charge of an electron \Xvariable{e}[] \cite{He_2001}.
    The differential distances \Xvariable{dx} are written for electrons and holes respectively.
    The idealized relationship shown in \Xrefequation{charge-induction-rate} neglects the real world losses seen in compound semiconductors.
    % ======================================================================== %
    \Xequationfile{charge-induction-rate}%
    % ======================================================================== %
    Once these electron-hole pairs reach the surface contacts, they create an image charge, collected by the front-end electronics.
    This charge will vary temporally as electrons generally exhibit better charge carrier transport properties compared to holes, arriving at their respective electrode first.
    The total charge on the electrode is given by integrating \Xrefequation{charge-induction-rate}.
    Ideally, the charge is equal to the charge of a single carrier multiplied by the number of carriers produce in the radiation detection event.
    When the effects of charge trapping are introduced via Hecht’s relation, the charge equation takes the following form in \Xrefequation{charge-induction-value} \cite{book:Knoll_2010}.
    % ======================================================================== %
    \Xequationfile{charge-induction-value}%
    % ======================================================================== %
    Here, the electron and hole components are separated since they have different carrier velocities \Xvariable{v} and carrier lifetimes \Xvariable{\tau}[].
    The difference in electron and hole charge carrier properties alters the induced image charge based on the location of charge carrier generation.
    The effect of charge trapping is shown below in \Xreffigure*{position-dependent-charge-collection}, where the ideal case is represented in \Xreffigure{position-dependent-charge-collection:a}, and the real case is represented in \Xreffigure{position-dependent-charge-collection:b}.
    The real case accounts for the fact that electrons typically move faster through the lattice, resulting in decreased charge generated from the holes. 
    As shown in \Xreffigure{position-dependent-charge-collection:a}, the maximum induced charge occurs when the interaction is located at the center.
    When centered, the combined effect of distance based charge trapping is minimized for the electron-hole pair.
    As the location is moved to the edge, the increased distance the electron, or hole, must travel causes non-linear trapping losses.
    Furthermore, the normalized induced charge increases inversely with the detector thickness because more of the charge carriers are able to reach the electrodes without being trapped in the substrate.
    % ======================================================================== %
    \par%
    % ======================================================================== %
    When the difference in charge carrier velocities is factored in the equation, the plot transforms to \Xreffigure{position-dependent-charge-collection:b}.
    It can be seen that for the range of electron to hole mobility ratios, the charge collection is degraded the further from the cathode the interaction occurs.
    This effect is worsened when a large number of trap states exist, either intrinsically or due to imperfections in the crystal after processing.
    For this reason, it was decided the cathode should face the neutron beam to maximize charge collection.
    Since the neutron interaction rate is maximum on the incident surface, the majority of the charge should be deposited near the cathode and follow an attenuation curve through the sensor towards the anode \cite{book:Knoll_2010}.
% ============================================================================ %
\end{document}%\\
