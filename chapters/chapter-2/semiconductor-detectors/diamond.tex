% ./chapters/chapter-2/semiconductor-detectors/diamond.tex
% ============================================================================ %
\documentclass[../../../main.tex]{subfiles}
\begin{document}%
% ============================================================================ %
    \subsection{Diamond}%
    \label{sec:chapter-2:semiconductor-detectors:diamond}%
    % ======================================================================== %
    Among its wide range of applications, diamond has demonstrated radiation detection capabilities.
    Early on, \citeauthor*{Kania_1993} proposed the use of an undoped, intrinsic diamond as an ohmic semiconductor detector with a material resistivty over \SI{E12}{\ohmcm} \cite{Kania_1993}.
    As a generalized ``solid ionization chamber", the diamond detector could theoretically convert any ionizing radiation that interacted with the lattice into a detectable signal.
    As the \gls{cvd} growth process became more reliable, the simple detector was fabricated by depositing thin film metal contacts on both faces of the crystal to form electrodes, tested by \citeauthor*{Behnke_1998}.
    They confirmed a larger resistivity, exceeding \SI{E13}{\ohmcm}, noting the polarisation effect created from trapped charge carriers \cite{Behnke_1998}.
    The high bandgap of diamond (\Xmath{E_g = \SI{5.5}{\electronvolt}}) eliminated the need for additional doping to suppress leakage current at room temperature.
    However, with a bandgap \num{5} times larger than silicon, more energy is required for each charge carrier pair, reducing material sensitivity.
    % ======================================================================== %
    \par%
    % ======================================================================== %
    A decade later, \citeauthor*{Pernegger_2005} was able to successfully apply the \gls{tct} to diamonds under \gls{alpha-particle} radiation  from a \isotope[241]{Am} source, characterizing charge carrier transport times.
    The tests were conducted with a sample measuring \SI[product-units=power]{4 x 4}{\milli\meter}, with an electrical field of \SI[per-mode=symbol]{1}{\volt\per\micro\meter} applied across the \SI{470}{\micro\meter} thick sensor \cite{Pernegger_2005}.
    \citeauthor*{Pomorski_2006} confirmed the high speed operation of the diamond, demonstrating the materials nanosecond pulse response, unintuitively favoring hole drift over electron drift \cite{Pomorski_2006}.
    They also studied the signal degradation apparent after high flux charged particle bombardment.
    While the radiation hard diamond did not see an increase in leakage current or drift velocity, the sensor required annealing to respond with the same pre-irradiation signal amplitude \cite{Pomorski_2007}.
    Because of its radiation hardness, \citeauthor*{Wallny_2007} of the RD42 Diamond Detector Collaboration proposed diamond strip and pixel detectors for beam monitoring in the high radiation environment seen in the \gls{atlas} experiment at the \gls{lhc} \cite{Wallny_2007}.
    % ======================================================================== %
    \par%
    % ======================================================================== %
    Being one of the hardest materials known, diamond can be polished down and maintain an extremely smooth surface (\Xmath{R_{a}\sim\SI{5}{\nano\meter}}), eliminating electronic trapping sites.
    However, the smooth surface creates difficulties in the contact metal adhesion process.    
    \citeauthor*{Galbiati_2009} demonstrated the use of a diamond-like carbon as a tunneling junction to effectively coupled the \ce{Pt} and \ce{Au} metal contacts \cite{Galbiati_2009}.    
    % ======================================================================== %
    \par%
    % ======================================================================== %
    Diamond has been proposed as a potential fast neutron detector for its radiation hardness and high energy resolution.
    \citeauthor*{Pillon_2011} studied fast neutron detection through the \gls{c12-neutron-alpha} reaction as demonstrated in \Xrefequation{reaction-carbon-12} \cite{Pillon_2011}.
    % ======================================================================== %
    \Xequationfile{reaction-carbon-12}%
    % ======================================================================== %
    They also explored the possibility of broad spectrum neutron detection by applying a \gls{lif} converter layer (\SIrange[range-phrase = --]{0.5}{4.0}{\micro\meter}) to the surface of the diamond.
    The \gls{li6-neutron-alpha} reaction converts a neutron to an \glsname{alpha-particle}, escaping the thin conversion layer and depositing in the diamond for secondary detection.
    While limited by size, the system resolved neutron interactions across a broad energy range and demonstrated utility as a dose monitor \cite{Angelone_2011}.
    Recently shown by \citeauthor*{Wulz_2017}, deep channels were etched into the surface of electronics grade diamonds \cite{Wulz_2017}.
    The vias measured \SI{30}{\micro\meter} diameter and penetrated the full \SI{150}{\micro\meter} thick diamond sensor.
    Electroplating was used to fill each channel with chromium, creating a conductive wire through the sensor bulk.
    The \gls{3d} diamond detector prototype successfully demonstrated a response to a \gls{pube} neutron source \cite{thesis:Wulz_2017}.
% ============================================================================ %
\end{document}%