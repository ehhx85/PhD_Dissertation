% ./chapters/chapter-2/semiconductor-detectors/properties-of-semiconductors.tex
% ============================================================================ %
\documentclass[../../../main.tex]{subfiles}
\begin{document}%
% ============================================================================ %
    \subsection{Properties of Semiconductors}%
    \label{sec:chapter-2:semiconductor-detectors:properties-of-semiconductors}%
    % ======================================================================== %
    Semiconductors are a special class of non-metal solids, widely used in modern electronics for their specific material properties.
    Unlike true conductors (\gls{latin:ie} metals) or insulators (\gls{latin:eg} glass), semiconductors fall in the middle, with resistivities commonly ranging from \SIrange{E-2}{E8}{\ohmcm} \cite{book:Owens_2016}.
    Conduction arises as electrons move from atom to atom in a bulk material, carrying with them a negative charge while leaving a positively charge hole as they move.
    In metals, conduction is free to occur because the electron valence band and the conduction band overlap (\Xmath{E_g<\SI{0}{\electronvolt}}), permitting the free exchange of electrons between atomic valence shells.
    Alternatively, insulators have such a wide bandgap, it becomes difficult or impossible to excite the valence electrons with enough energy to reach the conduction band.
    Exhibiting a bandgap \Xvariable{E_g} ranging from \SIrange{0}{4}{\electronvolt}, semiconductors behave like insulators that will conduct once a threshold excitation event occurs.
    A key physical characteristic of semiconductors, the resistance decreases inversely with temperature, approaching conductive behavior at high temperatures and insulating behavior at cold, or even room temperature.
    For this reason, radiation detectors are often operated at liquid nitrogen temperatures (\SI{77}{\kelvin}) to minimize free electrons, with room temperature operation limited to materials with larger band gaps.
    Typically, semiconductors exhibit a crystalline lattice structure, however thin films applications may utilize amorphous or even liquid compounds as well.
    Similar to temperature effects, the electronic behavior of the bulk material is highly sensitive to lattice defects, generally requiring single crystal sensing material.
% ============================================================================ %
\end{document}%