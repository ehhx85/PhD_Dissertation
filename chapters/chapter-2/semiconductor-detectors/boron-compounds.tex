% ./chapters/chapter-2/semiconductor-detectors/boron-compounds.tex
% ============================================================================ %
\documentclass[../../../main.tex]{subfiles}
\begin{document}%
% ============================================================================ %
    \subsection{Boron Compounds}%
    \label{sec:chapter-2:semiconductor-detectors:boron-compounds}%
    % ======================================================================== %
    The boron semiconductor concept functions as an analog of the lithium based \gls{lise} detector.
    By embedding one of the light-weight isotopes (\isotope[6]{Li} or \isotope[10]{B}) into a semiconductor compound, the number density largely exceeds what is possible in a proportional gas.
    The increased reaction density would result in higher neutron stopping power and detection efficiency while reducing the required size of the device.
    % ======================================================================== %
    \par%
    % ======================================================================== %
    Citing earlier studies by \citeauthor*{Kumashiro_1988}, using \ce{^{10}BP}, and \citeauthor*{Bykovskii_1993}, using lithium doped \ce{B_4C} semiconductors, \citeauthor*{Robertson_2002} investigated the application of \ce{B_4C} grown on silicon substrates for neutron detection \cite{Kumashiro_1988, Bykovskii_1993, Robertson_2002}.
    In the study, a \SI{276(5)}{\nano\meter} thick boron carbide layer formed the neutron sensitive film, utilizing natural \isotope[10]{B} enrichment.
    The detector was able to sense reactor neutrons coming from a \gls{triga} neutron beam port, however, the exceedingly thin boron carbide film only achieved a reported \num{3.2e-4} neutron detection efficiency.
    Deviating from the standard semiconductor design, \citeauthor*{Emin_2005} proposed a detector made of two cells, one enriched in \isotope[10]{B} and the other in \isotope[11]{B}.
    He proposed the large Seebeck coefficient of boron carbide could be utilized to detect thermal excitation from neutron absorption in the \isotope[10]{B} side.
    Electronically coupling the two sides, a neutron interaction would produce a small amount of heating, creating a detectable electrical potential between the chemically identical compounds \cite{Emin_2005}.
    Building on prior works implementing boron carbide in heterojunctions, \citeauthor*{Caruso_2006} proposed a heteroisomeric stack, using two distinct isomers of boron carbide to create the diode junction.
    The neutron pulse height spectrum differed from the conversion layer systems, favoring the reaction sum peak over the individual product peaks \cite{Caruso_2006}.
    % ======================================================================== %
    \par%
    % ======================================================================== %
    Conversion layer detectors suffer from conflicting design goals, limited in thickness by the range of secondary particles, creating a ceiling on the detection efficiency.  
    Leaning on decades of research in the semiconductor industry, \citeauthor*{Li_2011} presented a design for a hexagonal boron nitride (\ce{hBN}) micro-strip detector.
    The \SI{1}{\micro\meter} thick \ce{hBN} detector exhibited a conversion efficiency of \SI{80}{\percent} for absorbed neutrons \cite{Li_2011}.
    A few years later, \citeauthor*{Doan_2015} reported the team was able to resolve the individual particle peaks in the neutron pulse height spectrum.
    The refined design switched to an interdigitated micro-strip detector layout (\SI{6}{\micro\meter} spacing and width), across a \SI{0.3}{\micro\meter} thick \ce{hBN} substrate.
    They noted the probability of a sum peak response would remain low until the thickness was increased beyond \SI{5}{\micro\meter}, exceeding the secondary particle ranges \cite{Doan_2015}.
% ============================================================================ %
\end{document}%