% ./chapters/chapter-2/lise/crystal-structure.tex
% ============================================================================ %
\documentclass[../../../main.tex]{subfiles}
\begin{document}%
% ============================================================================ %
    \subsection{Crystal Structure}%
    \label{sec:chapter-2:lise:crystal-structure}%
    % ======================================================================== %
    Single crystal \gls{lise}, was one of many lithium containing \glspl{ternary-chalcogenide} explored by the synthesis team for neutron imaging applications, shown in \Xreffigure*{lise-cube}.
    This ternary I-III-VI semiconductor was designed around the neutron sensitive \isotope[6]{Li} isotope.
    Following the molecular formula \glschemical{ternary-chalcogenide-formula}, the prospects included \ce{^6LiBC_2} (\Xmath{B=} \ce{In} or \ce{Ga}, \Xmath{C=} \ce{S}, \ce{Se} or \ce{Te}) \cite{Tupitsyn_2014}.
    These compounds offer desirable characteristics for radiation detection, including structural and chemical stability, minimizing the use of heavy metals or radioactive compounds.
    \Xreffigure*{lise-crystal-lattice} depicts the orthorhombic unit cell for \gls{lise}, exhibiting a wurtzite crystal structure following the Pna\Xmath{2_1} space group \cite{Wiggins_2016}.
    The \ce{Li} and \ce{In} atoms form tetrahedral bonds, \ce{LiSe_4} and \ce{InSe_4}, connected at shared \ce{Se} atoms to form the larger crystalline structure.
    A summary of crystal lattice parameters have been tabulated in \Xreftable*{lise-lattice-parameters}, capturing the historical changes in experimentally evaluated crystal structure and its correlation with crystal color.
    While the reported empirical values have varied, they remain within the calculated uncertainty, accommodating slight differences following composition and color.
    The consistency in lattice structure across decades of experimentation indicates the coloration and overall electrical performance varies in response to other material characteristics.
% ============================================================================ %
\end{document}%