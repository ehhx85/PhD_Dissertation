% ./chapters/chapter-2/lise/introduction.tex
% ============================================================================ %
\documentclass[../../../main.tex]{subfiles}
\begin{document}%
% ============================================================================ %
    \Xsubsection%
    % ======================================================================== %
    Originally, the semiconductor \gls{lise} was investigated for visible and infrared spectrum optoelectronics \cite{Kamijoh_1980}. 
    In the past decade, the desire for a \gls{solid-state}, direct conversion, neutron detector revived the investigation of lithium based semiconductors.
    In \gls{solid-state} neutron detection, the isotopes \isotope[6]{Li} and \isotope[10]{B} offer the most viable options for semiconductor integration.
    As previously described in \Xrefsection{chapter-2:semiconductor-detectors:boron-compounds}, multiple \isotope[10]{B} containing compounds have been the subject of semiconductor detector research, benefiting from developments in the microelectronics industry.
    Despite their well understood electronic behavior, boron semiconductor sensors have proven difficult to fabricate with an appreciably thickness, limiting the device detection efficiency and applicability in real world detection systems.
    Alternatively, lithium compounds have received less consideration, having a narrower application space, a smaller thermal cross section for \isotope[6]{Li}, and a lower natural isotopic abundance. 
    Nonetheless, recent advances in lithium purification and precursor compound synthesis have promoted \isotope[6]{Li} to the forefront of semiconductor neutron detector technology \cite{Tupitsyn_2012}.
% ============================================================================ %
\end{document}%