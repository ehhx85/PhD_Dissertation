% ./chapters/chapter-2/lise/material-growth.tex
% ============================================================================ %
\documentclass[../../../main.tex]{subfiles}
\begin{document}%
% ============================================================================ %
    \subsection{Material Growth}%
    \label{sec:chapter-2:lise:material-growth}%
    % ======================================================================== %
    In the early 1980's, \citeauthor*{Kamijoh_1981a} conducted initial studies on \glsname{lise} in Tokyo, Japan at Hosei University \cite{Kamijoh_1981a}. 
    The study focused on the development of lithium based \glspl{ternary-chalcogenide}, including \gls{lise}, for optoelectronic applications.
    Their work documented the basic growth process beginning with \ce{LiInS_2}, evolving into \ce{LiInSe_2} and \ce{LiGaSe_2} \cite{Kamijoh_1979, Kamijoh_1981, Kuriyama_1981}.
    The raw precursor materials for \gls{lise} had lower purity, using \ce{Li} (\SI{3}{\nines}), \ce{In} (\SI{5}{\nines}) and \ce{Se} (\SI{5}{\nines}), while the growth process itself has been maintained in current research.
    The final, as-grown boule measured \SI{20}{\milli\meter} long by \SI{10}{\milli\meter} diameter, exhibiting red coloration and a bandgap of \SI{1.88}{\electronvolt}.
    Bulk material properties for \gls{lise} are listed below in \Xreftable*{lise-bulk-properties}.
    % ======================================================================== %
    \par%
    % ======================================================================== %
    The patented detector grade \gls{lise} material was prepared by the synthesis and growth team collaborating between \gls{y12}, \gls{university-fisk} and \gls{university-vanderbilt} \cite{patent:Bell_2010,Tupitsyn_2012}. 
    Both Bridgman growth techniques, horizontal and vertical, were implemented in initial crystal growth processes, with later growth cycles relying on the vertical process (see \Xreffigure*{lise-crystal-growth}) \cite{Stowe_2013, Stowe_2013a}.
    To enhance the sensitivity of \gls{lise} to thermal neutrons, the lithium precursor is enriched to a nominal \SI{95}{\percent} in the \isotope[6]{Li} isotope.
    A custom vacuum distillation process, developed by \citeauthor*{Stowe_2011} at \gls{y12}, provides the ultra high purity (\SI{5}{\nines}) enriched lithium metal suitable for semiconductor applications \cite{Stowe_2011}.
    The \gls{lise} synthesis begins with production of the binary alloy \ce{LiIn}.
    Mixing the enriched lithium with commercially available \SI{6}{\nines} purity indium, the metals are heated in a sealed, \gls{pbn} crucible to \SI{800}{\celsius}.
    Afterwards, \SI{5}{\nines} purity selenium is introduced to the alloy and heated at a temperature of \SI{940}{\celsius}, to form the ternary compound.
    It was shown that \SI{1}{\percent} excess \ce{Li} produced the red sensor coloration while increasing to \SI{3}{\percent} excess \ce{Li} yielded the more desirable yellow sensors \cite{Tupitsyn_2014}.
    % ======================================================================== %
    \par%
    % ======================================================================== %
    The two zone growth furnace operated at a temperature of \SI{940}{\celsius} on the hot side and \SI{760}{\celsius} on the cold side, moving the boule at a linear rate of roughly \SI[per-mode=symbol]{0.7}{\centi\meter\per\day} \cite{Wiggins_2015}.
    The extracted crystal boule measured roughly \SI{50}{\milli\meter} long by \SI{20}{\milli\meter} in diameter.
    Individual sensors were cut from the growth boule into \SI{13}{\milli\meter} diameter wafers ranging in thickness from \SIrange{1.0}{2.0}{\milli\meter} \cite{Stowe_2013a}.
    Preliminary performance validation and material studies were conducted to determine if the raw sensor would yield a viable neutron detector.
% ============================================================================ %
\end{document}%