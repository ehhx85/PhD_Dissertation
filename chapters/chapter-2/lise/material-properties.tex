% ./chapters/chapter-2/lise/material-properties.tex
% ============================================================================ %
\documentclass[../../../main.tex]{subfiles}
\begin{document}%
% ============================================================================ %
    \subsection{Material Properties}%
    \label{sec:chapter-2:lise:material-properties}%
    % ======================================================================== %
    At room temperature, the band gap for \gls{lise} (\Xmath{E_{g}\leq\SI{3}{\electronvolt}}) offers ideal semiconductor operation suitable for radiation detection.
    The high grade single crystal is translucent, ranging in color from deep red (\Xmath{E_{g}\approx\SI{1.9}{\electronvolt}}) as described by \citeauthor*{Kamijoh_1981}, to a light yellow (\Xmath{E_{g}\approx\SI{2.8}{\electronvolt}}) as studied by \citeauthor*{Isaenko_2002}, shown in \Xreffigure*{lise-sensor-pair} \cite{Kamijoh_1981, Isaenko_2002}.
    As reported by \citeauthor*{Wiggins_2013}, the synthesis team collaborating with this study attained a bright chartreuse or greenish yellow (\Xmath{E_{g}\approx\SI{3.0}{\electronvolt}}) in some growths, exhibiting near ideal stoichiometry.
    The visible color is directly correlated to the band gap (see \Xreftable*{lise-bandgap}), indicating a shift from ideal stochiometric atomic ratios and electrical properties \cite{Wiggins_2013}.
    % ======================================================================== %
    \par%
    % ======================================================================== %
    Through ongoing experimentation, the coloration was indicative of the \ce{Se}/(\ce{Li}+\ce{In}) ratio, with \ce{Se} depleted crystals (relative excess of \ce{Li} and \ce{In}) exhibiting a red hue, and stoichiometric crystals appearing more yellow \cite{Petrov_2010, Wiggins_2013, Ma_2015a}.
    Studying the phase diagram for the \gls{solid-state} solution \ce{Li2Se}\--\ce{In2Se3}, \citeauthor*{Weise_2003} noted the color change from yellow to red with increasing molar concentration of \ce{In2Se3}.
    At an even mixture of the two compounds (\SI{50}{\mol\percent} of \ce{In2Se3}) the resultant crystal maintained a yellow coloration with the single ternary phase, \ce{LiInSe2}, present in the solution.
    The solution continued to show a single phase down to a \SI{44}{\mol\percent} concentration of \ce{In2Se3}, after which the binary \ce{Li2Se} phase was present.
    In the other direction, increasing \ce{In2Se3} concentration past \SI{52}{\mol\percent} quickly introduced a new ternary phase, \ce{LiIn5Se8}, shifting the coloration to a light red, then deep red \cite{Weise_2003}.
    During the growth process, the relatively large vapor pressures of lithium and selenium induce evaporation, reducing both \ce{Li}/\ce{In} and \ce{Se}/{In} ratios.
    Furthermore, the preferential evaporation of selenium over lithium, also reduces the \ce{Se}/\ce{Li} ratio, shifting the stoichiometry and possibly introducing secondary phases. 
    % ======================================================================== %
    \par%
    % ======================================================================== %
    Charge trapping in \glschemical{lise} was evaluated by \citeauthor*{Cui_2013} and later confirmed by \citeauthor*{Hamm_2018}, identifying at least three electron and three hole trap states present in the single phase.
    Both \ce{In} and \ce{Se} vacancies exist, agreeing with a \ce{In2Se3} concentration \Xmath{<}\SI{50}{\mol\percent} for yellow crystals.
    Also present are two hole related defects at \ce{Li} on \ce{In} antisites as well as their correspond \ce{In} on \ce{Li} antisites corresponding to electrons \cite{Cui_2013, Hamm_2018}.
    These point defects reduce the carrier lifetime and mobility (see \Xreftable*{lise-optoelectronic-properties}), degrading the electronic performance in semiconductor mode of operation while also increasing self absorption and reducing the light output in scintillator mode \cite{book:Seebauer_2008, Stowe_2014}.
    As a preliminary exclusion criterion, a yellow coloration indicates a sensor suitable for neutron detector fabrication, while red regions in the growth boule are suitable for material studies and fabrication process testing.    
    Furthermore, only the highest quality samples were viable semiconductor candidates, while most yellow sensors showed some degree of scintillation in response to neutrons. 
    % ======================================================================== %    
    \par%
    % ======================================================================== %    
    Investigated for applications in \gls{nlo}, single crystal \gls{lise} exhibited desirable characteristics for \gls{mid-ir} spectrum devices \cite{Garmire_2013,Isaenko_2002}.
    The material exhibits broad optical transparency across the visible spectrum, also beneficial for light yield in scintillators.
    The thermomechanical properties of \glschemical{lise} were evaluated across multiple studies, compiled in \Xreftable*{lise-thermomechanical-properties}.
    These parameters play a vital role in understanding the growth process and resulting phase formed in the crystal bulk.
% ============================================================================ %
\end{document}%