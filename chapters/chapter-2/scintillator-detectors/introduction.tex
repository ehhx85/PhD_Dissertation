% ./chapters/chapter-2/scintillator-detectors/introduction.tex
% ============================================================================ %
\documentclass[../../../main.tex]{subfiles}%
\begin{document}%
% ============================================================================ %
    \Xsubsection%
    % ======================================================================== %
    Scintillation materials for neutron detection have been studied since the discovery of the neutron.
    The physical phenomena of scintillation occurs in response to excitation from a radiation source (see \Xrefsection{chapter-2:radiation-detection:types-of-radiation}), generating photons characteristic of the scintillation material.
    When light is promptly generated with respect to the excitation event (nanoseconds), the process is known as fluorescence, typically producing visible spectrum light.
    If the light output is delayed, the process is termed phosphorescence, yielding light with a longer wavelength and lower energy than fluorescent reactions.
    The phosphorescence in some materials may release absorbed radiation energy for hours after the initial exposure. 
    A chemical compound responsible for fluorescence is termed a fluorophore, while the more general term, phosphor, includes compounds producing either form of luminescence.    
    In the discussion of scintillation materials, highly luminous fluorescent materials are desirable, producing a large amount of light in rapid response to excitation from radiation.
    Phosphorescence should be minimized through the chemical makeup to limit the amount of delayed light output, effectively creating signal noise long after the radiation event transpired \cite{book:Knoll_2010}.
% ============================================================================ %
\end{document}%