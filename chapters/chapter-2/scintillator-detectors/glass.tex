% ./chapters/chapter-2/scintillator-detectors/glass.tex
% ============================================================================ %
\documentclass[../../../main.tex]{subfiles}%
\begin{document}%
% ============================================================================ %
    \subsection{Glasses}%
    \label{sec:chapter-2:scintillator-detectors:glass}%
    % ======================================================================== %
    Glass scintillators offered some of the earliest forms of neutron detection.
    Lithium and boron based silicate glasses utilize experience derived from centuries of usage, more readily available than many of the polymers today.
    Though manufacture requires high temperatures to melt and form the glass (\SI{500}[>]{\celsius}), the resulting material properties offer favorable characteristics for challenging environments.
    Often transparent, glasses allow scintillation light to pass through the bulk, in may cases with a quick response time (\SI{100}[<]{\nano\second}).
    % ======================================================================== %
    \par%
    % ======================================================================== %
    In \citeyear*{Bollinger_1959}, \citeauthor*{Bollinger_1959} noted a lack of suitable \gls{neutron-thermal} detectors for \gls{tof} spectroscopy.
    Two glass scintillators containing boron were reported: the Type GL-55 with composition \ce{Na2O} : \ce{B2O3} : \ce{Al2O3} : \ce{Ce2O3} at a ratio of (1:3:1.3:0.1) and the Type GL-127 with composition \ce{Na2O} : \ce{B2O3} : \ce{SiO2} : \ce{Al2O3} : \ce{Ce2O3} at a ratio of (1:1:1.5:1.3:0.1).
    The decay time constants for the glass were measured to be \SIlist{0.045;0.2;10}{\micro\second} \citeyear*{Bollinger_1959}.
    \citeauthor*{Firk_1961} presented the results of \ce{Li2O} : \ce{SiO2} : \ce{Al2O3} : \ce{Ce2O3} at a ratio of (0.355:0.075:0.02:0.55) enriched to \SI{95}{\percent} \isotope[6]{Li}.
    The glass demonstrated excellent time resolution and better neutron/\gls{gamma-ray} sensitivity over boron-loaded glasses at the time \cite{Firk_1961}.
    % ======================================================================== %
    \par%
    % ======================================================================== %
    \citeauthor*{Katagiri_2004} tested a \ce{Ce^3+} doped (\SI{1}{\percentmole}) scintillating glass consisting of \ce{^6Li^11BP} at a composition of \SIlist{45;20;35}{\percentweight}.
    The \SI{4x5x1}{\milli\meter} sample demonstrated \SI{63.3}{\percent} \gls{neutron-thermal} efficiency \cite{Katagiri_2004}.   
    \citeauthor*{Mizukami_2004} sought to expand the acquisition rate of lithium-loaded glass scintillators for high rate \gls{tof} spectroscopy. 
    They coupled an array of \num{8x8} glass scintillators to a \gls{pmt} operating with \num{64} discrete channels.
    In this way, the combined counts from each channel would be able to surpass the saturation limited count rate from a single larger scintillator.
    For this configuration, a total detector count rate of \SI{15.4}{\mega\countrate} (\SI{8.3}{\mega\countrate\per\centi\meter\squared}) was achievable in a \gls{neutron-thermal} beam \cite{Mizukami_2004}.    
    % ======================================================================== %
    \par%
    % ======================================================================== %
    \citeauthor*{Ishii_2005} synthesized \num{5} mixture ratios of \ce{B2O3}/\ce{Li2O} glass (\SIrange{66}{90}{\percentweight} \ce{B2O3}).
    To enhance light output, \num{6} recipes for \ce{CeO2} doping were used, maxing out at \SI{12.8}{\percentmole}.
    While they reported a \gls{ly} of only \SI{9.2}{\percent}, compared to Bicron GS20 lithium glass, they demonstrated a much fast decay constant of \SI{45}{\nano\second} \cite{Ishii_2005}.
    \citeauthor*{Matsumoto_2005} designed a miniature neutron detector using lithium-6 glass for active neutron field monitoring of inaccessible locations, such as inside a graphite pile.
    The inner \isotope[6]{Li}-glass scintillator, a cylinder of \SI{1}{\milli\meter} diameter by \SI{2}{\milli\meter} long is coupled to a \SI{1}{\milli\meter} optical fiber, embedded in the end of the scintillator.
    To reduce the interference from \gls{gamma-ray}, the original system was wrapped in \SI{40}{\micro\meter} of \ce{Al}, and placed inside a hollowed \ce{CsI(Tl)} scintillator.
    The outer shell of \ce{CsI(Tl)} serves as a \gls{gamma-ray} detector, with the aluminum layer optically decoupling the shell from the inner \isotope[6]{Li} glass.
    The small unit is designed to accurately measure neutron dose, overcoming the extensive time requirements of other methods, such as foil activation and subsequent \gls{naa} \cite{Matsumoto_2005}.   
    \citeauthor*{Ban_2009} used a lithium glass stack to increase the peak separation between \glspl{neutron-ultra-cold} and \glspl{gamma-ray}.
    The top layer in the stack consisted of \SI{100}{\micro\meter} GS3 glass (\SI{0.01}{\percent} \isotope[6]{Li}) while the bottom layer, also \SI{100}{\micro\meter} thick, consisted of GS20 glass (\SI{95}{\percent} \isotope[6]{Li}).
    The resulting stack was senstive to \glspl{neutron-ultra-cold} with velocities as low as \SI{4}{\meter\per\second} \cite{Ban_2009}.
    % ======================================================================== %
    \par%
    % ======================================================================== %
    \citeauthor*{Park_2016} tested a gadolinium and boron-loaded silicate glass, doped with \ce{Ce^3+} to enhance luminescence.
    The glass was varied across compositions conforming to a \ce{25Gd2O3} : \ce{10CaO} : \ce{10SiO2} : \ce{(B_2O3)_{55-x}} : \ce{(CeF3)_x} recipe with doping ranging from \SIrange{0.05}{2.5}{\percentmole} \ce{Ce^3+}.
    The absorption spectrum for the glass varied between \SIrange{330}{440}{\nano\meter} with \ce{Ce^3+} concentration exhibiting an emission peak at \SI{380}{\nano\meter}.
    The experimental decay time constant was evaluated at \SI{36}{\nano\second} at the \SI{0.15}{\percent\mole} \ce{Ce^3+} \cite{Park_2016}.
% ============================================================================ %
\end{document}%