% ./chapters/chapter-2/scintillator-detectors/plastics/introduction.tex
% ============================================================================ %
\documentclass[../../../../main.tex]{subfiles}%
\begin{document}%
% ============================================================================ %
    \Xsubsubsection%
    % ======================================================================== %
    Plastic scintillation materials incorporate a broad range of polymers, loaded with neutron sensitive isotopes.
    The fluorescent component, also called a `fluor' in historical convention, is embedded in the polymer matrix.
    By carefully selecting appropriate base polymers and additives, the compound may be engineered to produce a visibly transparent material, increasing \gls{ly}.
    Plastic scintillators are also desirable for their rapid signal generation time (few nanoseconds) as well as malleability.
    Thermoplastics offer the widest variety of shapes and sizes in comparison to glasses and other ceramics.
    Some of the polymer bases most commonly used in scintillator complexes include: \gls{plastic:pmma}, \gls{plastic:pvt} and \gls{plastic:ps}.
% ============================================================================ %
\end{document}%