% ./chapters/chapter-2/scintillator-detectors/plastics/gadolinium-loaded.tex
% ============================================================================ %
\documentclass[../../../../main.tex]{subfiles}%
\begin{document}%
% ============================================================================ %
    \subsubsection{Gadolinium{\--}Loaded Plastics}%
    \label{sec:chapter-2:scintillator-detectors:plastics:gadolinium-loaded}%
    % ======================================================================== %
    Gadolinium is typically used in scintillators to boost the neutron detection efficiency, possessing an exponentially larger cross section than the other common absorbers.
    \citeauthor*{Ovechkina_2009} investigated a gadolinium-loaded scintillator in a \gls{plastic:ps} base with \gls{plastic:ppo} as a minor additive.
    Using transparent scintillator disks of \SI{14}{\milli\meter} diameter and \SI{3}{\milli\meter} length, they tested mixtures of with \SIlist{0;0.5;1;3}{\percentweight} gadolinium.
    At the \SI{3}{\percentweight} mixture, the relative \gls{ly} was \num{1.08} (\SI{8850}{\lightyield}) compared to \gls{bgo} (\SI{8200}{\lightyield}), as tested \cite{Ovechkina_2009}.
    \citeauthor*{Bedrik_2011} enhanced the \gls{ly} by \SI{20}{\percent} for a \gls{plastic:ps} incorporating gadolinium phenylpropionate using between \SIrange{1.5}{2}{\percentweight} of the luminescent additive \gls{dmdpa} \cite{Bedrik_2011}.
    \citeauthor*{Dumazert_2016} proposed a new design concept for a gadolinium based replacement for the Bonner sphere type neutron detector.
    The system utilizes a gadolinium (metal) core inside a larger plastic sphere, similar to the \isotope[3]{He} filled proportional counters or \isotope[6]{LiI} scintillators operating inside Bonner spheres.    
    Coupled to a \gls{pmt}, the system further requires a method for \gls{gamma-ray} discrimination to avoid pulse pile up, along with optimization of the gadolinium core geometry, as the high cross section leads to interaction concentration at the surface of the sphere \cite{Dumazert_2016}.
% ============================================================================ %
\end{document}%