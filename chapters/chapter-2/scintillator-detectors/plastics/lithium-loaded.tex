% ./chapters/chapter-2/scintillator-detectors/plastics/lithium-loaded.tex
% ============================================================================ %
\documentclass[../../../../main.tex]{subfiles}%
\begin{document}%
% ============================================================================ %
    \subsubsection{Lithium{\--}Loaded Plastics}%
    \label{sec:chapter-2:scintillator-detectors:plastics:lithium-loaded}%
    % ======================================================================== %
    \citeauthor*{Breukers_2013} described a novel \gls{plastic:ps} scintillator incorporating \gls{lime} for neutron detection, with \gls{plastic:ppo} as a \gls{wls} additive.
    Up to this study, \isotope[6]{Li} scintillators were primarily developed as glasses and ceramics (powders or crystals).
    The desire to combine the higher \gls{neutron-thermal} cross section for \isotope[6]{Li} and fast response time (\SI{3}{\nano\second} as tested) of plastic scintillators motivated the project.    
    Furthermore, a bulk plastic scintillator containing \isotope[6]{Li} offers a larger \gls{q-value} (almost double) over comparable \isotope[10]{B} plastics, without the associated \gls{gamma-ray} produced in the more probable \gls{b10-neutron-alpha} reaction branch.
    They were able to successfully incorporate \SI{0.63}{\percentweight} \isotope[6]{Li} in the scintillator \cite{Breukers_2013}.
    \citeauthor*{Cherepy_2015}  tested two carboxylates for separate use in \gls{gamma-ray} detection and neutron detection, both having high solubility in \gls{plastic:pvt}.
    The first material, \gls{bismuth-pivalate} measured a \gls{ly} of \SI{5000}{\lightyield} at \SI{15}{\percentweight} bismuth loading.
    The second material, \gls{lithium-pivalate} was able to maintain transparency with up to \SI{1.9}{\percentweight} lithium, with a \SI{98}{\percent} \gls{neutron-thermal} capture efficiency at a thickness of \SI{3.75}{\milli\meter}.
    Coating the \gls{bismuth-pivalate} plastic in \SI{200}{\micro\meter} of \gls{lif-zns} produced a \gls{phoswich} detector.
    The fast \gls{gamma-ray} detection (\SI{2}{\nano\second}) and slower neutron detection (\SI{200}{\nano\second}), allowed the use of a single \gls{pmt} by implementing \gls{psd} \cite{Cherepy_2015}.
    % ======================================================================== %
    \par%
    % ======================================================================== %
    \citeauthor*{Mabe_2014} provided a method for creating a transparent film containing \SI{2.96}{\percentweight} of \isotope[6]{Li}. 
    The polymer \gls{vppo} was selected, integrating all components of the scintillator mixture directly in the matrix polymer chain, instead of additives commonly used in other mixtures.    
    They reported a decrease in \gls{ly} for charged particles while demonstrating improved neutron/\gls{gamma-ray} discrimination \cite{Mabe_2014}.
    \citeauthor*{Mabe_2016} later investigated two \isotope[6]{Li}-loaded non-aromatic monomers, using them to introduce the lithium component into aromatic polymers (\gls{plastic:ps} or \gls{plastic:pvt}) through copolymerization, adding the potential for \gls{psd}.
    Of the two chemicals, \gls{lisal} was deemed the most favorable over the more difficult to synthesize \gls{lipsa}.
    Lithium-6 was introduced into the polymer chain as anhydrous \ce{^6LiOH}.
    The final result quantified the upper limit of \SI{0.40}{\percentweight} for \isotope[6]{Li} in \gls{lisal}, exhibiting a relative \gls{ly} from \SIrange[separate-uncertainty=false]{55(3)}{70(4)}{\percent} in comparison to a pure \gls{plastic:ps} matrix.
    \cite{Mabe_2016}.
% ============================================================================ %
\end{document}%