% ./chapters/chapter-2/scintillator-detectors/ceramics/clyc.tex
% ============================================================================ %
\documentclass[../../../../main.tex]{subfiles}%
\begin{document}%
% ============================================================================ %
    \subsubsection{CLYC}%
    \label{sec:chapter-2:scintillator-detectors:ceramics:clyc}%
    % ======================================================================== %
    Developed by \gls{rmd}, the scintillator \gls{clyc} aims to provide a suitable alternative to \isotope[3]{He} proportional counters.
    Enriched in \isotope[6]{Li}, the material is operated as a simultaneous \gls{gamma-ray} and neutron detector using \gls{psd} techniques.
    \citeauthor*{DOlympia_2013} demonstrated the sensitivity to thermal neutrons via the \gls{li6-neutron-alpha} reaction.
    The material was also able to distinguish between fast neutrons, interacting via the \gls{cl35-neutron-proton} reaction.
    The measured rise time for \glspl{gamma-ray}, \glspl{neutron-fast}, and \glspl{neutron-thermal} were \SIlist{16; 40; 47}{\nano\second}, respectively \cite{DOlympia_2013}.
    \citeauthor*{Glodo_2013} documented the ten year anniversary by compiling achievements using the \gls{clyc} detector.
    In their work, the detector produced a \gls{gamma-ray} energy resolution of \SI{3.6}{\percent} for a \SI{1}{\inch} diameter sample and \SI{4.1}{\percent} for a \SI{2}{\inch} diameter sample, both \SI{1}{\inch} long \cite{Glodo_2013}.
    \citeauthor*{Smith_2013} presented an analysis of \gls{clyc} coupled to silicon based devices, replacing the traditional \gls{pmt} used in previous configurations.
    They concluded, as a first investigation, the performance of the silicon devices was insufficient to provide \gls{gamma-ray} energy resolution comparable to previous studies, attributed to the lower detection efficiency of the scintillation photons in the silicon devices \cite{Smith_2013}.
    % ======================================================================== %
    \par%
    % ======================================================================== %
    \citeauthor*{Bourne_2014} presented the results of \isotope[137]{Cs} and \isotope[252]{Cf} exposures along with \gls{mcnp} simulations.
    They reported a value of \SI{10}{\percent} for capture efficiency from intrinsic \isotope[6]{Li}, yielding a \num{5} times increase over a matched \isotope[3]{He} volume \cite{Bourne_2014}.
    \citeauthor*{Machrafi_2014} used \gls{mcnp} to generate a set of response functions for \gls{clyc}. 
    The thermal neutron energy peak from the \gls{li6-neutron-alpha} reaction demonstrated a resolution of \SI{3.3}{\percent} with a \SI{1.55}{\mega\electronvolt} shift in the simulated spectrum, indicating \SI{33}{\percent} conversion losses in the scintillator \cite{Machrafi_2014}.
    \citeauthor*{Whitney_2015} showed the results of a \gls{clyc} based radiation camera developed by \gls{rmd}.
    The camera system was able to image \isotope[252]{Cf} and \isotope[241]{AmBe} sources using \gls{psd} to isolate the \gls{gamma-ray} and neutron components \cite{Whitney_2015}.
    \citeauthor*{Giaz_2016} explored the difference between \gls{clyc6} and \gls{clyc7}, using the \isotope[6]{Li} enriched compound to detect \glspl{neutron-thermal} and the \isotope[7]{Li} enriched version for \glspl{neutron-fast}, suppressing the thermal neutron response.
    The results illustrated a significant removal of the \gls{neutron-thermal} signal in \gls{clyc7}, with an efficiency \SI{0.3}[<]{\percent} compared to the geometryically identical \gls{clyc6}.
    In response to \glspl{neutron-fast} the \gls{clyc7} showed a peak for \SI{2.5}{\mega\electronvolt} neutrons and a contiuum for \SI{14.1}{\mega\electronvolt} neutrons \cite{Giaz_2016}.
% ============================================================================ %
\end{document}%