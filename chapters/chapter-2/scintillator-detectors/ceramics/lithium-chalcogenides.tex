% ./chapters/chapter-2/scintillator-detectors/ceramics/lithium-chalcogenides.tex
% ============================================================================ %
\documentclass[../../../../main.tex]{subfiles}%
\begin{document}%
% ============================================================================ %
    \subsubsection{Lithium Chalcogenides}%
    \label{sec:chapter-2:scintillator-detectors:ceramics:lithium-chalcogenides}%
    % ======================================================================== %
    For around four decades, lithium containing \glspl{ternary-chalcogenide}, including \gls{lise}, have been studied as a tunable bandwidth material for \gls{mid-ir} non-linear optics, and more recently as semiconductor detectors.    
    Studying \glspl{ternary-chalcogenide} for neutron detection applications, \citeauthor*{Tupitsyn_2012} refined the growth technique, producing higher quality sensor material than achievable with prior methods.
    In contrast to the red or grayish crystals synthesized by other groups, the bright yellow material exhibited optical transparency, with a room temperature absorption edge near \SI{450}{\nano\meter}, at the spectral transition from blue to violet light \cite{Tupitsyn_2012}.
    While refining the growth mechanics, \citeauthor*{Tupitsyn_2014} confirmed the absorption edge for the yellow material, noting an edge shift for the annealed yellow-red crystals to \SI{525}{\nano\meter} and around \SI{610}{\nano\meter} for red material \cite{Tupitsyn_2014}.
    % ======================================================================== %
    \par%
    % ======================================================================== %
    \citeauthor*{Wiggins_2015} encountered the scintillation mechanism while characterizing \gls{lise} for semiconductor detector performance.
    The light output from a \gls{bgo} scintillator was used for reference, yielding \SI{9000}[\sim]{\lightyield} for \glspl{gamma-ray} with a peak wavelength of \SI{480}{\nano\meter}.
    In comparison, the \gls{lise-scint} demonstrated a \gls{ly} of \SI{4400}{\lightyield} at a peak intensity of \SI{512}{\nano\meter}.
    Absorbing the full \gls{q-value} from the \gls{li6-neutron-alpha} would produce an estimated \SI{21000}{\lightyieldneutron} (thermal energy).
    Furthermore, the study quantified the detection speed of \glspl{lise-scint}, revealing a fast decay constant of \SI{31(1)}{\nano\second} and a slow constant of \SI{143(9)}{\nano\second} \citeyear*{Wiggins_2015}.
    Additional \gls{ly} measurements were conducted as part of the neutron imaging experiments discussed in \Xrefsection{chapter-4:scintillator-array:open-beam} \cite{Lukosi_2016a, Lukosi_2017}.
% ============================================================================ %
\end{document}%