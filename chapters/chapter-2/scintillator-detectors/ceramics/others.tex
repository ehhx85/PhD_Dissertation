% ./chapters/chapter-2/scintillator-detectors/ceramics/others.tex
% ============================================================================ %
\documentclass[../../../../main.tex]{subfiles}%
\begin{document}%
% ============================================================================ %
    \subsubsection{Others}%
    \label{sec:chapter-2:scintillator-detectors:ceramics:others}%
    % ======================================================================== %
    \citeauthor*{Yanagida_2011} grew crystals of \ce{LiCaAlF6}, doped with \ce{Eu^{2+}}, to evaluate the potential for neutron imaging.
    In a sample doped with \SI{2}{\percentmole} \ce{Eu^{2+}}, the measured \gls{ly} came to \SI[separate-uncertainty=false]{29000(2900)}{\lightyield}.
    This material exhibited an emission peak at \SI{375}{\nano\meter} with the smallest decay time for the \SI{2}{\percentmole} sample at \SI{1.15}{\micro\second}.
    The sample was used for \gls{neutron-thermal} imaging with a sub{\--}\si{\milli\meter} spatial resolution \cite{Yanagida_2011}.
    In addition to \ce{Eu^2+} doping, \citeauthor*{Yanagida_2014} later studied the effects of \ce{Ce^3+} doping in \ce{LiCaAlF6}.
    The emission wavelength was again \SI{370(15)}{\nano\meter} for \ce{Eu^2+} and \SI{290(15)}{\nano\meter} for \ce{Ce^3+}.
    The \ce{Ce^3+} exhibited a significantly improved decay time (\SI{40}{\nano\second}) over that for \ce{Eu^2+} (\SI{1.5}{\micro\second}), agreeing with the previous study \cite{Yanagida_2014}.
    \citeauthor*{Wu_2015} developed a new eutectic scintillator composed of \ce{LiCl}/\ce{BaCl2}:\ce{Eu^2+}.
    The eutectic demonstrated two emission peaks at wavelengths \SIlist{406; 526}{\nano\meter}, producing a \SI{412}{\nano\second} decay constant.
    The relative \gls{ly} was determined to be comparable to the lithium glass scintillator, Nucsafe \cite{Wu_2015}.
% ============================================================================ %
\end{document}%