% ./chapters/chapter-2/radiation-detection/radiography.tex
% ============================================================================ %
\documentclass[../../../main.tex]{subfiles}
\begin{document}%
% ============================================================================ %
    \subsection{Radiography}%
    \label{sec:chapter-2:radiation-detection:radiography}%
    % ======================================================================== %
    The \gls{nrc} defines \textit{radiography} as ``The use of sealed sources of ionizing radiation for nondestructive examination of the structure of materials" \cite{website:Nuclear_Regulatory_Commission}.
    The first radiographs were produced by the German physicist Wilhelm R\"{o}ntgen, on November 8, 1895 \cite{book:Turner_1995}.
    R\"{o}ntgen was experimenting with a Crookes-Hittorf tube, a relatively new technology at the time, studying the fluorescent response produced by the energetic cathode rays.
    Unbeknownst to him, the electrons formed in the vacuum had enough energy to produce \glspl{x-ray} as they impinged on the glass surface.
    He discovered this phenomenon after observing a luminescent effect on a barium platinocyanide coated screen across the room from the opaque vacuum tube.
    Shortly after witnessing an image of his own skeleton on the screen during experimentation, he captured the first \gls{x-ray} radiograph of his wife's hand; thus the field of radiography was formed \cite{Riesz_1995}.
    % ======================================================================== %
    \par%
    % ======================================================================== %
    While modern radiography uses more advanced technology than R\"{o}ntegn's experiments, the premise remains the same.
    A controlled source emits radiation on an absorbing material, the incident beam is attenuated and the transmitted beam is recorded via imaging detector.
    Originally developed for \gls{x-ray} applications, radiography may be extended to other forms of radiation as a set of \gls{ndt} techniques.
    It wasn't until the 1960's when the burgeoning nuclear industry prompted the advancement of neutron detection and radiography \cite{Bollinger_1959, Firk_1961}.
    % ======================================================================== %
    \par%
    % ======================================================================== %
    Neutron and \gls{x-ray} imaging are complementary techniques, revealing distinct internal material structures as shown in \Xreffigure*{transmission-imaging-comparison}.
    The absorption cross section for \glspl{x-ray} increases proportional to the atomic number \Xvariable{Z} of the target material, with low-\Xmath{Z} elements appearing more transparent and high-\Xmath{Z} elements more opaque \cite{book:Banhart_2008}.
    Bone has a high calcium content, with a larger \Xmath{Z} value than other components of the human body, thus \glspl{x-ray} are commonly used in medical imaging of skeletal systems.
    In contrast, neutron absorption cross sections vary across elements and even isotopes.
    The average energy lost by a neutron undergoing elastic scattering is maximized for low atomic weights.
    A neutron colliding with hydrogen will have an average energy loss of roughly 1/2 the incident kinetic energy.
    Biological samples tend to concentrate hydrogen, attributed to water content, making neutrons suitable for imaging biological samples.
    In engineering systems, neutron imaging may be used to detect hydrogen gas trapped in defect sites or contrast hydrocarbons from the surrounding mechanical structures.
    % ======================================================================== %
    \par%
    % ======================================================================== %
    Significant contributions to the advancement of radiographic imaging have been accomplished at \gls{psi} \cite{website:Paul_Scherrer_Institute}.
    The metallic Buddha statue offers a clear example of the different capabilities for each imaging technique.
    The \gls{x-ray} image shown in \Xreffigure{transmission-imaging-comparison:a} captures the details corresponding to the metallic shell, consisting of heavy elements.
    The internal structure of the statue is revealed in the neutron image (\Xreffigure{transmission-imaging-comparison:b}), showing a wooden support beam in the center, surrounded by organic plant matter such as flowers and buds.
    As a ritualistic artifact, opening or adulterating the statue would defile the object, a restriction overcome using neutron imaging \cite{Lehmann_2010}.
    Similarly, the analog camera shown in \Xreffigure{transmission-imaging-comparison:c}, reveals metal pins, knobs and mechanical components in the lens focal stage.
    In the neutron image (\Xreffigure{transmission-imaging-comparison:d}), the photographic film, consisting of a plastic strip coated in gelatin emulsion is highly visible.
    Also visible are plastic components, such as the receiving film cartridge, and some glass components including the lens and viewing optics. 
% ============================================================================ %
\end{document}%