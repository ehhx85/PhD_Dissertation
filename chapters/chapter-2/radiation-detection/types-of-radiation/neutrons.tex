% ./chapters/chapter-2/radiation-detection/types-of-radiation/neutrons.tex
% ============================================================================ %
\documentclass[../../../../main.tex]{subfiles}%
\begin{document}%
% ============================================================================ %
    \subsubsection{Neutrons}%
    \label{sec:chapter-2:radiation-detection:types-of-radiation:neutrons}%
    % ======================================================================== %
    Free neutrons are short lived particles (half-life of \SI{636.6(96)}{\second}) emitted from atomic nuclear interactions, with a mass slightly larger than a proton \cite{Christensen_1972}.
    Unlike the other forms of radiation, neutrons do not possess an electric charge, predominately interacting with the nucleus of target atoms or molecules.
    For this reason, they penetrate much deeper than other types of particle radiation, requiring multiple collisions to slow down and stop a neutron.
    Typically, hydrogenous materials are used to moderate, or slow down neutrons, reducing their kinetic energy \cite{book:Knief_2008}.
    % ======================================================================== %
    \par%
    % ======================================================================== %
    Neutrons are often categorized into energy ranges, based on their interaction characteristics and source of generation.
    \Glspl{neutron-fast} are born from nuclear reactions with energies ranging from \SIrange{1.0}{20.0}{\mega\electronvolt} \cite{book:Turner_1995}.
    These energies are characteristic of neutrons created during fission reactions in nuclear reactors.
    Between energies ranging from \SI{1.0}{\electronvolt} to \SI{1.0}{\mega\electronvolt}, multiple sub-ranges are defined, typically useful in reactor engineering.
    In context of neutron imaging, cadmium neutrons are those with energies between \SIrange{0.4}{0.5}{\electronvolt}, and strongly absorbed by the isotope \isotope[113]{Cd}.
    \Glspl{neutron-thermal} have an energy ranging from \SIrange{0.025}{1.0}{\electronvolt},  with the thermal energy of \SI{0.025}{\electronvolt} defined as the most probable energy along a Maxwellian distribution for a particle in equilibrium with an environment at \SI{290}{\kelvin} \cite{book:Turner_1995}.
    % ======================================================================== %
    \par%
    % ======================================================================== %
    As a neutron undergoes a series of collisions, it loses kinetic energy, finally arriving close to the \gls{neutron-thermal} energy.
    Typically, isotopes have an exponentially larger probability of interaction with \glspl{neutron-thermal}, leading to the neutron's removal from most systems once they reach thermal energy.
    \Glspl{neutron-cold}, those with energy less than \SI{0.025}{\electronvolt}, are generated in specially engineered systems.
    To lower the neutron energy beyond the thermal energy, \gls{neutron-cold} sources utilize substances such as liquid deuterium at temperatures around \SI{20}{\kelvin}, further scattering the neutrons without absorbing them \cite{Farrell_2001, Quach_2007, Anghel_2009}.
    The production of low energy \glspl{neutron-cold} accordingly maximizes the neutron wavelength, enhancing resolution in neutron imaging systems.
% ============================================================================ %
\end{document}%