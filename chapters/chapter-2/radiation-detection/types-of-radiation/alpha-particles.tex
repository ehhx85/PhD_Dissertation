% ./chapters/chapter-2/radiation-detection/types-of-radiation/alpha-particles.tex
% ============================================================================ %
\documentclass[../../../../main.tex]{subfiles}
\begin{document}%
% ============================================================================ %
    \subsubsection{Alpha Particles}%
    \label{sec:chapter-2:radiation-detection:types-of-radiation:alpha-particles}%
    % ======================================================================== %
    \Glspl{alpha-particle} are a type of ionizing radiation, akin to a \isotope[4][2]{He} nucleus stripped of its orbital electrons, consisting of two neutrons and two protons.
    \Glspl{norm} emit \glspl{alpha-particle} as a form of decay and include isotopes of uranium, thorium and radium.
    Because of their large size and +2 electronic charge, \glspl{alpha-particle} tend to travel the shortest distances.
    \Glsplname{alpha-particle} can generally be stopped by a normal sheet of paper, human skin (epidermis), or a few inches of atmospheric air \cite{book:Knoll_2010}.
% ============================================================================ %
\end{document}%