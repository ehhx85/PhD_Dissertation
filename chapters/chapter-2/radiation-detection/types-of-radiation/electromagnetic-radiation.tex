% ./chapters/chapter-2/radiation-detection/types-of-radiation/electromagnetic-radiation.tex
% ============================================================================ %
\documentclass[../../../../main.tex]{subfiles}
\begin{document}%
% ============================================================================ %
    \subsubsection{Electromagnetic Radiation}%
    \label{sec:chapter-2:radiation-detection:types-of-radiation:electromagnetic-radiation}%
    % ======================================================================== %
    Electromagnetic radiation can be explained as waves in the electromagnetic field, carrying energy through a space or medium.
    These waves are interchangeably described by their frequency, wavelength or energy.
    While visible light falls in the middle of the electromagnetic spectrum, electromagnetic radiation, as discussed in nuclear engineering, typically lies along the shorter wavelength (higher energy) end of the spectrum.  
    Because electromagnetic radiation exhibits wave-particle duality, it becomes useful to quantize a unit of electromagnetic radiation. 
    The term photon is used to define an elementary particle of zero rest mass, with a velocity defined by the speed of light in a vacuum \cite{book:Leroy_2004}.
    % ======================================================================== %
    \par%
    % ======================================================================== %
    At the middle of the spectrum, visible light ranges from wavelengths of around \SI{750}{\nano\meter} (red) to \SI{380}{\nano\meter} (violet), corresponding to photon energies between \SI{1.65}{\electronvolt} and \SI{3.26}{\electronvolt}.
    The ultraviolet radiation threshold is crossed at a wavelength of \SI{400}{\nano\meter} (\SI{3.1}{\electronvolt}) and increases in energy to a wavelength of \SI{10}{\nano\meter} (\SI{124}{\electronvolt}).
    \Glspl{x-ray} span the wavelengths between \SI{10}{\nano\meter} down to \SI{0.01}{\nano\meter}, possessing enough energy to ionize atoms (\SI{100}{\electronvolt}{\--}\SI{100}{\kilo\electronvolt}), through interactions with orbital electrons.
    Higher energy photons (typically \Xmath{>}\SI{100}{\kilo\electronvolt}) are conventionally termed \glspl{gamma-ray}.
    The generally accepted trend defines the type of radiation based on generation source, with \glspl{x-ray} stemming from electron interactions and gammas from atomic nuclei interactions (thus higher energies) \cite{book:Podgorsak_2006}.
% ============================================================================ %
\end{document}%