% ./chapters/chapter-2/radiation-detection/types-of-radiation/beta-particles.tex
% ============================================================================ %
\documentclass[../../../../main.tex]{subfiles}
\begin{document}%
% ============================================================================ %
    \subsubsection{Beta Particles}%
    \label{sec:chapter-2:radiation-detection:types-of-radiation:beta-particles}%
    % ======================================================================== %
    \Glspl{beta-particle} are a type of ionizing radiation, corresponding to a high-speed, high-energy electrons (\glsplname{beta-minus-particle}) or positrons (\glsplname{beta-plus-particle}).
    This type of radiation is capable of increased penetration depths, maintaining a charge of \Xmath{\pm1} and a relative mass roughly \num{3} orders of magnitude smaller than a proton.
    While \gls{beta-particle} radiation may penetrate human skin, this type of radiation typically requires only a thin layer of metal, plastic, or any other dense material to stop particles \cite{book:Knoll_2010}.
% ============================================================================ %
\end{document}%