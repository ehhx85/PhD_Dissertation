% ./chapters/chapter-2/radiation-detection/neutron-interactions/lithium-6.tex
% ============================================================================ %
\documentclass[../../../../main.tex]{subfiles}
\begin{document}%
% ============================================================================ %
    \subsubsection{Lithium-6}%
    \label{sec:chapter-2:radiation-detection:neutron-interactions:lithium-6}%
    % ======================================================================== %
    Lithium is the lightest solid metal compound, readily available in natural and enriched forms.
    The isotope \isotope[6]{Li} only accounts for a small fraction of the natural lithium, \SI{7.59}{\percent}, and was stockpiled in its enriched form by major nuclear powers across the globe for their nuclear defense industries \cite{Chadwick_2011}.
    This isotope may be converted to tritium through neutron bombardment, useful in general tritium production as well as in thermonuclear weapons to enhance the fusion process \cite{Kouzes_2009}.
    For the same reason, the \SI{940}{\barn} thermal cross section of \isotope[6]{Li} serves a more benign utility in neutron detection \cite{Chadwick_2011}.
    Following a smooth \Xmath{1/v} curve, similar to \isotope[3]{He}, lithium is best suited to detect cold and thermal energy neutrons (see \Xreffigure{cross-section-003-li-6-total}).
    Compared to the other isotopes, the neutron reaction produces a larger \gls{q-value} with the daughter nuclei proceeding along a single branch to the ground state, shown in \Xrefequation{reaction-lithium-6}.
    % ======================================================================== %
    \Xequationfile{reaction-lithium-6}%
    % ======================================================================== %
    These characteristics of the lithium reaction produce a large, fixed energy pulse when a neutron is intercepted.
    As the paramount neutron interaction in this research, the \gls{li6-neutron-alpha} reaction has been graphically illustrated in \Xreffigure*{lithium-6-neutron-reaction}.
    The incident neutron (shown in blue) collides with the \isotope[6]{Li} nucleus consisting of 3 neutrons and 3 protons (shown in red).
    The neutron and \isotope[6]{Li} briefly form an unstable nucleus, then split into two daughter products, an \gls{alpha-particle} and a \gls{triton-particle}.
    Before this research, \isotope[6]{Li} was utilized in scintillator detectors or as a conversion layer in semiconductor neutron imaging systems.
    The isotope has been manufactured in plastic, crystal, glass and liquid scintillators.
    As a conversion layer, the isotope is mixed into a thin film coating, transmuted by the neutron into the \gls{alpha-particle} and \gls{triton-particle} pair for detection in a subsequent charge particle detector.
% ============================================================================ %
\end{document}%