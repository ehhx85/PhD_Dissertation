% ./chapters/chapter-2/radiation-detection/neutron-interactions/helium-3.tex
% ============================================================================ %
\documentclass[../../../../main.tex]{subfiles}
\begin{document}%
% ============================================================================ %
    \subsubsection{Helium-3}%
    \label{sec:chapter-2:radiation-detection:neutron-interactions:helium-3}%
    % ======================================================================== %
    Helium is the second lightest element and non-reactive as a noble gas.
    The isotope \isotope[3]{He} has a very low natural abundance of \SI{0.000134}{\percent}, as its low mass allows any free atoms to boil off in the atmosphere \cite{Bievre_1984}.
    Aside from natural \isotope[3]{He} sources, terrestrial production primarily stems from \gls{beta-minus-particle} decay of stored tritium, produced for the nuclear weapons industry \cite{Zerriffi_1996}.
    The \isotope[3]{He} neutron reaction, \gls{he3-neutron-proton}, is shown below in \Xrefequation{reaction-helium-3}.%
    % ======================================================================== %
    \Xequationfile{reaction-helium-3}%
    % ======================================================================== %
    The isotope's interaction probability follows an almost ideal \Xmath{1/v} trend, with the cross section following an inverse relationship with the particle velocity (see \Xreffigure{cross-section-002-he-3-total}).
    With a large thermal cross section of \SI{5333}{\barn}, gaseous \isotope[3]{He} is commonly used as a neutron conversion gas in \gls{gm} tubes and proportional counters \cite{Chadwick_2011,book:Dawidowski_2013}.
    This isotope is typically used in portal monitors where prior applications aimed to scale up the technology for national security, detecting the proliferated nuclear materials.
    With the decline of active tritium production and total stockpile, the availability of \isotope[3]{He} has been limited, requiring other isotopes as replacement for mass scale neutron detection \cite{website:DOE:helium3, Kouzes_2015}.
% ============================================================================ %
\end{document}%