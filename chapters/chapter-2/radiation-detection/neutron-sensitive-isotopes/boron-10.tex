% ./chapters/chapter-2/radiation-detection/neutron-interactions/boron-10.tex
% ============================================================================ %
\documentclass[../../../../main.tex]{subfiles}%
\begin{document}%
% ============================================================================ %
    \subsubsection{Boron-10}%
    \label{sec:chapter-2:radiation-detection:neutron-interactions:boron-10}%
    % ======================================================================== %
    The most common isotope for slow neutron absorption, \isotope[10]{B} is commonly integrated into nuclear reactor systems for neutron moderation.
    In solid form, borosillicate control rods are strategically placed throughout the core matrix, while boric acid is dissolved in reactor cooling water to control neutron flux.  
    With a large thermal cross section of \SI{3840}{\barn}, and significant natural abbundance of \SI{19.9}{\percent}, this isotope also makes an effective low energy neutron counter \cite{Chadwick_2011}.
    Following the \Xmath{1/v} trend (see \Xreffigure{cross-section-004-b-10-total}), neutrons react with \isotope[10]{B} following one of two \gls{b10-neutron-alpha} reactions, shown in \Xrefequation{reaction-boron-10}.
    % ======================================================================== %
    \Xequationfile{reaction-boron-10}%
    % ======================================================================== %
    Only about \SI{6}{\percent} of reactions branch to the ground state, releasing the full \gls{q-value} for the reaction, while the remaining \SI{94}{\percent} of neutrons produce an excited \isotope[7]{Li}\Xmath{^*} nucleus \cite{book:Knoll_2010}.
    The smaller \gls{q-value} will generate the primary detection peak in proportional counters such as \ce{BF_3} gaseous detectors.
    The \isotope[10]{B} enriched gas offers a high molecular density of the neutron sensitive isotope while performing well as a proportional gas.
    Because the daughter products may travel around \SI{1}{\centi\meter} in the proportional gas, there is a chance one or both of the particles may impact the wall, depositing only a portion of their energy in the gas.
    Termed the ``wall effect'', the resulting spectrum will contain two broad continuum across lower energies, with drop off energies corresponding to the kinetic energy of the daughter products.
    Furthermore, these detectors become susceptible to interference from \glspl{gamma-ray} interacting with the tube wall.
    \Glsplname{gamma-ray} scattering into the tube generate electrons which ionize the proportional gas, resulting in non-neutron energy counts.
    As the flux increases, the resulting counts increase in energy, potentially contaminating the neutron signature.
    % ======================================================================== %
    \par%
    % ======================================================================== %
    As with lithium, \isotope[10]{B} may be integrated into scintillating devices, offering enhanced signal response time.
    Used in neutron \gls{tof} measurements, conversion layers incorporating \isotope[10]{B} provide scintillation light in response to neutrons, but are limited in thickness, and thus overall efficiency, by self attenuation \cite{book:Knoll_2010}.
    Plastic scintillators offer increased \isotope[10]{B} concentration and response time, but suffer from increased gamma sensitivity as the secondary electrons may overwhelm the response of the reaction products.
% ============================================================================ %
\end{document}%