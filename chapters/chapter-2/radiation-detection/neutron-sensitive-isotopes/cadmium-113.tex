% ./chapters/chapter-2/radiation-detection/neutron-interactions/cadmium-113.tex
% ============================================================================ %
\documentclass[../../../../main.tex]{subfiles}
\begin{document}%
% ============================================================================ %
    \subsubsection{Cadmium-113}%
    \label{sec:chapter-2:radiation-detection:neutron-interactions:cadmium-113}%
    % ======================================================================== %
    Cadmium is a heavy metal used in nuclear engineering for its large cross section in the thermal energy regions.
    The isotope \isotope[113]{Cd} is one of 8 naturally occurring isotopes, with a natural abundance of \SI{12.22}{\percent} \cite{Chadwick_2011}.
    This isotope follows the \Xmath{1/v} trend until a dramatic drop off for energies exceeding \Xmath{\sim}\SI{1.0}{\electronvolt} (see \Xreffigure{cross-section-048-cd-113-total}).
    Common with heavier elements, \isotope[113]{Cd} exhibits a resonance region featuring large, narrow peaks, with the principal resonance peak occurring at \SI{0.178}{\electronvolt}.
    The \gls{cd113-neutron-gamma} reaction produces an excited intermediate state, before deenergizing via \gls{gamma-ray} cascade, as shown in \Xrefequation{reaction-cadmium-113} \cite{Pringle_1952, Rusev_2013}.
    % ======================================================================== %
    \Xequationfile{reaction-cadmium-113}%
    % ======================================================================== %
    The large \gls{q-value} from the sum of \gls{gamma-ray} emissions makes this isotope an effective neutron detector for thermal neutrons.
    Because the initial reaction produces \glspl{gamma-ray} instead of charged particles, more sophisticated detector designs are required.
    The \glspl{gamma-ray} must be slowed through nuclear interactions, leading to the production of secondary electrons or scintillation light for detection.
    % ======================================================================== %
    \par%
    % ======================================================================== %
    Cadmium, a known carcinogen in humans, poses health risks targeting organs in the renal and cardiovascular systems.
    The metal may also lead to bone damage, weaken the immune system and increase the rate of reproductive complications \cite{Godt_2006}.
    The modern electronics industry has increased the use of cadmium in compounds such as \ce{CdSe} and \ce{CdTe}, which have the potential to increase the exposure rate to workers as well as the general public \cite{Fowler_2009}.
% ============================================================================ %
\end{document}%