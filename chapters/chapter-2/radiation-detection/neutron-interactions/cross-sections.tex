% ./chapters/chapter-2/radiation-detection/neutron-interactions/cross-sections.tex
% ============================================================================ %
\documentclass[../../../../main.tex]{subfiles}
\begin{document}
% ============================================================================ %
    \subsubsection{Cross Sections}%
    \label{sec:chapter-2:radiation-detection:neutron-interactions:cross-sections}%
    % ======================================================================== %
    In the context of radiation interactions with materials, the term cross section describes the probability that a given type of radiation, will interact with a target particle.
    Furthermore, nuclear cross sections \Xvariable{\sigma(E)} are given as an area, indicating the relative size of a target nucleus to a radiation particle with an energy \Xvariable{E}[].
    The cross section value may be given in standard metric units of \SI{}{\centi\meter\squared} or in the the colloquial unit of barns used in nuclear engineering, with \Xvariable[]{\SI{1}{b}=\SI{E-24}{\centi\meter\squared}} representing a significantly large cross section.
    The total cross section \Xvariable{\sigma_{tot}} is the summation of the scattering \Xvariable{\sigma_{scat}} and absorption \Xvariable{\sigma_{abs}} cross sections, shown in \Xrefequation{neutron-cross-section-total}.%
    % ======================================================================== %
    \Xequationfile{neutron-cross-section-total}%        
    % ======================================================================== %
    In a scattering interaction, \Xrefequation{neutron-cross-section-scattering}, the incident radiation is emitted as a byproduct at the same energy for elastic scattering \Xvariable{\sigma_{e}} or at a lower energy for inelastic scattering \Xvariable{\sigma_{i}}[].%
    % ======================================================================== %
    \Xequationfile{neutron-cross-section-scattering}%
    % ======================================================================== %
    Inelastic collisions will impart some of the energy from the incident particle to the target nucleus, leaving it in an excited state.
    The nucleus may release radiation directly or through nuclear decay as it drops back to its lower energy ground state.
    The absorption cross section \Xvariable{\sigma_{abs}} consists of more terms, varying with the target nucleus as described in \Xrefequation{neutron-cross-section-absorption}.%
    % ======================================================================== %
    \Xequationfile{neutron-cross-section-absorption}%        
    % ======================================================================== %
    Radiative capture \Xvariable{\sigma_{\gamma}} occurs when the incident particle is absorbed by the target nucleus, emitting one or more photons, typically in the gamma energy range.
    In fissionable isotopes, the fission cross section \Xvariable{\sigma_{f}} describes events where the incident particle is absorbed, producing one or more neutron emissions following the decay of the target nucleus.
    Particle emission cross sections are designated by their daughter particle such as the proton emission cross section \Xvariable{\sigma_{p}} and the alpha particle emission cross section \Xvariable{\sigma_{\alpha}}[].
    When all the cross sections are considered, the total cross section for a given isotope may be expressed by its constituents as shown in \Xrefequation{neutron-cross-section-total-expanded}.
    % ======================================================================== %
    \Xequationfile{neutron-cross-section-total-expanded}%        
    % ======================================================================== %
    The microscopic cross sections \Xvariable{\sigma} are useful in describing the probability of interaction with a single target nucleus.
    When describing bulk materials, the macroscopic cross section \Xvariable{\Sigma} is used to evaluate the probability per unit path length a reaction will occur.
    For a pure, uniform material of microscopic cross section \Xvariable{\sigma_{tot}} and atomic number density \Xvariable{N} the total macroscopic cross section \Xvariable{\Sigma_{tot}} will be given by \Xrefequation{neutron-cross-section-macroscopic}.
    % ======================================================================== %
    \Xequationfile{neutron-cross-section-macroscopic}%
    % ======================================================================== %
    For mixed materials, the macroscopic cross section will account for stoichiometric variations, using the molecular number density and the appropriate microscopic cross sections for the included isotopes.
% ============================================================================ %
\end{document}