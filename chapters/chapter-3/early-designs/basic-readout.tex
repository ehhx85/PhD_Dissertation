% ./chapters/chapter-3/early-designs/basic-readout.tex
% ============================================================================ %
\documentclass[../../../main.tex]{subfiles}
\begin{document}%
% ============================================================================ %
    \subsection{Basic Readout System}%
    \label{sec:chapter-3:early-designs:basic-readout}%
    % ======================================================================== %
    The readout electronics provide the bias, collect the charge induced from the radiation event, process the charge into a current or voltage pulse, and generate a conditioned output for further signal processing.
    A \gls{csp} handles the signal processing and must be carefully tailored to the charge produced from the radiation sensor.
    The following three \glspl{csp} were tested with \gls{lise}: the \gls{caen} A1422, the \gls{cr110}, and the \gls{amptek} A250CF CoolFet featuring the optional Peltier cooled input \gls{fet}.
    The \gls{cr110} (shown in \Xreffigure{cr110-pinout:a}) was selected for multiple reasons including: smaller size, circuit simplicity and low cost ($\sim$\SI{50}[\$]{}).
    The basic function of the \gls{preamp} is to rapidly store the total charge created by the detector, as it generates a current pulse.
    The feedback capacitor and resistor \Xvariable{C_f}[] and \Xvariable[]{R_f}[] respectively (see \Xreffigure*{cr110-pinout}), function as an \gls{ac} op-amp integrator with \gls{dc} gain control, creating a slow discharge, at \Xmath{V_{out}}, in the form of a tail pulse signal.
    The rise time of the output signal is rapid and typically on the same time scale as the detector charge collection.
    Ideally, the rise time of the \gls{preamp} should be roughly equivalent to the pulse width of the current pulse from the detector.
    The decay time is much longer to ensure the total charge from the detector is collected, allowing the \gls{preamp} to produce a voltage output proportional to the integrated charge.   
    % ======================================================================== %
    \par%
    % ======================================================================== %
    While the \gls{preamp} chip handles signal processing on the readout \gls{pcb}, the supporting power electronics provide \gls{high-voltage} bias to the detector via \gls{ac} coupling, as shown in \Xreffigure*{cr150-schematic}.
    The use of \gls{ac} coupling isolates the \gls{dc} bias voltage from the \gls{preamp} while still permitting the superimposed \gls{ac} signal from the detection event to be passed to the \gls{preamp}.
    The \gls{high-voltage} bias is cleaned via passive filters following a ladder design; very large resistors (\Xmath{R=\SI{10}{\mega\ohm}}) are connected in series with the bias and small capacitors (\Xmath{C=\SI{0.01}{\micro\farad}}) are connected in parallel to the ground after each resistor.
    Regarding the interconnects, the \gls{high-voltage} bias is applied via \gls{shv}, permitting voltages exceeding \SI{500}{\volt}, while the output signal uses the more traditional \gls{bnc} (\SI{50}{\ohm}).
    \Gls{low-voltage} (\Xmath{\pm\SI{12}{\volt}}) power supplies are used to drive the \gls{preamp} circuitry and connectors vary based on user or designer preference.
% ============================================================================ %
\end{document}%