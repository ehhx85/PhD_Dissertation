% ./chapters/chapter-3/early-designs/single-channel.tex
% ============================================================================ %
\documentclass[../../../main.tex]{subfiles}
\begin{document}%
% ============================================================================ %
    \subsection{Single Channel Counting Detector}%
    \label{sec:chapter-3:early-designs:single-channel}%
    % ======================================================================== %
    The prototype \gls{lise} single channel detector was designed around a \gls{to} header, shown in \Xreffigure*{lise-to-header}.
    Commonly used for transistors, rectifiers, and integrated circuits, the TO-3 packaging specification provided an ample mounting pedestal for the \gls{lise} sensor, typically used by power electronics as a heatsink.
    Two signal leads are integrated into the packaging, backfilled with a glass insulator.
    The top of the leads serve as a bonding pad, facilitating wirebonds to the top contact.
    While the packaging was useful in handling the sensor, an enclosure was required for \gls{emi} shielding, as common with most radiation detectors.
    % ======================================================================== %
    \par%
    % ======================================================================== %
    A simple enclosure was designed to accommodate the specific \gls{to} header, shown in \Xreffigure*{to-header-enclosure}.
    The enclosures fabricated in this work all utilized an aluminum construction with beveled flanges to create a light-tight seal.
    Both softer and lighter than steal, aluminum may be formed into rigid enclosures that are easy to machine and relatively neutron transparent.
    Standoffs were machined from aluminum cylinder stock (\Xreffigure{to-header-enclosure:a}), providing the ground connection to the enclosure (\Xreffigure{to-header-enclosure:b}).
    A bulkhead mounted \gls{shv} (\Xreffigure{to-header-enclosure:c}) provided a matched connection to couple with one of the \glspl{preamp} discussed in \Xrefsection{chapter-3:early-designs:basic-readout}.
    The center pin of the \gls{shv} connector carried the \gls{high-voltage} bias, connected to the \gls{to} header via pin clamp, soldered to the back lead.
    The assembled enclosure is shown in \Xreffigure{to-header-enclosure} along with the TO-3 package for reference.
% ============================================================================ %
\end{document}%