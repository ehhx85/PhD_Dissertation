% ./chapters/chapter-3/advanced-imaging/timepix-sensor.tex
% ============================================================================ %
\documentclass[../../../main.tex]{subfiles}
\begin{document}%
% ============================================================================ %
    \subsection{LISePix Sensor}%
    \label{sec:chapter-3:advanced-imaging:timepix-sensor}%
    % ======================================================================== %
    Fabricating the \gls{lisepix} sensor followed an identical process to the \gls{lise16} sensor, however process control was critical in producing the microscopic \SI{55}{\micro\meter} pitch pixels.
    Invisible to the naked eye, the \SI{30}{\micro\meter} pixel contacts require micrograph analysis throughout the photolithography process to ensure proper pattern transfer and metallization.
    The undercut present in the resist sidewall contributed significantly to the resulting pixel diameter, shown in \Xreffigure*{timepix-sensor-pattern}.
    The deep undercut significantly increased the release rate during liftoff, cleanly removing the large perforated top layer of metal without damaging the contact pads.
    After removing the excess metal, the contact pads are inspected to ensure even deposition across the diameter and to very the size corresponds to the nominal value, shown in \Xreffigure*{timepix-sensor-metal}.
    Carefully scrutinizing the full pixel array served as a final quality control measure before sending the \gls{lisepix} sensor to be bump-bonded.
    Large defects are not suitable for such a device and required subsequent cleaning and reprocessing.
    Any small defects were noted to account for potential hot or dead pixels in the final device.
    Generally, a properly prepared surface produced consistent metallization across the sensor seen in \Xreffigure*{timepix-sensor-microscopy}.
    Reflective \gls{lm} images (\Xreffigure{timepix-sensor-microscopy:a}) reveal pixels with thin metallization, generally characterized by their dull reflection and shift in emitted light spectrum.
    Meanwhile transmitted \gls{lm} images (\Xreffigure{timepix-sensor-microscopy:b}) offer stark contrast between the opaque metal and transparent crystalline regions, illuminating inconsistencies in pixel size or shape.
% ============================================================================ %
\end{document}%