% ./chapters/chapter-3/advanced-imaging/imaging-module.tex
% ============================================================================ %
\documentclass[../../../main.tex]{subfiles}%
\begin{document}%
% ============================================================================ %
    \subsection{LISePix Imaging Module}%
    \label{sec:chapter-3:advanced-imaging:imaging-module}%
    % ======================================================================== %
    The \gls{lisepix} imaging module, shown in \Xreffigure*{timepix-imager-module}, contains the sensor, \gls{asic}, and readout \gls{pcb} (see \Xrefsupplement*{lise-timepix-imager-module}).
    Designed by \gls{xray-imatek}, the readout \gls{pcb} accommodates a single \gls{timepix} controlled by the readout electronics module, shown in \Xreffigure*{timepix-system}.
    As a prototyping \gls{pcb}, the board was installed in a \gls{hammond} 1550B cast aluminum enclosure, providing \gls{emi} shielding.
    Using aluminum standoffs, the \gls{pcb} was mounted to the lid, raising the sensor plane to approximately \SI{2}{\milli\meter} offset from the \SI{1.5}{\milli\meter} thick enclosure.
    A slot was machined into the enclosure and lid providing access for the \SI{0.5}{\meter} long, high-channel density ribbon cable.
    The module was bolted to an aluminum plate with mounting holes spaced at \SI{1.0}{\inch} intervals to mate with T-slotted aluminum extrusions.
% ============================================================================ %
\end{document}%