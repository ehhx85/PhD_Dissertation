% ./chapters/chapter-3/sensor-fabrication/photolithography.tex
% ============================================================================ %
\documentclass[../../../main.tex]{subfiles}
\begin{document}%
% ============================================================================ %
    \subsection{Photolithography and Patterning}%
    \label{sec:chapter-3:sensor-fabrication:photolithography}%
    % ======================================================================== %
    Semiconductor detectors require electrical connections to couple the senor bulk to a voltage bias.
    Applying a voltage through the bulk creates an internal electrical field, used to drive radiation induced charge particles towards the electrical contacts.
    The contacts play a vital roll in signal charge generation, with well adhered metal providing a uniform electrical field.
    The metal contacts throughout this study were deposited using \gls{rf} plasma sputtering via the \gls{sputterer}.
    % ======================================================================== %
    \par%
    % ======================================================================== %
    Photolithography is a fabrication technique using precision optics to transfer a microscopic pattern to the surface of a thin film or bulk crystal material.
    Imaging detectors use photolithography to obtain the micron level features required for \gls{soa} resolution capabilities.
    The sensor is coated in a photosensitive resin, activated by the \gls{uv} light, hardening the resin as described in \Xrefappendix{thin-film-processing:negative-photolithography}.
    Forming a protective layer, the photoresist selectively blocks the metal sputtered by the plasma deposition process.
    % ======================================================================== %
    \par%
    % ======================================================================== %
    Multiple contact recipes were tested in experimentation, using the \gls{rf} plasma deposition process \cite{Hamm_2015}.
    Pure metals were tested including: aluminum, indium and gold.
    Ohmic contacts were successfully formed using indium and gold layers, suitable for reverse biased radiation detection.
    Having a low melting point and a stoichiometric component in the \gls{lise} crystal, indium served as an initial contact candidate.
    After refining the photolithography process and subsequent packaging sequence, the use of gold contacts became the standard practice.
% ========================================================s==================== %
\end{document}%