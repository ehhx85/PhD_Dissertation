% ./chapters/chapter-3/sensor-fabrication/sensor-preparation.tex
% ============================================================================ %
\documentclass[../../../main.tex]{subfiles}
\begin{document}%
% ============================================================================ %
    \subsection{Sensor Preparation}%
    \label{sec:chapter-3:sensor-fabrication:sensor-preparation}%
    % ======================================================================== %
    Independent of their future role, \gls{lise} scintillators and semiconductors are both prepared for radiological testing in identical fashion.
    After growth and sectioning of raw material, the \gls{lise} sensors have sharp edges and rough surfaces.
    An optically smooth surface offers benefits in both modes, decreasing photon scattering at the material interface as well as increasing thin film adhesion during contact deposition.
    Accordingly the first step in sensor preparation begins with etching and polishing.
    % ======================================================================== %
    \par%
    % ======================================================================== %
    The raw \gls{lise} crystal attains the geometric shape specified by the sensor design through rough polishing.
    Using a wet polish with \SI{1.0}{\micro\meter} diamond particle polishing compound, the sensor is manually polished with a \SI{1200}{\grit}, and then \SI{2400}{\grit} silicon carbide polishing pad.
    This stage is used to flatten uneven edges and remove large lattice defects from the crystal.
    The addition of diamond polishing paste helps to smooth and grind any \gls{lise} shards that break free during bulk removal, preventing large streaking or potential cracks.
    % ======================================================================== %
    \par%
    % ======================================================================== %    
    A subsequent mechanical wet polish further reduces surface roughness while also planing the two primary sensor surfaces.
    The crystal is mounted to the polishing chuck using \gls{crystal-bond}, providing a soft layer between the sensor back face and stainless steel chuck.
    The lapping pads consist of \SI{3}{\mil} thick polyester disks coated with a diamond resin, ranging in diamond particle size from \SIrange{30}{0.1}{\micro\meter}.
    Again, the diamond paste containing \SI{1}{\micro\meter} particle compound is used, transitioning to a \SI{0.1}{\micro\meter} compound once the pad particle size drops below \SI{1}{\micro\meter}.
    The weighted piston crystal chuck applies uniform pressure across the crystal surface, evenly polishing the sensor face along the abrasive wheel.
    Polishing speed is minimized to reduced vibrational shock in the relatively soft \gls{lise} crystal.
    After the first side is polished, the polishing chuck is heated on a hot plate, softening the wax so the crystal may be flipped to polish the back face.
    % ======================================================================== %
    \par%
    % ======================================================================== %    
    After mechanical polishing, the sensor is removed from the \gls{crystal-bond} and thoroughly cleaned in \gls{acetone} to remove any remaining \gls{crystal-bond}.
    The \gls{crystal-bond} has reduced sensitivity to the subsequent chemical etch and may lead to uneven surfaces if residual wax coats the surface.
    A final \gls{methanol} rise is used to remove \gls{acetone} streaking and any surface deposits remaining from the mechanical etching process.
    It was beneficial to store the sensor in a small beaker of \gls{methanol} during the elapsed time between polishing and chemical etching.
    The \gls{methanol} helped prevent any further contamination or surface reaction, prepping the sensor for the chemical etching process.
    At this stage, the sensor was chemically etched using a \gls{brometh} solution as described in \Xrefappendix{general-chemistry:sensor-etching}.
    The chemical etch provides a final clean before photolithographic processing, obtaining an optical finish and minimum surface roughness.
% ============================================================================ %
\end{document}%