% ./chapters/chapter-3/sensor-fabrication/packaging.tex
% ============================================================================ %
\documentclass[../../../main.tex]{subfiles}%
\begin{document}%
% ============================================================================ %
    \subsection{Packaging and Electronic Coupling}%
    \label{sec:chapter-3:sensor-fabrication:packaging}%
    % ======================================================================== %
    The final step in fabricating the radiation detection system is coupling the sensor to the readout electronics.
    Electronic coupling takes many forms, including permanent and non-permanent mounting solutions.
    In the laboratory, crystal samples may be readily exchanged for testing using non-permanent, spring loaded contact systems.
    Generally the sensor is placed on a conducting pad, and held down by the very small compressive force from a helical or cantilever type spring.
    Because the sensor is removable, the sample may be cleaned and properly stored in between testing to preserve the material quality and sensitive surfaces.
    Some characterization systems have specialized substrate mounting to subject the sensor to low vacuum or a temperature gradient, also requiring a non-permanent coupling solution. 
    Furthermore, probe stations with microscopic positioning are essential for testing individual channels in pixelated devices.
    % ======================================================================== %
    \par%
    % ======================================================================== %
    After a sensor demonstrates preliminary radiation sensitivity and favorable electronic behavior, the sensor is installed in a more permanent configuration.
    In single channel counters and few channel pixel detectors, wirebonding achieves a semi-permanent bond (see \Xrefappendix{thin-film-processing:wire-bonding}). 
    The sensor is first mounted to a substrate carrier or \gls{pcb} using high purity silver paste if a bottom contact is present, or non-conductive crystal bonding wax if the device only utilizes the top surface.
    A micron scale metal wire is then bonded between each pixel on the detector and a corresponding bond pad on the substrate carrier or \gls{pcb}, forming an electrical connection with the appropriate node in the readout system.
    The substrate carrier or \gls{pcb} may be removed from the detection system, with minor soldering, when not actively in use to store the sensitive crystal.
    This method of coupling is considered semi-permanent because the wires may be carefully pulled from the crystal, and the bonding paste or wax may be chemically dissolved, returning the sensor with only cosmetic imperfections.
    Generally, wirebonding offers enhanced contact performance over compression methods, creating a consistent electrical junction that is more resistant to electromechanical vibrations and shock. 
    % ======================================================================== %
    \par%
    % ======================================================================== %
    The most preferred method of coupling in refined imaging systems is flip-chip bump bonding (see \Xrefappendix{thin-film-processing:bump-bonding}). 
    This technique places the active sensor in proximal contact with the readout electronics, reducing signal line losses and undesirable electronic interference.
    Each pixel on the detector is directly bonded to its corresponding channel on the readout \gls{asic} using a micron scale bead of metal, typically indium.
    This bonding scheme creates the most robust mechanical union, however, removing the bonded sensor from the \gls{asic} can be difficult to near impossible without significantly damaging both the \gls{asic} and the sensor.
    The sensor/\gls{asic} stack is mounted to a readout \gls{pcb} and wirebonded to applicable signal bond pads similar to the previous coupling explanation.
    Using a combination of techniques, an imaging device with thousands of channels may be feasibly constructed with only tens of wirebonds, reducing the overall package form and simplifying the design.
% ============================================================================ %
\end{document}%