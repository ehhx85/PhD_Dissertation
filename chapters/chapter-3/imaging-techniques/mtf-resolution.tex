% ./chapters/chapter-3/imaging-techniques/mtf-resolution.tex
% ============================================================================ %
\documentclass[../../../main.tex]{subfiles}%
\begin{document}%
% ============================================================================ %
    \subsection{Modulation Transfer Function Resolution Calculation}%
    \label{sec:chapter-3:imaging-techniques:mtf-resolution}%
    % ======================================================================== %
    The contrast in an imaging system may be evaluated using the experimental technique shown in \Xreffigure*{mtf-process}, followed by \gls{mtf} analysis as detailed in \Xrefappendix{image-processing:modulation-transfer-function}.
    The left column of the process diagram represents conditions where the spatial period \Xvariable{P_s} (or inverse spatial frequency \Xvariable{f^{-1}_{s}}[]), is equivalent or greater than the resolution limit \Xvariable{R_\textrm{lim}} of the imaging device.
    As the spatial frequency increases past the resolution limit, the response will follow the right column, where contrast reduction limits the ability to fully resolve image features.
    The resolution experiment begins with the selection of an appropriate line pair pattern or sharp knife-edge feature, shown in Step (a).
    An idealized knife-edge feature fully attenuates the incident beam, providing a sudden change in response ranging from the fully open beam of maximum intensity, to the dark field of minimum intensity.
    Step (b) shows the attenuating mask in the neutron radiation field, where a highly absorbing film (such as \ce{Gd}) of given thickness \Xvariable{x} removes a significant portion of the incident beam \Xvariable{I_0} resulting in a modulated output \Xvariable{I_m} captured by the imaging system. 
    The resulting output image, shown in Step (c), follows a periodic shift from open beam (white) to the dark field (black), where minimal gray region indicates high contrast.
    While the left image demonstrates fully black regions along the centerline of attenuating mask, the right image only achieves a gray color, indicative of contrast deficit. 
    This concept is further illustrated in Step (d), where the image is flattened to yield the repeating \glsreset{esf}\gls{esf}.
    Again, the sinusoidal \glspl{esf} (blue) on the left span the full contrast range, in close agreement with the idealized step response (gray), while those on the right side cover a shallow contrast range, significantly above the dark field level.
    The experimentally determined \gls{esf} serves as the input for \gls{mtf} analysis to calculate the resolution limit.
% ============================================================================ %
\end{document}%