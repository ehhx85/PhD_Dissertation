% ./chapters/chapter-3/imaging-techniques/contrast.tex
% ============================================================================ %
\documentclass[../../../main.tex]{subfiles}%
\begin{document}%
% ============================================================================ %
    \subsection{Contrast}%
    \label{sec:chapter-3:imaging-techniques:contrast}%
    % ======================================================================== %
    A key parameter for any imaging system, contrast has been redefined over the past two centuries.
    In its simplest form, contrast relates the intensity difference of two objects in an image or \gls{fov}, giving rise to the concept of bright and dark regions. 
    A basic definition of contrast \Xvariable{C} is shown in \Xrefequation{contrast-basic} where the brightest region has an intensity \Xvariable{I_\textrm{max}} and the darkest \Xvariable{I_\textrm{min}}[].
    % ======================================================================== %
    \Xequationfile{contrast-basic}%
    % ======================================================================== %
    An early definition of contrast was developed by \citeauthor*{book:Weber_1996}, who studied the physiology of the human eye and its ability to distinguish changes in sensory input. 
    He found that the smallest discernible change from a baseline stimulus remained constant across the range of ocular sensitivity.    
    Applying to scenarios where a small, luminous object of intensity\Xvariable{I_\textrm{max}} is present among a broad background of lower intensity \Xvariable{I_\textrm{min}} the Weber contrast uses a normalized intensity difference, shown in \Xrefequation{contrast-weber} \cite{book:Weber_1996}.
    % ======================================================================== %
    \Xequationfile{contrast-weber}%
    % ======================================================================== %
    In these types of images, a relatively uniform background produces an average intensity near one of the extremes, useful in applications such as astronomical images or on-screen text.
    Alternatively, \citeauthor*{book:Michelson_1927} developed a contrast \Xvariable{C_m} for imaging periodic features using the minimum and maximum intensities \Xvariable{I_\textrm{min}}[] and \Xvariable[]{I_\textrm{max}} as described in \Xrefequation{contrast-michelson} \cite{book:Michelson_1927}.
    % ======================================================================== %
    \Xequationfile{contrast-michelson}%
    % ======================================================================== %
    The modulation based contrast proposed by \citeauthor*{book:Michelson_1927} applies to neutron imaging, where the noise is generated from a partially transparent target or mask obscuring the view of the open beam.
    This definition of contrast is applied in the \glsreset{mtf}\gls{mtf} technique used to evaluate the spatial resolution limit for an imaging system \cite{book:Shih_2010}.
    For less standard patterns and geometries, \citeauthor*{Peli_1990} described a \gls{rms} contrast method, relating the normalized gray level in a given pixel \Xvariable{x_i} to the average for the image \Xvariable{\bar{x}} shown in \Xrefequation{contrast-rms}, then applying advanced band-filtering routines to enhance feature visibility \cite{Peli_1990}.
    % ======================================================================== %
    \Xequationfile{contrast-rms}%
% ============================================================================ %
\end{document}%