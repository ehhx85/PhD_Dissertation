% ./chapters/chapter-3/imaging-techniques/simulation.tex
% ============================================================================ %
\documentclass[../../../main.tex]{subfiles}
\begin{document}%
% ============================================================================ %
    \subsection{Simulation}%
    \label{sec:chapter-3:imaging-techniques:simulation}%
    % ======================================================================== %
    During the construction of the single channel \gls{lise} neutron detector, the anticipated \gls{siemens-star} experiment using the \gls{lise16} was simulated using \gls{mcnp}. 
    Because the experiment would require \gls{beamline} time at \gls{hfir} to obtain an appropriate neutron flux, the simulation served as a method to justify the experiment, inspect the pixel array geometry prior to fabrication, and create an idealized output image for reference. 
    Compared against a photo of the \gls{siemens-star} on the left of \Xreffigure{psi-mask-sim-layout}, the right side depicts the experimental geometry.
    The green array illustrates the sensor starting position while the red array represents the final scan position, with the black outline representing the swept area covered during the scan.
    The blue lines, corresponding to spoke cutouts in the \ce{Gd} mask, extend into the swept area, with the outer two concentric rings (\SIlist{400; 500}{\micro\meter}) spanning the scan area.
    % ======================================================================== %
    \par%
    % ======================================================================== %
    The pixel detector design reflected dimensions closer to the first iteration of the \gls{lise16}, using \SI{900}{\micro\meter} pixels.
    For design simplicity, the pixels were simulated edge-to-edge, omitting inter pixel spacing and the surrounding guard ring necessary for real detector electronic operation.
    A plane wave neutron source, positioned \SI{15}{\centi\meter} from the detector surface along the \Xmath{z}-axis, simulated the transport of \num{E9} neutrons at \SI{25}{\milli\electronvolt} for each position.
    The detector was translated in the \Xmath{x} and \Xmath{y}-directions at \SI{50}{\micro\meter} steps, resulting in a total of \num{5329} individual simulations \cite{Herrera_2016}.
    To generate the large number of input deck files, a single file using the starting geometry provided the template.
    Reading this template into \gls{matlab}, a series of near identical files were created, modifying the detector geometry for each input deck, indexing the file name with the positional coordinates.
    These simulations exploited the computational power of the Nuclear Engineering Department Cluster, batch processing the large file set \cite{website:NECluster}.
    The output files were again processed in \gls{matlab}, first rendering 16 individual \gls{super-sampled} images for each channel, composed of a \num{73 x 73} array of \SI{50 x 50}{\micro\meter} virtual pixels, as described in \Xrefappendix{image-processing:super-sampling}.
    Afterwards, the \gls{super-sampled} images were recombined, following the methods in \Xrefappendix{image-processing:mosaic-stitching}, to produce a single image of the \gls{siemens-star} \gls{super-sampling} experiment, shown in \Xreffigure{psi-mask-sim-lise}.
    The simulated \gls{lise16} response confirmed the application of \gls{super-sampling} to achieve sub-pixel pitch resolution with a limited number of readout channels.
    The line pairs are distinctly visible between the \SIlist{400;500}{\micro\meter} rings, indicating a resolution \Xmath{2.25\times} smaller than the \SI{900}{\micro\meter} pixel pitch.
    As an idealized response, the dark field appears uniform after averaging the images together, while the radial markers exhibit smooth curvature across the \gls{fov}.
    Furthermore, noticeable rounding occurs near the spoke to ring interface, deviating from the sharp geometry in the physical mask and the ideal edges created by the simulated mask.
    Shown as the darkest red spots where the spokes perpendicularly meet the ring, these artifacts can be attributed to diffraction effects at the edge of the sharp attenuating feature.
    The presence of these rounded features near the spoke/ring union is likely exacerbated by the use of \gls{super-sampling} with a large pixel, and should also appear to some degree in the physical experiments.
    % ======================================================================== %
    \Xfigurefile{psi-mask-sim-layout}%
    \Xfigurefile{psi-mask-sim-lise}%
% ============================================================================ %
\end{document}%