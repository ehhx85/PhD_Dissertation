% ./chapters/chapter-3/multi-channel-system/sensor-design.tex
% ============================================================================ %
\documentclass[../../../main.tex]{subfiles}%
\begin{document}%
% ============================================================================ %
    \subsection{Sensor Design}%
    \label{sec:chapter-3:multi-channel-system:sensor-design}%
    % ======================================================================== %
    The raw \gls{lise} crystal becomes a semiconductor detector through the negative photolithography process outlined in \Xrefappendix{thin-film-processing:negative-photolithography}.
    A photomask template is drawn using a \gls{cad} editor such as \gls{autocad} and the schematic file is professionally processed into a \SI{5}{\inch} quartz photomask.
    A completed sensor undergoes two rounds of lithography, one side producing the broadface contact and the other side receiving the pixel array and guard ring, shown in \Xreffigure*{lise-16ch-patterned}.
    The \gls{lise16} sensor measured \SI{5x5}{\milli\meter} with a \SI{560}{\micro\meter} thickness as reported for the second version of the \num{4x4} array \cite{Herrera_2016}.
    The sensor was bonded to the substrate \gls{pcb} using \gls{silver-paste} and the pixel metalization was wirebonded to the \gls{pcb} bond pads using the \gls{kns} \gls{wirebonder} following the process described in \Xrefappendix{thin-film-processing:wire-bonding}.
    The first version of the sensor used larger \SI{1000}{\micro\meter} pixels to span the majority of the active area as shown in \Xreffigure*{lise-16ch-wirebonded}.
    Initially, gold contacts demonstrated poor adhesion during wirebonding, due to the unrefined fabrication process.
    To overcome this issue, \ce{In2Se3} contacts were deposited in attempts to form stronger covalent bonds with the crystal bulk.
    While the first version of the sensor was bonded at each channel, some bonds did not make adequate electrical connection and others eventually broke during transport or operation; only two channels were able to effectively collect neutron data.
    % ======================================================================== %
    \par%
    % ======================================================================== %
    Serving as a proof of concept, the v1 sensor demonstrated functional neutron detection paired with the \gls{preamp} and external acquisition electronics.
    Lessons learned during processing and fabrication were applied in the second version sensor, shown in \Xreffigure{lise-16ch-wirebonded:b}.
    A refined polishing and wet etch process resulted in reduced surface roughness and presence of scratching (see \Xrefappendix{thin-film-processing}).
    Fine tuning the plasma sputtering process parameters, including substrate heating, improved the gold contact adhesion on the \gls{lise} crystal, returning to the original design.
    Reducing the pixel pitch to \SI{500}{\micro\meter} focused the active array to the center of the crystal, far enough from the edge to avoid field effects.    
    Depositing a layer of gold roughly \SI{150}{\nano\meter} thick proved sufficient to successfully wirebond all 16 channels and the surrounding guard ring.
% ============================================================================ %
\end{document}%