% ./chapters/chapter-3/multi-channel-system/sensor-board-design.tex
% ============================================================================ %
\documentclass[../../../main.tex]{subfiles}%
\begin{document}%
% ============================================================================ %
    \subsection{Sensor Board Design}%
    \label{sec:chapter-3:multi-channel-system:sensor-board-design}%
    % ======================================================================== %
    The sensor board or substrate \gls{pcb} provides the physical mount for the radiation sensor while also containing the electronic traces needed to couple the readout electronics.
    In attempts to minimize the neutron interactions with the sensor board itself, a thin \gls{alumina} substrate was fabricated into a sensor board.
    The \gls{alumina} board (\Xreffigure*{alumina-pcb-design}), measuring \SI{3 x 3}{\inch} and \SI{635}{\micro\meter} thickness was first metalized using the negative photolithography process and subsequent plasma sputtering (see \Xrefappendix{thin-film-processing:negative-photolithography}).
    The board design was routed in \gls{eagle} with emphasis on shallow bends and near equivalent length signal traces.
    A standard quartz photomask (see \Xreffigure{clean-room:a}) was generated from the circuit design (see \Xreffigure{alumina-pcb-design:a}) and a thin layer of aluminum was deposited on a single side of the \gls{pcb}.
    The initial aluminum metalization offered a relatively inexpensive pattern guide for locating the circuit vias.
    Using a \SI{2}{\milli\meter} diameter diamond studded ceramic drill, each hole was manually carved through the ultra hard \gls{alumina} substrate.
    Holes were drilled while submerged in a water bath to remove heat and debris from the drilling site while also lubricating the cutting surface.
    Even so, some boards cracked during this fabrication step and the process had to be restarted.
    After successfully drilling all vias in the substrate, the board was polished to remove the guide metal and a second photolithograhy run was used to deposit the final gold layering as shown in \Xreffigure{alumina-pcb-design:b}.
    % ======================================================================== %
    \par%
    % ======================================================================== %
    While a prototype \gls{alumina} substrate board was successfully fabricated, the final product did not exhibit favorable electronic characteristics.
    The line resistances across the surface were considerably large due to the trace layer thickness, measuring \Xmath{<}\SI{1}{\micro\meter} versus \SI{34.79}{\micro\meter} for a \SI{1}{\ounce} copper clad \gls{fr4} board.
    Soldering to the board also proved incredibly difficult, with only limited wetting occurring between the copper wire and the solder vias.
    These disadvantages outweighed the benefits of the \gls{alumina} board, and a standard \gls{fr4} board was elected as the replacement.
    The revised substrate board was sized similar to the professionally manufactured \gls{preamp} boards, fitting in the enclosure \gls{pcb} slots for easier mounting.
    \Gls{high-voltage} bias lines (red) were routed on the top of the board, shown in \Xreffigure*{sensor-pcb-layers}, while the signal output traces (blue), run perpendicular across the bottom of the board.
    The bond pads form a circle around the main sensor mounting pad, abruptly connected to a via traversing the \gls{pcb} to the signal trace on the bottom of the board.
    Furthermore, top and bottom ground planes filled the remaining space around the traces to eliminate stray current leaking between signal lines.    
    % ======================================================================== %
    \par%
    % ======================================================================== %
    Both ground planes feature a cutout region surrounding the sensor bonding pad, minimizing the amount of material immediately next to the sensor.
    Copper is used in neutron activation experiments due to its appreciable cross section, and activating the copper cladding on both sides of the board would introduce undesirable \gls{gamma-ray} background (see \Xrefappendix{nuclear-cross-sections:engineering-materials}).
    The light green circle surrounding the sensor in \Xreffigure*{sensor-pcb-fabricated} represents the cutout region.
    The primary experimental \gls{beamline} at \gls{hfir} offered an adjustable beam window via attenuating slits, so neutrons could be directed to only the cutout region and crystal.
    The nine red copper wires at the top of the image correspond to the signal and guard ring lines, soldered directly between the substrate and \gls{preamp} \glspl{pcb}.
    Not shown are the eight matching white wires connecting the remaining channels to the second \gls{preamp} \gls{pcb}.
% ============================================================================ %
\end{document}%