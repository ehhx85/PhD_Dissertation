% ./chapters/chapter-3/beamline-testing/semiconductor-operation.tex
% ============================================================================ %
\documentclass[../../../main.tex]{subfiles}%
\begin{document}%
% ============================================================================ %
    \subsection{Semiconductor Operation}%
    \label{sec:chapter-3:beamline-testing:semiconductor-operation}%
    % ======================================================================== %
    The \gls{cg1d} \gls{beamline} offered a T-slotted aluminum extrusion frame, adjustable to accommodate a range of mounting configurations.
    Aluminum frames provided rigid, lightweight supports offering minimal neutron interaction to reduce background radiation.
    A sliding T-slot bracket was attached to the \gls{lise16} enclosure, with a polyethylene spacer to electrically isolate the enclosure from the surrounding stage.
    Mounting the \gls{lise16} from the top extrusion allowed the system to hang in front of the neutron beam, without obstructing the sample stage below, as demonstrate in the bolt experiment in \Xreffigure*{lise-16ch-beamline-bolt}.
    The \gls{lise16} was installed at an aperture distance of roughly \Xmath{L=} \SI{6.3}{\meter} using the \SI{16}{\milli\meter} aperture to give an \Xmath{L/D=} \num{394}. 
    Cabling was secured to the frame, carrying the \gls{high-voltage}, \gls{low-voltage} and output signal lines to the supporting electronics, shown in \Xreffigure*{lise-16ch-beamline-setup}.
    An \gls{nhq-203m-power-supply} provided the \SI{250}[+]{\volt} \gls{high-voltage} bias, housed in an \gls{ortec} \gls{4001c-nim-bin}.
    The \gls{low-voltage} chip power came from a \gls{keysight} \gls{e3630A-power-supply} operating at \Xmath{\pm\SI{12}{\volt}} with a \SI{170}{\milli\ampere} current draw.
    The detection pulse generated by the \gls{cr110} exited the enclosure via \gls{lemo-00}, routed to the \gls{mcx} inputs on the digitizer using a converter cable.
    Housed in a \gls{caen} \gls{vme8010-crate}, the signals were processed by dual \gls{caen} \glspl{v1724-digitizer} featuring built-in \gls{dpp-pha}.
    A \gls{caen} \gls{v2718-bridge} routed the processed spectra to the acquisition PC, via fiber optic cable, connecting to the \gls{caen} \gls{a3818-card}. 
    To control the acquisition timing and stage position, a custom \gls{labview} module was designed and integrated with the \gls{caen} supplied \glspl{vi}.
    The program connected to the in-house stage controls via ethernet, designating positional information used to index the output files.
    This automation made the \gls{super-sampling} technique possible, requiring thousands of acquisitions across the translation sequence, lasting over \SI{12}{\hour} \cite{Herrera_2016}.
% ============================================================================ %
\end{document}%