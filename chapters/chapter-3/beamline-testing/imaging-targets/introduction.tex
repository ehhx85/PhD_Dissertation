% ./chapters/chapter-3/beamline-testing/imaging-targets/introduction.tex
% ============================================================================ %
\documentclass[../../../../main.tex]{subfiles}%
\begin{document}%
% ============================================================================ %
    \Xsubsubsection%
    % ======================================================================== %
    At \gls{cg1d}, a motorized stage oriented samples in the \gls{beamline}, giving the user fine position control while the beam was open.
    The base of the stage contained a large lift, capable of translating the sample vertically to raise or lower the position within the imager's \gls{fov}. 
    Attached to the lift, two linear drives moved the stage parallel to the ground, one drive translating the sample left and right in the \gls{fov} and the other adjusting the distance between the sample and detector.
    Operating the linear drives with step sizes as low as \SI{25}{\micro\meter} yielded extremely fine positioning accuracy for the \gls{super-sampling} technique.
    An optional fourth axis was attached to the stage for the neutron \gls{ct}, rotating the sample about the vertical axis with \SI{0.25}{\degree} angular steps.
% ============================================================================ %
\end{document}%