% ./chapters/chapter-3/beamline-testing/imaging-targets/biological.tex
% ============================================================================ %
\documentclass[../../../../main.tex]{subfiles}%
\begin{document}%
% ============================================================================ %
    \subsubsection*{Biological}%
    \label{sec:chapter-3:beamline-testing:imaging-targets:biological}%
    % ======================================================================== %
    As an analog to \gls{x-ray} imaging, neutron imaging can be used to explore the internal structure of biological samples.
    \citeauthor*{Bilheux_2014} noted the use of \gls{cg1d} for imaging biological samples including bones, muscles and organ tissues \cite{Bilheux_2014}.
    In the second round of \gls{lise} scintillator array experiments, a neutron \gls{ct} was conducted on a post mortem \textit{halyomorpha halys}, shown in \Xreffigure*{halyomorpha-halys}.
    Observed as non-indigenous agricultural pest in Eastern Tennessee, \textit{halyomorpha halys} frequently intrudes households during the winter, becoming trapped and dying from lack of resources.
    The female sample captured by \citeauthor*{Jones_2009} measured over \SI{16}{\milli\meter} in length, exhibiting the characteristic shield shaped thorax \cite{Jones_2009}.
    For the imaging experiment, a deceased insect was collected and preserved in an aluminum foil pouch, ready for neutron imaging.
    Using the rotational stage, the insect sample was imaged at multiple angles, and a \gls{3d} reconstruction was created using \gls{octopus} \cite{Lukosi_2017}.
% ============================================================================ %
\end{document}%