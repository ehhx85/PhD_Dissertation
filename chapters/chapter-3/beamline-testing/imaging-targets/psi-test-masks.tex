% ./chapters/chapter-3/beamline-testing/imaging-targets/psi-test-masks.tex
% ============================================================================ %
\documentclass[../../../../main.tex]{subfiles}%
\begin{document}%
% ============================================================================ %
    \subsubsection*{PSI Resolution Test Mask}%
    \label{sec:chapter-3:beamline-testing:imaging-targets:psi-test-masks}%
    % ======================================================================== %
    \citeauthor*{Grunzweig_2007} described a resolution test mask for neutron imaging designed and fabricated by \gls{psi} \cite{website:Paul_Scherrer_Institute}.
    Shown in \Xreffigure{psi-mask-small}, the target prescribes a \gls{siemens-star} design, a common resolution assessment tool in optical instrumentation.
    The \gls{siemens-star} is constructed from spokes radiating outwards from the center of a circle.
    The spokes form equally spaced, alternating bright and dark regions, typically on an attenuating background.
    The design used in the \glsreset{siemens-star}\gls{siemens-star} utilizes \num{128} spokes across a \SI{20}{\milli\meter} diameter.
    As the spokes get closer to the center, their period decreases, or the number of line pairs per unit length increases (in units of \si{\linepairs}).
    Concentric rings indicate spoke periods of \SIlist[list-units=single]{50;100;200;300;400;500}{\micro\meter}, used for estimating the resolution of the imaging system.
    In first approximation, the spatial resolution is determined by the period at the radius which no longer distinguishes transparent and attenuating spokes.
    To create transparent and attenuating regions, the quartz mask contains a \SI{6}{\micro\meter} thin film of \ce{Gd} sandwiched between a \SI{25}{\nano\meter} adhesion layer and a \SI{1}{\micro\meter} capping layer.
    The photo negatives in \Xreffigures*{psi-mask-small}{psi-mask-large} show the metalized layers as a reflective surface, with the transparent quartz mask indicated by the non-reflective regions.
    In beam, the \ce{Gd} thin film heavily attenuates the thermal neutron flux, creating attenuating regions while the relatively neutron transparent quartz transmits with approximately open beam intensity. 
    Also featured in the larger mask, a \SI[product-units=power]{24 x 24}{\milli\meter} grid, spaced at \SI{1}{\milli\meter} intervals, offers a means for measuring distortion and magnification \cite{Grunzweig_2007}.
    The bottom of the mask contains a series of knife-edge line pairs, decreasing in spacing, to measure the \gls{mtf} as described in \Xrefappendix{image-processing:modulation-transfer-function}. 
% ============================================================================ %
\end{document}%