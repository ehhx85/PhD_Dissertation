% ./chapters/chapter-3/beamline-testing/timepix-operation.tex
% ============================================================================ %
\documentclass[../../../main.tex]{subfiles}
\begin{document}%
% ============================================================================ %
    \subsection{Timepix Operation}%
    \label{sec:chapter-3:beamline-testing:timepix-operation}%
    % ======================================================================== %
    The \gls{lisepix} system maintained a smaller footprint than its predecessor, the \gls{lise16}, shown in \Xreffigure*{timepix-beamline}.
    The data processing module sat on top of the T-slotted aluminum extrusions, safely out of the neutron beam to protect the sensitive electronics.
    The module only required two input cables, a standard ethernet patch cord to relay data to the acquisition computer, and a standard \gls{dc} power supply running off \SI{110}{\voltac}.
    Hung from an aluminum mounting plate, the \gls{lisepix} imaging module took position immediately behind the imaging targets, roughly centered on the beam axis.
    The blue, high-channel density ribbon cable connected the imaging \gls{pcb}, carrying the \gls{high-voltage} bias and \gls{asic} data to the readout module.
    Positioned in front of the scintillator \gls{dark-box}, the \gls{lisepix} established an aperture distance of \Xmath{L=} \SI{6}{\meter}, tested with two aperture settings (\Xmath{D=} \SIlist[list-units=single]{8.2; 16}{\milli\meter}), resulting in \Xmath{L/D} ratios of \numlist{732; 375}, respectively.
    Operating in \gls{medipix2} event counting mode, the \gls{high-voltage} bias of \SI{300}[+]{\volt} produced an internal electric field of \SI{545}{\electricfieldmm} in the sensor \cite{Herrera_2018}.
% ============================================================================ %
\end{document}%