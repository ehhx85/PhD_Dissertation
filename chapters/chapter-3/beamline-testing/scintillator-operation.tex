% ./chapters/chapter-3/beamline-testing/scintillator-operation.tex
% ============================================================================ %
\documentclass[../../../main.tex]{subfiles}
\begin{document}%
% ============================================================================ %
    \subsection{Scintillator Operation}%
    \label{sec:chapter-3:beamline-testing:scintillator-operation}%
    % ======================================================================== %
    The \gls{cg1d} \gls{beamline} features a fixed \gls{high-res}, scintillator based imaging system installed at the end of the beam line (see \Xreffigure*{scintillator-experiment}).
    A \gls{dark-box} houses the \gls{andor-ccd}, featuring a \SI{7 x 7}{\centi\meter} \gls{fov}, yielding a typical spatial resolution around \SI{75}{\micro\meter} using the \gls{scintillator-screen} \cite{Santodonato_2015}.
    The \gls{andor-ccd} rests at the bottom of the \gls{dark-box}, aiming vertically towards an aluminum reflecting mirror.
    Oriented at a \SI{90}{\degree} angle to the \gls{andor-ccd}, the \gls{scintillator-screen} consists of a thin aluminum plate (\SI[product-units=power]{22 x 22}{\centi\meter} by \SI{0.4}{\milli\meter} thick) coated in a neutron sensitive film mixture of a \gls{lif} absorber, \ce{ZnS} fluorescent pigment, and \ce{Cu} activator, ranging from \SIrange{50}{200}{\micro\meter} thick \cite{website:RC-TRITEC}.
    When a neutron strikes the plate, an absorption reaction produces scintillation light that reflects off the \SI{45}{\degree} mirror, redirected downwards through a magnification lens to the \gls{andor-ccd} \cite{Bilheux_2014}.
    In this research, a \gls{scintillator-screen} with a \SI{50}{\micro\meter} thick coating was used.    
    % ======================================================================== %
    \par%
    % ======================================================================== %
    This neutron imaging system served as a standard for comparison against both \gls{lise} modes, as well as positioning the semiconductor systems in the neutron beam.
    Using the in-house optics stage, \gls{lise} scintillating sensors were attached to an identical aluminum plate (see \Xreffigure*{lise-scintillator-array}), installed in the \gls{ccd} \gls{dark-box}.
    Sensors were affixed using aluminum tape, minimizing neutron attenuation and scattering, with both transparent (reflective) and opaque (anti-reflective) coatings.
    Operating with an aperture of \Xmath{D=} \SI{16}{\milli\meter} and sitting at the full \gls{beamline} distance, the \gls{dark-box} produced an \Xmath{L/D=} \num{412} \cite{Lukosi_2016a}.
    The first version of the scintillator array (\Xreffigure{lise-scintillator-array:a}) tested the \num{11} available sensors to compare light output, measure spatial resolution, and test the \gls{super-sampling} technique.
    Tested individually in an earlier experiment, sample \gls{lise}-1 provided the multi-position images used for image reconstruction \cite{Lukosi_2016a}.
    In the second iteration of the scintillator array (\Xreffigure{lise-scintillator-array:b}) the sensors formed a pseudo open beam area, packing the crystals as close together as possible.
    To better asses the significance of surface finish and material thickness, the scintillators were specifically prepared at varying finishes, shown in \Xreftable*{lise-scintillator-finishes}.
    This experiment focused on determining the spatial resolution limit of the scintillator mode, along with applications of neutron \gls{ct} \cite{Lukosi_2017}.
% ============================================================================ %
\end{document}%