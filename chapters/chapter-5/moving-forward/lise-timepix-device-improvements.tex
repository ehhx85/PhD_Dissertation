% ./chapters/chapter-5/moving-forward/lise-timepix-device-improvements.tex
% ============================================================================ %
\documentclass[../../../main.tex]{subfiles}
\begin{document}%
% ============================================================================ %
    \subsection{LISe Timepix Device Improvements}%
    \label{sec:chapter-5:moving-forward:lise-timepix-device-improvements}%
    % ======================================================================== %
    The \gls{lisepix} fabrication still requires the following processing improvements: optimized sensor thickness, usage of a guard ring to stabilize electrical field, material growth to reduce inclusions and improve hole carrier transport, and bump bonding consistency.
    Perfecting the fabrication process is the first critical step in continuing with the \gls{lisepix} technology.
    The \gls{lise-scint} array demonstrated the ability to tile multiple sensors together to create a large active area, which may offer a temporary avenue for covering an entire \gls{timepix} surface while the growth diameter improves.
    Upon successfully utilizing an entire single \gls{asic}, the next logical step would be to move towards a multi-\glstext{timepix} module described in \Xreftable{lisepix-cost}.
    The readout platform used in the \gls{lisepix} prototype currently supports \num{4} \glspl{asic} running in tandem using the provided multi-\glstext{timepix} detector \gls{pcb}.
    Currently, this technology avenue has been pursued by the company \gls{advacam}, collaborating with the \gls{medipix} team at \gls{cern}.
    Pushing the commercialization of \gls{high-res} radiography platforms, \gls{advacam} has demonstrated a unique \num{10x10} \gls{timepix} array device, known as the \gls{widepix}, shown in \Xreffigure*{timepix-multi-module}.
    The \gls{soa} system boasts a \SI{6.5}{\mega\pixel} (\SI[product-units=power]{14.3x14.3}{\centi\meter}) \gls{fov} used for \gls{high-res} \gls{x-ray} imaging at a maximum \SI{20}{\framespersecond} \cite{Jakubek_2014}.
% ============================================================================ %
\end{document}%