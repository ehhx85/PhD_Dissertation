% ./chapters/chapter-5/moving-forward/time-of-flight.tex
% ============================================================================ %
\documentclass[../../../main.tex]{subfiles}
\begin{document}%
% ============================================================================ %
    \subsection{Time of Flight Measurements}%
    \label{sec:chapter-5:moving-forward:time-of-flight}%
    % ======================================================================== %
    The \gls{tof} technique offers a method for distinguishing neutrons based on their kinetic energy.
    The basic structure of a \gls{tof} experiment involves a neutron source, a crystalline monochromator (effecitvely a neutron mirror), a chopper, the sample and the detector array.
    Neutrons are generated and directed as in other beam lines, however they are Bragg reflected off the monochromator, and pulsed using a chopper (a revolving neutron attenuator periodically interupting the beam).
    By changing the momentum and wavelength of the neutrons with a well documented experiment geometry, the energy of the neutrons striking the target may be calculated \cite{Copley_1992}. 
    To properly utilize such measurements, the pulse rise time must be sufficiently fast to distinguish between neutron arrival times.
    \citeauthor*{Wiggins_2015} measured a fast and slow time constant for the \gls{lise-scint} at \SIlist{31(1); 143(9)}{\nano\second}, respectively, meanwhile \citeauthor*{Hamm_2015} measured a time constant of \SIrange{100}{150}{\nano\second} for a range of semiconductor sensors \cite{Wiggins_2015, Hamm_2015, thesis:Hamm_2018}.
    With these timing charactersistics, refined \gls{lise} neutron detectors exhibiting neutron sensitivity across the broader neutron energy spectrum may find application as a spectrometer in new \gls{tof} experiments such as \gls{sns-venus} \cite{Bilheux_2015}.
    This type of measurement is particularly useful in differentiating materials where the attenuation cross section is largely energy dependent.
    Some areas of research include materials science, \gls{am}, engineering materials and failure studies, and biological systems. 
% ============================================================================ %
\end{document}%