% ./chapters/chapter-5/moving-forward/coplanar-device.tex
% ============================================================================ %
\documentclass[../../../main.tex]{subfiles}
\begin{document}%
% ============================================================================ %
    \subsection{Coplanar Device}%
    \label{sec:chapter-5:moving-forward:coplanar-device}%
    % ======================================================================== %
    A common technique for improving the energy resolution in radiation detectors is the usage of coplanar electrodes.
    Functioning similar to Frisch grids, this semiconductor elctrode structure is designed to accomodate the charge carrier with better transport properties (electrons for \gls{lise}), not relying on the other carrier for signal production \cite{Luke_1994, Hamm_2015}.
    A prototype detector using the interdigitated electrode design was fabricated, shown in \Xreffigure*{coplanar-device-comparison}.
    Each electrode featured \num{5} digits of \SI{200}{\micro\meter} width at a \SI{400}{\micro\meter} spacing and \SI{5}{\milli\meter} length, surrounded by a \SI{1.5}{\milli\meter} guard ring.
    These type of sensors have shown to improve the spectroscopic performance in other semiconductor detectors, increasing the energy peak resolution \cite{He_1998, He_2005, thesis:Hamm_2018}.
% ============================================================================ %
\end{document}%