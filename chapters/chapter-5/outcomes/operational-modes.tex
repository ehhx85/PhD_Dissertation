% ./chapters/chapter-5/outcomes/operational-modes.tex
% ============================================================================ %
\documentclass[../../../main.tex]{subfiles}%
\begin{document}%
% ============================================================================ %
    \subsection{Operational Modes}%
    \label{sec:chapter-5:outcomes:operational-modes}%
    % ======================================================================== %
    One of the most interesting aspects of \gls{lise} is its operational duality, functioning as both semiconductor and scintillator.
    The \SI{34}{\micro\meter} material spatial resolution, as evaluated with the \gls{mtf}, is in fact smaller than the range of \glspl{triton-particle} in the material (\SI{39.4}{\micro\meter}), attributed to the non-zero kinetic energy of the real incident neutron.
    Furthermore, the optical self transparency demonstrated in the \glspl{lise-scint} led a wide range of sample thicknesses to provide a similar spatial resolution, unlike the \gls{scintillator-screen}, which has a spatial resolution that trends proportionally with thickness, optimized near \SI{50}[~]{\micro\meter}.
    % ======================================================================== %
    \par%
    % ======================================================================== %
    A potential clue in understanding the difference between scintillation and biased charge carrier transport mechanisms was revealed in the \gls{lisepix} experiments.
    As a scintillator, \gls{ly} was more intense towards the outer region (concave side) of the response transition edge, presumably corresponding to the outside of the boule, as grown.
    Contrarily, the \gls{lisepix} semiconductor showed more intense charge collection towards the inner region (convex side), corresponding to the internal region of the boule following the same logic.
    As a rule of thumb, the coloration of the material was considered to be a loose indicator of device performance, with low inclusion, yellow to chartreuse sensors providing the maximum performance.
    However, the sensors exhibiting the response transition were an almost uniform yellow color across the edge, nearly indistinguishable by visible observation.
    The transition edge still remains a point of debate in \gls{lise} research and further investigation may offer new insights towards optimizing for either scintillation or semiconductor operation.
% ============================================================================ %
\end{document}%