% ./chapters/chapter-5/outcomes/fabrication.tex
% ============================================================================ %
\documentclass[../../../main.tex]{subfiles}%
\begin{document}%
% ============================================================================ %
    \subsection{Fabrication Techniques}%
    \label{sec:chapter-5:outcomes:fabrication}%
    % ======================================================================== %
    A major outcome of this research was laying the foundation for \gls{lise} device fabrication.
    With the \glspl{lise-scint} offering a simpler approach to imaging, the basic polishing and etching requirements to prepare an optically smooth surface have been documented.
    Handling the relatively delicate samples required special provisions to ensure that each semiconductor fabrication step would not induce residual mechanical stresses in the sensor, degrading operational performance.
    The photolithography and packaging processes exposed the material to a broader range of chemicals and physical stresses, some resulting in poorer device performance or sample destruction.
    Sequentially eliminating the incompatibilities provided a regimen for successfully producing semiconductor detectors without incurring sample harm or loss.
    Most importantly, the material showed compatibility with semiconductor fabrication equipment typically utilized with silicon wafers, including: spin-coating, photolithography, wirebonding and flip-chip bump bonding.
% ============================================================================ %
\end{document}%