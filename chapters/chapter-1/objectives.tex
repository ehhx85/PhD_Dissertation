% ./chapters/chapter-1/objectives.tex
% ============================================================================ %
\documentclass[../../main.tex]{subfiles}
\begin{document}%
% ============================================================================ %
    \section{Aims and Objectives}%
    \label{sec:chapter-1:objectives}%
    % ======================================================================== %
    The overall aim of the project was the following: design and build a functional, \gls{high-res} neutron imager using a direct conversion semiconductor, exhibiting spatial resolution on par or better than current \gls{soa} technologies while operating under similar experimental conditions. 
    This achievement introduces a new category of \gls{solid-state} neutron detector, overcoming the cost and complexity restrictions of previous systems while offering a new platform for expanding the capabilities of neutron detection and imaging sciences.
    % ======================================================================== %
    In evaluation of this promising neutron sensitive material, a set of objective criteria was established to confirm the success or failure of the investigation.
    The preliminary task required validating the neutron sensitivity of the material, demonstrating its basic functionality as a neutron detector.
    A single-channel counter was fabricated to capture neutron spectra, yielding information regarding spectroscopic capabilities, along with experiments using \glspl{alpha-particle} and \glspl{gamma-ray} to study charge collection and the possibility of neutron/\Xmath{\gamma} discrimination, respectively.
    Because the material had not been studied in the context of neutron detection in the past, fabrication techniques were developed to accommodate the material specific requirements for detector construction.
    Expanding on the success of a functional counter, a few-channel pixel detector was designed to serve as a proof of concept verifying the usage of \gls{lise} for neutron imaging.
    The dual-nature of the material was examined through scintillator experiments to evaluate performance metrics between both operational modes and potential energy loss mechanisms, producing inefficiencies within the material. 
    In producing a scintillation response, the performance as a semiconductor may be degraded or sub-optimal in comparison to theoretical expectations.
    Encompassing the aim of this research, a final imaging system was constructed, integrating a \gls{high-res} \gls{asic} to evaluate the material's performance in a direct comparison with other current neutron imaging platforms.
    The anticipated spatial resolution aimed to reach values less than the \SI{55}{\micro\meter} pixel size of the \gls{timepix}.
% ============================================================================ %
\end{document}%