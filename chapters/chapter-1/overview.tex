% ./chapters/chapter-1/overview.tex
% ============================================================================ %
\documentclass[../../main.tex]{subfiles}
\begin{document}%
% ============================================================================ %
    \section{Overview}%
    \label{sec:chapter-1:overview}%
    % ======================================================================== %
    This work documents the design, development and experimental evaluation of a novel neutron detection material.
    A literature review is presented in \Xrefchapter{literature-review}, beginning with an summary of radiation types, their interactions with matter and the fundamentals of neutron detection.
    The subsequent sections survey neutron detection and imaging technologies including scintillator, semiconductor and advanced hybrid systems.
    The material under study has demonstrated both scintillator and semiconductor capabilities, requiring evaluation along both design avenues to determine the extent of the material's capabilities. 
    Prior materials studies were also included as a framework for understanding electronic behavior and fabrication requirements.
    % ======================================================================== %
    The experimental process is described in \Xrefchapter{materials-and-methods}, covering preparation, implementation and data analysis.
    The chapter begins with fabrication procedures, establishing a routine for device construction.
    In progression, the next sections report the designs for the single channel counter, the multi-channel pixel detector, and the \gls{high-res} imager.
    The chapter concludes with an outline of the neutron imaging experiments and the postliminary data processing methods.
    % ======================================================================== %
    \Xrefchapter{results-and-discussion} presents the results of the study, comparing the two modes of operation with other \gls{soa} technologies.
    The performance of the system is qualified and any shortcomings due to experimental methods or design have been described.
    % ======================================================================== %
    Final conclusions are drawn in \Xrefchapter{conclusion}, validating the goals and success of the investigation.
    Suggestions for improvements on experimental methods are offered along with suggestions for moving forward in future studies.
    % ======================================================================== %
    A series of appendices have also been included for more thorough explanation of technical processes and methods.
% ============================================================================ %
\end{document}%