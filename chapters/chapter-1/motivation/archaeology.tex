% ./chapters/chapter-1/motivation/archaeology.tex
% ============================================================================ %
\documentclass[../../../main.tex]{subfiles}%
\begin{document}%
% ============================================================================ %
    \subsubsection{Archaeology}%
    \label{sec:chapter-1:motivation:archaeology}%
    % ======================================================================== %
    Broadening the application of neutron imaging, archaeological studies benefit from the \gls{ndt} techniques offered through radiographic imaging. 
    Because the artifacts being studied may be hundreds or thousands of years old, they are often delicate and irreplaceable, requiring minimal handling.
    Like photography, \gls{x-ray} and more recently neutron imaging offer complementary methods of capturing the details and structure of artifacts without direct taction.  
    \citeauthor*{Deschler-erb_2004} used these methods, at \gls{psi}, to illuminate the structure of Iron Age and Roman era weaponry and garments.
    A combination of metals, organics, animal hides and fabrics, these artifacts exhibited striking contrast between each imaging technique.
    Comparing \gls{x-ray} and neutron images, archaeologists were able to better document the construction methods and metalurgical technology available during the respective periods of human history \cite{Deschler-erb_2004}.
    Similarly, \citeauthor*{Rant_2006} discussed the trend in archaeology to utilize neutron radiography facilities across Europe, conducting their investigations at \gls{jsi} using their \gls{triga}.
    They were able to distinguish the lead shot, gunpowder, and internal mechanisms of a \SI{16}{\thsuper} century iron wheel lock pistol. 
    Furthermore, the team captured organic internal fragments in an iron cross reliquary using neutron radiographs, not visible in the \gls{x-ray} images \cite{Rant_2006}.
    % ======================================================================== %
    \par%
    % ======================================================================== %
    As an engineering design tool, \citeauthor*{Hameed_2009} used neutron imaging to analyze the penetration and distribution of stone strengtheners to be used in historic building restoration.
    In particular, the solid bodies created with \gls{ct-neutron} measured the ingress of actual consolidants into material samples of the time wethered edifice stones, prescribing a more effective concentration of consolidant for each application.
    The measurements were conducted at \gls{ati} with \gls{antares} \cite{Hameed_2009, Schulz_2015}.
    Using the imaging facilities at \gls{psi}, \citeauthor*{Lehmann_2010} investigated bronze Tibetan Buddha statues from the \SIlist{14; 15; 17}{\thsuper} centuries. 
    A comparison between \gls{x-ray} and neutron imaging techniques is shown in \Xreffigure{transmission-imaging-comparison}, as described in \Xrefsection{chapter-2:radiation-detection:radiography}.
    To avoid desecrating the religious and cultural artifacts, radiographic imaging provided an ideal method for exploring the internal components \cite{Lehmann_2010}.
    Similarly, \citeauthor*{Ryzewski_2013} conducted neutron imaging experiments at \gls{cg1d}, imaging a Late Roman bronze lamp and dog statue.
    The team successfully verified the structural integrity of the metal body while also illuminating the existance of internal organic residue from burnt material \cite{Ryzewski_2013}.
% ============================================================================ %
\end{document}%