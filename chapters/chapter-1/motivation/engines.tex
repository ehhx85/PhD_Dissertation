% ./chapters/chapter-1/motivation/engines.tex
% ============================================================================ %
\documentclass[../../../main.tex]{subfiles}%
\begin{document}%
% ============================================================================ %
    \subsubsection{Engines}%
    \label{sec:chapter-1:motivation:engines}%
    % ======================================================================== %
    \citeauthor*{Schillinger_2005} captured a short neutron video, at \gls{psi}, of an operational BMW engine driven by an electric motor at \SI{1000}{\rpm}.
    Covering a \SI{24x24}{\centi\meter} \gls{fov}, the \gls{lif-zns} based scintillator was coupled to a \gls{mcp} detection system, showing the internal dispersion of oil under the piston while it lubricated the connecting rod \cite{Schillinger_2005}.
    Likewise, \citeauthor*{Kardjilov_2005} used a \gls{lif-zns} scintillator at \gls{hzb} to record a model air-craft engine operating at \SI{1110}{\rpm}.
    The large flux at \gls{hzb} (\SI{E9}{\neutronflux}) across the \SI{30x50}{\centi\meter} \gls{fov} provided the high speed imaging capabilities needed for stroboscopic neutron imaging \cite{Kardjilov_2005}.
    Later, \citeauthor*{Butler_2013} demonstrated the edge enhancement effect in a diesel fuel injector.
    The abrupt transition at the air-to-steel interface created increased contrast, potentially useful in future applications for detection of internal cracks and voids, following the development of a suitable \gls{high-res} imager \cite{Butler_2013}.
% ============================================================================ %
\end{document}%