% ./chapters/chapter-1/motivation/alloys.tex
% ============================================================================ %
\documentclass[../../../main.tex]{subfiles}%
\begin{document}%
% ============================================================================ %
    \subsubsection{Alloys}%
    \label{sec:chapter-1:motivation:alloys}%
    % ======================================================================== %
    \citeauthor*{Beyer_2011} investigated hydrogen effusion in austenitic stainless steel, a common engineering material.
    They were able to measure hydrogen concentrations down to \SI{20}{\partpermillion} reducing the detection limit from the previously tested \SI{100}{\partpermillion} \cite{Beyer_2011}.
    In the same year, \citeauthor*{Grosse_2011} reported on hydrogen and oxygen ratios in zirconium alloys, favored by the nuclear industry for their low thermal neutron cross sections and desirable mechanical properties.
    The zirconium based nuclear fuel cladding alloys demonstrated rapid hydrogen uptake on the bare metal with decreased rates in the presence of a surface oxidation layer \cite{Grosse_2011}.
% ============================================================================ %
\end{document}%