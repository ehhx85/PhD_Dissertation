% ./chapters/chapter-1/motivation/hydrogen-and-water.tex
% ============================================================================ %
\documentclass[../../../main.tex]{subfiles}%
\begin{document}%
% ============================================================================ %
    \subsubsection{Hydrogen and Water}%
    \label{sec:chapter-1:motivation:hydrogen-and-water}%
    % ======================================================================== %
    Hydrogen in its most common terrestrial form, water, is used to moderate nuclear reactors because of its sizable interaction cross section.
    In neutron imaging, hydrogenous compounds, including water, create significant contrast through beam attenuation.
    \citeauthor*{Lehmann_2004} tested a series of practical scenarios where hydrogen concentration varied throughout a dynamic system including: moisture in clay, water entrainment in soil, methanol distribution in a fuel cell, and isotopic mixing between light and heavy water \cite{Lehmann_2004}.
    At \gls{cg1d}, \citeauthor*{Kang_2013} imaged an adjustable, water saturated sand column to evaluate the attenuation coefficient of water \cite{Kang_2013}.
    Time resolved imaging offered insight into the drainage of rain water in porous asphalt as studied by \citeauthor*{Poulikakos_2013}.
    The complex geometry of the paving mixture resulted in unusual water retention and associated drying, revealing the hydrological mechanisms in porous asphalt responsible for enhancing motorist safety \cite{Poulikakos_2013}.
    \citeauthor*{Perfect_2014} measured the drying of water saturated sand columns, permeation of water into clay and siliceous bricks, saturation of \glspl{pefc}, and phase changes in heat pipes \cite{Perfect_2014}.
% ============================================================================ %
\end{document}%