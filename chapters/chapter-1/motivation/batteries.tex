% ./chapters/chapter-1/motivation/batteries.tex
% ============================================================================ %
\documentclass[../../../main.tex]{subfiles}%
\begin{document}%
% ============================================================================ %
    \subsubsection{Batteries}%
    \label{sec:chapter-1:motivation:batteries}%
    % ======================================================================== %
    Mobile electronics rely heavily on \glspl{li-ion-battery} to supply the power consuming processors, sensors and displays. 
    The rechargeable batteries operate by creating a voltage potential across the electrolytic cell, containing lithium ions. %with an electrolyte containing lithium ions, creating the voltage potential across the cell.
    In attempts to better understand the electrochemical dynamics in \glspl{li-ion-battery}, \citeauthor*{Butler_2011} used neutron imaging and \gls{ct-neutron} to observe the changes in an operational cell.
    The visible effects of charging and discharging were revealed through changes in \ce{LiC6} concentration, with increased \ce{LiC6} providing larger beam attenuation, darkening the region. 
    While the \gls{ct} was successful, the results left much to be desired and the author made provisions for future experiments \cite{Butler_2011}.    
    In the follow-up experiment, \citeauthor*{Butler_2013a} used neutron imaging to reveal cell degradation in the \gls{li-ion-battery} over multiple charging and discharging cycles.
    They noted an increase in salt-and-pepper noise along with linear structures following the electrode fold structure.
    With a refined set of experiments, the study provided a basis for monitoring the dynamics of an actively performing \gls{li-ion-battery}, suggesting further benefits by increasing the energy and spatial resolutions as well as \gls{tof} techniques \cite{Butler_2013a}.
% ============================================================================ %
\end{document}%