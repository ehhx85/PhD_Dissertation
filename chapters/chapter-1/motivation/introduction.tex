% ./chapters/chapter-1/motivation/introduction.tex
% ============================================================================ %
\documentclass[../../../main.tex]{subfiles}%
\begin{document}%
% ============================================================================ %
    \Xsubsubsection%
    \label{sec:chapter-1:motivation:introduction}%
    % ======================================================================== %
    \citeauthor*{book:Anderson_2009} gives a general overview of neutrons, neutron imaging and \gls{ct-neutron}, neutron generation facilities, various detectors, and applications of the science \cite{book:Anderson_2009}.
    \citeauthor*{Brenizer_2013} reviewed the prior two decades of neutron imaging as a \gls{ndt} and the historical milestones achieved along the broader history of neutron sciences.
    Most importantly, the number of facilities directly supporting neutron imaging has increased, seeking improvements in beam quality (producing colder neutrons at higher fluxes) as well as increasing the timing, spatial and energy resolutions of detection systems \cite{Brenizer_2013}. 
    Summarizing the current state of neutron imaging sciences available globally, \citeauthor*{Anderson_2016} presents an overview of \gls{cans} and their potential for supplementing current reactor or spallation based facilities.
    The small size and lower overhead cost of infrastructure would boost the availability of such neutron beams for a wide variety of research, including neutron imaging.
    A number of facilities and their respective experimental setups are described in detail, looking to decouple neutron based sciences from nuclear reactor facilities and expand their applications \cite{Anderson_2016}.
    In support of neutron imaging sciences, \citeauthor*{Taylor_2017} reported on an electronic neutron generator using a deutirium target for neutron production.
    Specifically designed around neutron imaging applications, this technology opens the door for a broader range of scientists to utilize neutron imaging techniques.
    While the flux is significantly lower than reactor based sources, an electronic neutron source would permit many universities and scientific institutions to conduct experiments without petitioning for low-availability beam time.
    The study successfully generated neutron images using scintillation screens, while performance with a \gls{mcp} was deemed inconclusive and warrants further experimentation \cite{Taylor_2017}.
    % ======================================================================== %
    \par%
    % ======================================================================== %
    \citeauthor*{Lehmann_2017b} presents an up-to-date overview of neutron imaging methodologies and the history of the science.
    While the best feature resolution presently measured is sub \SI{10}{\micro\meter}, most neutron imaging facilities operate in the \SIrange{20}{100}{\micro\meter} resolution range to balance exposure time requirements \cite{Trtik_2015}.
    They also noted the successes in imaging lighter elements with high contrast, unavailable via \gls{x-ray}, as well as imaging thicker metallic samples while avoiding the beam hardening effects seen in \glspl{x-ray} \cite{Lehmann_2017b}. 
    As suggested by \citeauthor*{Lehmann_2017b}, the following studies implement neutron imaging experiments across a variety of material and engineering investigations.
    In developing more sophisticated imaging systems and techniques, the practical applications of neutron detection and imaging will expand to new fields while providing more detailed \gls{ndt} methods. 
    These experiments provide a strong justification for the continued advancement of neutron sources, detectors and \gls{high-res} imaging platforms.
% ============================================================================ %
\end{document}%