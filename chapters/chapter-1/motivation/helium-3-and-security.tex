% ./chapters/chapter-1/motivation/helium-3-and-security.tex
% ============================================================================ %
\documentclass[../../../main.tex]{subfiles}%
\begin{document}%
% ============================================================================ %
    \subsubsection{Helium-3 and National Security}%
    \label{sec:chapter-1:motivation:helium-3-and-security}%
    % ======================================================================== %
    The detection and regulation of \glspl{snm} presents a unique problem for law enforcement, noted \citeauthor*{book:Anderson_2009}, due to the physics of the system.
    Because these materials may be hidden in large shipping containers, masked by other non-regulated materials, \glspl{snm} require significant effort to detect.
    Chemical and radiographic techniques, commonly employed to detect conventional weapons and hazardous materials, are much less effective in application to \gls{snm}.
    New systems are being investigated that utilize neutron imaging, or a complementary approach including \glspl{x-ray}, to create a picture of the hidden materials inside cargo, identifying not only their density, but chemical composition.
    Using car-wash style portal checkpoints for shipping trucks, many of the technologies propose scanning the containers with neutron beams, detecting the secondary radiation produced from the cargo to establish an idea of what materials are present.
    Rapid and accurate verification that \gls{snm} is not present is paramount to a successful system \cite{book:Anderson_2009}.    
    % ======================================================================== %
    \par%
    % ======================================================================== %
    As the official steward of the U.S. nuclear stockpile and materials, the \gls{doe} is also responsible for production of helium-3.
    This particular isotope, while used in nuclear weapons, has played a critical role in radiation detection, particularly as a neutron sensitive isotope.
    Used in a wide variety of national security applications and sciences, \isotope[3]{He} was consumed at a rate of \SI[per-mode={symbol}]{70000}{\liter\per\year}.    
    In the past decade, fears of dwindling \isotope[3]{He} reserves have led the \gls{doe} to seek alternatives where possible, reducing the annual consumption by an estimated \SI{90}{\percent} and mitigating a premature shortage \cite{website:DOE:helium3}.
    Used in large \glspl{rpm}, \isotope[3]{He} filled tubes account for a majority of the national consumption, as noted by \citeauthor*{Kouzes_2015}.
    Both the \gls{doe} and \gls{dhs} have stringent requirements for minimizing false radiation detection alarms, requiring heavily verified technologies.
    % ======================================================================== %
    \par%    
    % ======================================================================== %
    Other direct alternatives, such as \ce{BF3}, pose a health risk and regulations limit the transport and utilization of the hazardous gas.
    Boron-10 lined proportional counters do not exhibit the same neutron detection efficiency as \isotope[3]{He}, while \isotope[6]{Li} loaded glass fibers do not meet the criterion for \gls{gamma-ray} insensitivity \cite{Kouzes_2010, Kouzes_2011}. 
    \citeauthor*{Kouzes_2015} revisited the issue in a follow-up report, documenting the continuing trend of \isotope[3]{He} consumption reduction and new developments in wide area neutron detection for national security.
    In this article, he expands the possibilities to scintillators and semiconductors, having improved in overall efficiency while becoming more available through improvements in fabrication and processing.
    He concludes that as of \citeyear*{Kouzes_2015}, ``{\ldots there appears to be no existing alternative that combines all the capabilities of \isotope[3]{He} for neutron detection efficiency, gamma-neutron separation, commercial availability, and robustness for deployment}" \cite{Kouzes_2015}.
% ============================================================================ %
\end{document}%