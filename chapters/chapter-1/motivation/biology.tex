% ./chapters/chapter-1/motivation/biology.tex
% ============================================================================ %
\documentclass[../../../main.tex]{subfiles}%
\begin{document}%
% ============================================================================ %
    \subsubsection{Biology}%
    \label{sec:chapter-1:motivation:biology}%
    % ======================================================================== %
    Offering an \textit{in situ} view of active biology, neutron imaging has been implemented to study the growth of plants.
    Because living organisms are comprised mostly of water, neutron imaging provides a means of capturing the uptake and flow of hydrogenous compounds within the living system.
    \citeauthor*{book:Anderson_2009} investigated the internal morphological development of plants grown in a thin aluminum planter.
    Imaging a soybean plan root after \SIlist[list-units = repeat]{8;15}{\day} showed extensive root penetration into the soil, a process that would have otherwise been interrupted by invasive techniques.
    Furthermore, the reuse of the original, unadulterated sample provided significant benefit over invasive measurements when studying root response to variations in soil composition.
    Neutron images of the same root system reduced systematic errors, showing progression of the root profile towards water-rich regions of soil, impossible to duplicate between samples due to the semi-random, dendritic growth of plant root structures. 
    Similarly, imaging Japanese tree trunk cross sections illustrated the differing mechanisms of water uptake between species.
    While some trees maintained a significant portion of moisture in the outer rings or sapwood, others presereved their moisture in the core or heartwood of the tree, creating difficulties in drying the lumber and future warping issues when used as a structural material \cite{book:Anderson_2009}.
% ============================================================================ %
\end{document}%