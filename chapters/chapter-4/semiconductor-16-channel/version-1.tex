% ./chapters/chapter-4/semiconductor-16-channel/version-1.tex
% ============================================================================ %
\documentclass[../../../main.tex]{subfiles}%
\begin{document}%
% ============================================================================ %
    \subsection{LISe Pixel Detector v1}%
    \label{sec:chapter-4:semiconductor-16-channel:version-1}%
    % ======================================================================== %
    In the first \gls{lise16} experiments, a large portion of the allotted experiment time consisted of system setup and calibration.
    At the time, the \gls{caen} acquisition electronics were brand new, switching from the previous analog electronics to a fully digital pulse processing system.
    Because the flux at \gls{cg1d} was significantly higher than the \gls{pube} source at \gls{university-tennessee}, the threshold settings were completely different.    
    After manually calibrating the set points to correctly register spectra for each of the 16 channels, the automated acquisition system was ready.
    As mentioned, a custom \gls{labview} application controlled both the acquisition timing as well as the linear drives for the sample positioning stage.
    Interfacing with the in-house motor controller required some on-site changes to the control software, but the final result was a fully automated acquisition system for neutron imaging with the \gls{super-sampling} technique.
    The scan consisted of \SI{200}{\micro\meter} steps across a \num{101x9} array, generating a total of \num{909} acquisition locations, covering an area of \SI{20x1.6}{\milli\meter}.
    With each exposure lasting \SI{10}{\second} and a generous \SI{5}{\second} slew window, the total experiment run time lasted right at \SI{4}{\hour} \cite{Herrera_2016}.
    % ======================================================================== %
    \par%
    % ======================================================================== %
    The \gls{lise16} scanned the stainless steel bolt, shown in \Xreffigure*{bolt-comparison}, receiving an estimated neutron fluence of \Xmath{\sim\SI{4E11}{\neutronfluence}} across the duration of experimentation.
    Artificial color has been added to the image to contrast the open beam region (green) from the bolt attenuated region (yellow).
    The two regions are separated with bands of black to enhance the overall contrast and show clear transition between the open beam and object image.
    The semiconductor data from the \gls{lise16} is outlined in red and superimposed on the comparison image captured using the \gls{scintillator-screen}.
    Two data points have been marked in red to capture the edge profile of the bolt and the center of the bolt where attenuation is greatest. 
    Approximately half-way through the automated experiment, the in-house motor control system power failed and had to be manually reset upon return to the line, leaving the system exposed to the open beam for \SI{10.5}{\hour} before subsequent acquisition resumed.
    The resulting reduction in signal is seen in the intensity profile overlaid on the bottom of the bolt.
    The roughly \SI{20}{\percent} decrease in efficiency was attributed to a combination of positive bias on the beam facing contact, where the interaction rate was highest, and the poor charge carrier transport properties for holes in this material.
    Because the electrons provided the majority of the detection pulse, damage induced on the beam side during extended operation magnified performance degradation, specifically interfering with electron collection \cite{Herrera_2016}.
% ============================================================================ %
\end{document}%