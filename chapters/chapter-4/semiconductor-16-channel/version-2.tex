% ./chapters/chapter-4/semiconductor-16-channel/version-2.tex
% ============================================================================ %
\documentclass[../../../main.tex]{subfiles}%
\begin{document}%
% ============================================================================ %
    \subsection{LISe Pixel Detector v2}%
    \label{sec:chapter-4:semiconductor-16-channel:version-2}%
    % ======================================================================== %
    Having confirmed the functionality of the readout electronics and worked through debugging issues with the acquisition and control software, the primary focus in the second system iteration was the sensor.
    After the first set of experiments, the sensor \gls{pcb} was removed from the imaging system and tested to determine the source of channel performance issues.
    Using a combination of \gls{iv} curve tests to determine electrical response and visual inspection under the \gls{lm}, the sensor showed signs of electronic decoupling.
    With an improved fabrication process, the sample was polished, stripping all previous metallization, and completely repackaged with a new sensor \gls{pcb} to facilitate wirebonding.
    A combination of properly deposited contact metal and properly formed wirebonds ensured the mechanical stability of the electrical connections.
    % ======================================================================== %
    \par%
    % ======================================================================== %
    The \gls{super-sampling} scan used \SI{125}{\micro\meter} steps across a \num{33x33} array generating a total of \num{1089} acquisition positions.
    Each position consisted of a \SI{10}{\second} exposure, and an additional \SI{2}{\second} window for stage position adjustment, running for a total experiment time of \SI{3}{\hour}:\SI{37}{\minute}.
    Covering a \SI{4x4}{\milli\meter} sweep area, the scan captured the two outermost radial marking rings, shown in \Xreffigure*{psi-mask-lise-semiconductor}.
    The reconstructed neutron image was able to resolve line pairs at the \SIlist{500;400;300}{\micro\meter} line pair spacings.
    At the \SI{300}{\micro\meter} limit, the technique provided a spatial resolution roughly \SI{55}{\percent} of the pixel size, corroborating the efficacy of the \gls{super-sampling} technique in achieving sub-pixel size resolution.
    Furthermore, in this experiment, the rejuvenated sensor showed minimal performance degradation over the exposure period, with each channel consistently performing through the duration of the experiment.
    It is hypothesized the performance degradation in the first iteration was incurred as a result of electrical decoupling from weak wirebonds, or delamination of pixels.
    While all channels were able to generate an image signal, a few showed poorer performance than the others (see \Xreffigure{lise-16ch-mosaic}), possibly due to material defects, fabrication or a combination of the two.
    At the neutron energies seen in \gls{cg1d}, the anticipated efficiency should have approached \SI{59}{\percent}, however the inefficiencies seen across the devices channels only measured between \SIrange{35}{55}{\percent}.
    The performance deviation from the idealized value was again attributed to the poor hole charge collection with the sensor bulk, present across all channels \cite{Herrera_2016}.
% ============================================================================ %
\end{document}%