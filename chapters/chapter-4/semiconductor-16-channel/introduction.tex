% ./chapters/chapter-4/semiconductor-16-channel/introduction.tex
% ============================================================================ %
\documentclass[../../../main.tex]{subfiles}%
\begin{document}%
% ============================================================================ %
    \Xsubsection%
    % ======================================================================== %
    In this semiconductor experiment, the \gls{lise16} validated the hypothesis that \gls{lise} could be used as a pixelated neutron detector.
    Using a limited number of channels helped to reduce complexity of the build, focusing on the fabrication of the sensor and ensuring the basic system could be constructed and operated in a high flux neutron environment across a practical exposure timeline.
    The automated acquisition sequence provided the groundwork for a control intensive \gls{super-sampling} scan, as well as other position dependent exposure experiments within the research group.
    The lessons learned in the two iterations of the \gls{lise16} were directly applied in the fabrication process of the \gls{lisepix} as its \gls{high-res} successor.
% ============================================================================ %
\end{document}%