% ./chapters/chapter-4/scintillator-array/computed-tomography.tex
% ============================================================================ %
\documentclass[../../../main.tex]{subfiles}%
\begin{document}%
% ============================================================================ %
    \subsection{Computed Tomography}%
    \label{sec:chapter-4:scintillator-array:computed-tomography}%
    % ======================================================================== %
    \Xsupplementfile{ct-scan-raw}%
    \Xsupplementfile{ct-scan-cleaned}%
    \Xsupplementfile{ct-scan-filtered}%
    % ======================================================================== %
    One of the most bold objectives in the \gls{lise} neutron imaging experiments was producing a \gls{ct-neutron} scan.
    Using samples \gls{lise-5a}, \gls{lise-5b} and \gls{lise-07}, a large enough area was available to capture the bulk of a \textit{halyomorpha halys} specimen.
    The scan used the rotational stage at \gls{cg1d}, sweeping \SI{183}{\degree} at \SI{0.25}{\degree} increments for a total of \num{733} projections, each across a \SI{60}{\second} exposure window.
    A scan was performed with each backing, lasting around \SI{13}{\hour} a piece.
    For comparison, a \gls{ct-neutron} was also conducted with the \gls{scintillator-screen} across the same intervals at \SI{15}{\second} exposures.
    % ======================================================================== %
    \par%
    % ======================================================================== %
    A sample of the raw \gls{lise-scint} scan is shown in \Xreffigure*{lise-scintillator-ct-insect-raw}, with the full scan available for download (\Xrefsupplement{ct-scan-raw}).
    The raw image exhibited significant salt-and-pepper (speckling) noise, as well as shadowing and overexposure along the crystal edges.
    The reflective backing also highlighted the split between \gls{lise-5a} and \gls{lise-5b}, along with a surface chip in \gls{lise-07}.
    Initial attempts to use this data with the \gls{octopus} imaging software produced abnormal volume artifacts, encapsulating the insect volume inside a shell corresponding to the edge of the crystals.
    To overcome the deficit in imaging area, the image data was cleaned, removing the regions around the edges of the sensors and the dark field region (see \Xrefsupplement{ct-scan-cleaned}).
    After cropping the outlying region and replacing their values with a saturated open beam, a despeckling filter was applied, producing the sample projection shown in \Xreffigure*{lise-scintillator-ct-insect-filtered}.
    The final cleaned and filtered data was sufficient to run the \gls{ct} analysis in \gls{octopus} (see \Xrefsupplement{ct-scan-filtered}).
    A comparison between the \gls{scintillator-screen} and the \gls{lise-scint} array is shown in \Xreffigure*{scintillator-insect-comparison}.
    The legs and lower abdominal features are visible in both reconstructed volumes.
    The internal structure of the shield shaped body can also be seen, as well as the rostrum (probing mouth part) in the \gls{lise-scint} reconstruction.
% ============================================================================ %
\end{document}%