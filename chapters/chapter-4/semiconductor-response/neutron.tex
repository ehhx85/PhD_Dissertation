% ./chapters/chapter-4/semiconductor-response/neutron.tex
% ============================================================================ %
\documentclass[../../../main.tex]{subfiles}%
\begin{document}%
% ============================================================================ %
    \Xsubsection%    
    % ======================================================================== %
    \subsubsection*{Neutron Detection}%
    \label{sec:chapter-4:semiconductor-response:neutron}%
    % ======================================================================== %
    The neutron response for \gls{lise} was routinely tested using the \gls{pube} source at \gls{university-tennessee} before imaging experiments at \gls{hfir}.
    The moderated source provided a characteristic thermal/cold neutron spectrum, producing a broad continuum response over the background (dark field). 
    At \gls{cg1d} the \gls{lise16} produced a single, broad energy peak with an extended tail, shown in \Xreffigure*{lise-response-spectra-16ch} (half the digitizer range shown) \cite{Herrera_2016}.
    A digital threshold was implemented on the digitizer to eliminate signal pile-up from low energy noise and pulses.
    Over the course of the experiment, the pulse height spectrum remained relatively consistent.
% ============================================================================ %
\end{document}%