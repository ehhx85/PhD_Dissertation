% ./chapters/chapter-4/semiconductor-response/gamma-ray.tex
% ============================================================================ %
\documentclass[../../../main.tex]{subfiles}%
\begin{document}%
% ============================================================================ %
    \Xsubsection%    
    % ======================================================================== %
    \subsubsection*{Gamma-ray Sensitivity}%
    \label{sec:chapter-4:semiconductor-response:gamma-ray}%
    % ======================================================================== %
    A major feature requirement of a robust neutron detector is the ability to discriminate between neutrons and \glspl{gamma-ray}.
    \citeauthor*{Kouzes_2011} specified a set of criteria for evaluating the effects of intrinsic \gls{gamma-ray} sensitivity, with a \SI{10}{\percent} limit imposed on the deviation between the intrinsic efficiency for neutrons and mixed neutron/\gls{gamma-ray} fields, necessary in national security applications \cite{Kouzes_2011}.
    Applied across a wide range of liquid and plastic scintillators, neutron/\gls{gamma-ray} discrimination has been achieved through a few different techniques.
    \citeauthor*{McBeth_1971} described a \gls{zc} \gls{psd} technique for distinguishing particle types based on their scintillation profiles.
    Pulse shape will vary between different types of radiation, corresponding to the decay time characteristic for each particle or photon.
    Using a series of analog \gls{nim} modules, the timing is converted to an amplitude for processing using a \gls{mca} \cite{McBeth_1971}.    
    Charge comparison is another method for discrimination, integrating the slow and fast components of the scintillation pulse to determine the associated particle type \cite{Szczesmak_2014}.
    % ======================================================================== %
    \par%
    % ======================================================================== %
    In neutron imaging, such techniques are not practical, instead favoring a material with low \gls{gamma-ray} detection efficiency, or the capability of direct energy discrimination. 
    Using \gls{mcnp} simulations, it was shown that \gls{lise} demonstrated poor \gls{gamma-ray} detection efficiency at thicknesses used for the semiconductor systems.
    For samples less than \SI{1}{\milli\meter} thickness, the \gls{gamma-ray} detection efficiency was less than \SI{5}{\percent} for energies \SI{100}[>]{\kilo\electronvolt}.
    This feature helps to explain why the higher energy \glspl{gamma-ray} produced from the various indium reactions did not contribute significant noise in any of the imaging experiments in either mode \cite{Lukosi_2016}.
    Furthermore, under full charge collection a neutron interaction produces energetic particles an order of magnitude larger than the \SI{100}{\kilo\electronvolt} limit: \SI{2.73}{\mega\electronvolt} for the \gls{triton-particle}, \SI{2.05}{\mega\electronvolt} for the \gls{alpha-particle} and \SI{4.78}{\mega\electronvolt} for the full event deposition.
% ============================================================================ %
\end{document}%