% ./chapters/chapter-4/semiconductor-timepix/introduction.tex
% ============================================================================ %
\documentclass[../../../main.tex]{subfiles}%
\begin{document}%
% ============================================================================ %
    \Xsubsection%
    % ======================================================================== %
    The preliminary results of the \gls{lisepix} system provided insight into the capabilities of the technology and proved the application of \gls{lise} as a \gls{high-res} semiconductor detector.
    At the time of fabrication, the current growth system did not provide for a single sensor of adequate area to cover the entire \gls{timepix}, limiting the number of functional channels on the \gls{asic}.
    Furthermore, the prototype \gls{lisepix} sensor received a contact pattern treatment across the entire surface to maximize usable area, omitting a guard ring in this design.
    Typically, a guard ring is implemented to stabilize the internal electrical field, however, without a sensor large enough to cover the entire \gls{asic}, a matching sensor guard ring would have been rendered ineffective \cite{book:Rossi_2006}.
    The device also exhibited channel and regional defects, likely due to imperfects in the novel fabrication process.
    Most notably, a large area of dead pixels occurs in the upper right corner of the following neutron radiographs.
    A long streak of dead pixels is also visible along the horizontal direction \cite{Herrera_2018}.
    The bottom left corner of the active area shows a reduction in signal output, a phenomenon similar to the response variation seen in the scintillator sensors and still under investigation \cite{Lukosi_2016a}.
    The reasoning behind these defects and their effects on the output response will be described later in this section.
% ============================================================================ %
\end{document}%