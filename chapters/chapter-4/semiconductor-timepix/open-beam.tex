% ./chapters/chapter-4/semiconductor-timepix/open-beam.tex
% ============================================================================ %
\documentclass[../../../main.tex]{subfiles}%
\begin{document}%
% ============================================================================ %
    \subsection{Open Beam}%
    \label{sec:chapter-4:semiconductor-timepix:open-beam}%
    % ======================================================================== %    
    As with previous experiments, the \gls{lisepix} system was tested at \gls{hfir} on \gls{cg1d}. 
    With prototype systems, the first experiment captures the open beam image, verifying neutron response, pixel electronic operation and exposure window timing.    
    The neutron source at \gls{hfir} provided a neutron fluence of \SI{4.5E6}{\neutronfluence} across the sensor area, or \SI{300}{\neutronrate} across each \SI{55 x 55}{\micro\meter} pixel.    
    Shown in \Xreffigure*{timepix-open-beam}, the imager responded to the neutron beam across the active sensor area, with a diminished response in the lower left corner.
    The acquisition settings used were \SI{50}{\frames} at \SI{1}{\second} exposures with the sensor biased at \SI{300}[+]{\volt}.
    Configured with the \SI{16}{\milli\meter} beam aperture to maximize neutron flux, \Xmath{L/D=375} for this setup \cite{Herrera_2018}.
    The smooth curvature and abrupt transition in contrast suggested a material or phase gradient, similar to \Xreffigure{lise-scintillator-array}, instead of a mechanical defect.
    A key difference is the direction of contrast transition, with the inner concave region creating a more intense semiconductor signal and the outer, convex region exhibiting a weaker electronic pulse.
    This behavior functions opposite of the same phenomena observed in the \glspl{lise-scint}, suggesting the scintillation and semiconductor mechanisms are complementary constituents of the total neutron response.
    % ======================================================================== %    
    \par%
    % ======================================================================== %    
    The variation between bright and dark response regions was characterized to determine relative intensity difference and intensity homogeneity, shown in \Xreffigure{timepix-open-beam:b}.
    Each \gls{roi} consists of approximately equivalent number of pixels, with the bright \gls{roi} concentrated around the same area as in the \gls{pube} analysis, demonstrating the largest contrast and smoothest response.
    The bright \gls{roi} yielded a mean intensity of \num{0.0086(8)}, while the dark \gls{roi} returned a more modest \num{0.00520(178)}, equating to a \SI{40}{\percent} intensity reduction across the sharp gradient.
    The open beam response image showed less deviation in intensity across the sensor, as expected, due to the energy and spatial uniformity of the neutron beam.
    Furthermore, the more luminous beam and lower neutron energies highlighted the true functional edge of the sensor area, not visible in the \gls{pube} experiments.
    The gradual drop off and rounded corners at the periphery are likely attributed to the omission of a guard ring as previously noted \cite{Herrera_2018}.
% ============================================================================ %
\end{document}%