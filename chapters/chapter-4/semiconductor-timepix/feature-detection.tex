% ./chapters/chapter-4/semiconductor-timepix/feature-detection.tex
% ============================================================================ %
\documentclass[../../../main.tex]{subfiles}%
\begin{document}%
% ============================================================================ %
    \subsection{Feature Detection}%
    \label{sec:chapter-4:semiconductor-timepix:feature-detection}%
    % ======================================================================== %    
    Along with resolution measurements, object identification experiments offered a practical example of the imaging systems capabilities for neutron imaging in other scientific areas of study.
    While a biological specimen, an adult red paper wasp (\textit{Polistes carolina}), was prepared for imaging, the specimen did not exhibit the same depth of contrast as seen in the brown marmorated stink bug (\textit{Halyomorpha halys}) specimen, and was omitted from further testing.
    A series of \gls{3d} printed imaging targets were investigated, providing simple geometric features to qualify the \gls{lisepix} system.
    The \gls{power-t} target, shown in \Xreffigure*{timepix-power-t}, provided the best preliminary performance and was selected at the beam line for extended exposures.
    The \SI{1}{\centi\meter\cubed} target block provided a thick medium for neutron attenuation, with the \gls{power-t} inscription permitting an unobstructed view of the open beam.
    The logo, following the university template, measured approximately \SI{6}{\milli\meter} tall, and less than \SI{1}{\milli\meter} at the most narrow points (see \Xrefsection{chapter-3:beamline-testing:imaging-targets:additive-manufacturing}).
    Two different image sequences were acquired, each for \SI{50}{\frames} at \SI{10}{\second}, to test the effects of signal threshold adjustment.
    The longer acquisition time was used to account for the decreases in count rate due to heavy attenuation as well as signal rejection.
    The lower threshold setting (\SI{4}[+]{\volt}) provided higher contrast and accommodated shorter exposure windows, thus being used for the bulk of the experiments. 
    The higher threshold setting (\SI{24}[+]{\volt}) significantly enhanced overall resolution, even cleaning the areas surrounding dead pixels, at the cost of contrast and timing.
    Furthermore, the higher threshold had significant reduction in signal along the dark \gls{roi} in the lower left corner, suggesting the neutron interaction rate may have been equivalent, while the charge carrier transport and pulse production suffered.
    At the higher threshold, the geometric structure in the \gls{3d} printed part is visible, with minor rounding along the sharp outer edges. 
% ============================================================================ %
\end{document}%