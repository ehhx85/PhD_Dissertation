% ./chapters/chapter-4/scintillator-single/open-beam.tex
% ============================================================================ %
\documentclass[../../../main.tex]{subfiles}%
\begin{document}%
% ============================================================================ %
    \subsection{Open Beam}%
    \label{sec:chapter-4:scintillator-single:open-beam}%
    % ======================================================================== %
    In early testing, the \gls{lise} samples were found to demonstrate scintillation properties, generating light in response to a neutron field.
    While testing the early \gls{lise} single channel counter at \gls{cg1d}, a set of complementary scintillator experiments were also conducted using a single \gls{lise} sample (\gls{lise-1}).
    The open beam image is shown in \Xreffigure*{lise-scintillator-light-output} captured using \SI{20}{\frames} at a \SI{120}{\second} exposure time.
    This image immediately verified the scintillator potential for \gls{lise}, with the open beam showing distinct contrast from the dark field baseline, and a clear outline of the sensor edge.
    While the \gls{dark-box} at \gls{cg1d} was specifically engineered for the \gls{scintillator-screen}, a direct replacement with the \gls{lise-scint} produced positive results.
    At only \SI{650}{\micro\meter} thick, the sensor boasted a neutron absorption efficiency over \SI{70}{\percent}, measured with the \gls{scintillator-screen} \cite{Lukosi_2016a}.
% ============================================================================ %
\end{document}%