% ./chapters/chapter-4/scintillator-single/power-t.tex
% ============================================================================ %
\documentclass[../../../main.tex]{subfiles}%
\begin{document}%
% ============================================================================ %
    \subsection{Power-T Target}%
    \label{sec:chapter-4:scintillator-single:power-t}%
    % ======================================================================== %
    The \gls{fdm} \gls{power-t} imaging target was the first object to be imaged with the \gls{lise-scint}.
    With a simple geometry, large feature size, and relatively uniform density, the object served as a preliminary test for comparing the response of the \gls{scintillator-screen} and \textit{LISe-1}, shown in \Xreffigure*{power-t-scintillator-comparison}.
    The image captured with the \gls{scintillator-screen} used \SI{5}{\frames} at \SI{60}{\second} exposures at a single location, using only a small fraction of the much larger \gls{fov}.
    Because \gls{lise-1} had an area slightly smaller than the \gls{power-t} dimensions, a simple mosaic stitching experiment used \num{3} different positions, each capturing \SI{2}{\frames} at \SI{60}{\second} exposures.
    The outline of each open beam section from \gls{lise-1} is visible in \Xreffigure{power-t-scintillator-comparison:b}, cropped to remove the inhomogeneous light output near the edges of the sensor.
    While the image clarity is comparable at equivalent exposure windows, the \gls{scintillator-screen} produced a higher \gls{ly} and thus more contrast, creating a darker \gls{power-t} (see \Xreffigure{power-t-scintillator-comparison:a}), when the two images are normalized to the same scale.
    In both images, the internal voiding is visible as a more transparent region in the center of the logo \cite{Lukosi_2016a}.
    When parts are \gls{3d} printed using a \gls{fdm} process, the outer profile is traced in plastic filament, creating a continuous surface perpendicular to the build plate (the page in this case).
    To conserve material and minimize sagging of the part as layers build and the plastic cools, the internal area is only partially filled using a varying web of plastic filament.
    This web effectively creates a porous, internal structure (visible in larger objects), producing less neutron attenuation and a more transparent image.
    The outline of the \gls{power-t} corresponds to the solid filament outline, effectively measuring the thickness of the part's sidewall, with only a small attenuating region in the lower leg of the object, likely due to a cross-over point along the nozzle tracing path.
% ============================================================================ %
\end{document}%