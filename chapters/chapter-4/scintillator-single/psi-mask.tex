% ./chapters/chapter-4/scintillator-single/psi-mask.tex
% ============================================================================ %
\documentclass[../../../main.tex]{subfiles}%
\begin{document}%
% ============================================================================ %
    \subsection{PSI Siemens Star}%
    \label{sec:chapter-4:scintillator-single:psi-mask}%
    % ======================================================================== %
    As a final performance comparison between the \gls{scintillator-screen} and the single \gls{lise-scint}, the \gls{siemens-star} was imaged with both systems.
    Shown in \Xreffigure*{scintillator-psi-mask-comparison}, the \gls{lise-scint} images are superimposed on the larger \gls{scintillator-screen} image, offering a direct visual comparison.
    The \gls{scintillator-screen} image (blue) consisted of \SI{5}{\frames} counted for \SI{60}{\second} exposures, a standard acquisition for resolution estimation.
    In the lower, left-hand corner of the image, the \gls{lise-scint} mosaic is outlined in red, covering approximately \num{1/4} of the mask using \num{12} different positions, with \SI{5}{\frames} each at \SI{60}{\second} exposures.
    Towards the center of the mask, a single position (yellow) was captured with \SI{5}{\frames} at an extended \SI{300}{\second} exposure, providing the necessary contrast to resolve the fine line pairs.
    The difference between the center image and the large mosaic can be seen in the bottom left corner of the yellow outline, where the radial marker is much darker than the rest of the \gls{lise-scint}.
    Between the \SIlist{200;100}{\micro\meter} radial markers, the line pairs are well distinguished, with some line pairs still resolvable just inside the \SI{100}{\micro\meter} radial marker.
    A concurrent experiment using a separate \gls{lise-scint} at \gls{psi} produced a similar spatial resolution at their neutron imaging line.
    Visual inspection indicates an equivalent spatial resolution limit between the \gls{scintillator-screen} and the \gls{lise-scint} for this experiment, coupled to the same optics and acquisition system \cite{Lukosi_2016a}.
% ============================================================================ %
\end{document}%