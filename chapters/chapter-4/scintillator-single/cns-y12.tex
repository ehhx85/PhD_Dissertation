% ./chapters/chapter-4/scintillator-single/cns-y12.tex
% ============================================================================ %
\documentclass[../../../main.tex]{subfiles}%
\begin{document}%
% ============================================================================ %
    \subsection{CNS-Y12 Target}%
    \label{sec:chapter-4:scintillator-single:cns-y12}%
    % ======================================================================== %
    The \gls{cns-y12} target was manufactured using the same \gls{fdm} process, however the part is both thicker and covers more imaging area (see \Xreffigure{cns-imaging-target}).
    The unique shape, internal voiding, and hollowed regions between the lettering created a more complex set of features, seen in \Xreffigure*{cns-y12-scintillator-comparison}.
    Again using the \gls{scintillator-screen} as a baseline, the neutron image averaged \SI{5}{\frames} at \SI{60}{\second} exposures.
    The larger target required \num{15} sampling positions with \gls{lise-1}, each with \SI{2}{\frames} at \SI{60}{\second} exposures.
    This experiment, along with the \gls{power-t}, further confirmed the mosaic stitching technique for imaging large areas using a smaller prototype sensor.
    Both scintillators reveal the internal voiding from the filament webbing, clearly visible in the number `12', while also showing the letter `C' was almost entirely solid plastic based on the circular path required to create the shape.
    The more gradual color change in the \gls{lise-scint} image (\Xreffigure{cns-y12-scintillator-comparison:b}) can again be attributed to lower \gls{ly} and contrast.
    The contrast is more prominent in the center of the part, where multiple position data overlaps, improving the data average.
% ============================================================================ %
\end{document}%