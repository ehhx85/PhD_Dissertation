% ./tables/default.tex
% ============================================================================ %
\documentclass[../main.tex]{subfiles}%
\begin{document}%
% ============================================================================ %
    \Xtable%    
    \begin{table}[p]%
    % ======================================================================== %
        \caption%
            [\Glstext{pcb} wet etching process]%
            {\Gls{pcb} wet etching process}%
        \label{tbl:pcb-etch-process}%
        % ==================================================================== %
        \begin{tabu}{ c p{5.00in} }%
        % ==================================================================== %
            \toprule%            
            \centering\textbf{Step} &%
            \centering\textbf{Action}\\%
            \toprule%
            % ================================================================ %
            1 & \textit{Add solution}\\%
            $(a)$ & Measure \SI{40}{\milli\liter} of \SI{3}{\percent} \gls{h2o2} in graduated cylinder.\\%
            $(b)$ & Pour contents of graduated cylinder into large Pyrex dish.\\%
            $(c)$ & Rinse graduated cylinder with \gls{di-water} and dispose in acid waste.\\%
            \midrule%
            % ================================================================ %
            2 & \textit{Add acid}\\%
            $(a)$ & Measure \SI{20}{\milli\liter} of \SI{37}{\percent} \gls{hcl} in graduated cylinder.\\%
            $(b)$ & Pour contents of graduated cylinder into large Pyrex dish.\\%
            $(c)$ & Rinse graduated cylinder with \gls{di-water} and dispose rinse solution in acid waste.\\%
            \midrule%
            % ================================================================ %
            3 & \textit{Charge solution}\\%
            $(a)$ & Prepare a section of thin \ce{Cu} foil of roughly \SI{1.0}{\centi\meter\squared} area.\\%
            $(b)$ & Add \ce{Cu} test strip to solution to charge etchant with \ce{Cu} ions.\\%
            $(c)$ & Monitor the dissolution rate and add \SI{35}{\percent} \gls{h2o2} with dropper to adjust rate.\\%
            \midrule%
            % ================================================================ %
            4 & Repeat \textbf{Step 3} until the dissolution rate is controllable.\\%
            \midrule%
            % ================================================================ %
            5 & Dip \gls{pcb} in solution using ceramic tweezers.\\%
            \midrule%
            % ================================================================ %
            6 & Lightly agitate board until edge of traces are clearly distinguished.\\%
            \midrule%
            % ================================================================ %
            7 & Rinse \gls{pcb} with \gls{di-water} until board is clean and etching has stopped.\\%
            \midrule%
            % ================================================================ %
            8 & Dispose of etching solution in acid waste bottle. (Alternatively, etching solution may be stored in clean, labeled bottle for later use.)\\%
            % ================================================================ %
            \bottomrule%            
        % ==================================================================== %
        \end{tabu}%
    % ======================================================================== %
    \end{table}%
% ============================================================================ %
\end{document}%