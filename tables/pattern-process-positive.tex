% ./tables/pattern-process-positive.tex
% ============================================================================ %
\documentclass[../main.tex]{subfiles}%
\begin{document}
% ============================================================================ %
    \Xtable%    
    \begin{table}[p]%
    % ======================================================================== %
        \caption%
            [Photolithography with positive photoresist for \glstext{pcb} manufacturing]%
            {Photolithography with positive photoresist for \gls{pcb} manufacturing}%
        \label{tbl:pattern-process-positive}%
        % ==================================================================== %
        \begin{tabu}{ c p{5.50in} }%
        % ==================================================================== %
            \toprule%            
            % ================================================================ %
            \centering\textbf{Step} &%
            \centering\textbf{Action}\\%
            % ================================================================ %
            \toprule%            
            % ================================================================ %
            \Xsubreffigure{photolithography-positive-pcb:a}%
            &%
            Clean \gls{pcb} surface and remove and protective layer with solvents as necessary.%
            \\%
            \midrule%
            % ================================================================ %
            \Xsubreffigure{photolithography-positive-pcb:b}%
            &%
            Generously apply \gls{pr1-1000a} across \gls{pcb} and run positive spin coating recipe.
            Follow with soft bake on hotplate at \SI{120}{\celsius} for \SI{120}{\second}.%
            \\%
            \midrule%
            % ================================================================ %
            \Xsubreffigure{photolithography-positive-pcb:c}%
            &%
            Install photomask in \gls{ma6}. Load \gls{pcb} and align photomask.%
            \\%
            \midrule%
            % ================================================================ % 
            \Xsubreffigure{photolithography-positive-pcb:d}%
            &%
            Expose resist for \SI{4}{\second} using hard contact setting on \gls{ma6}.%
            \\%
            \midrule%
            % ================================================================ %
            \Xsubreffigure{photolithography-positive-pcb:e}%
            &%
            Visually inspect pattern for full exposure at edges of \gls{pcb} and through holes.%
            \\%
            \midrule%
            % ================================================================ %
            \Xsubreffigure{photolithography-positive-pcb:f}%
            &%
            Dispense \gls{rd6} across surface of \gls{pcb} until transferred pattern is well defined.
            Rinse surface with \gls{di-water} to clean remaining \gls{rd6}.
            Dry patterned surface using compressed nitrogen gun, driving all fluid from center outward.
            Optional bake at \SI{145}{\celsius} for \SI{5}{\minute} to anneal resist.%
            \\%
            \midrule%
            % ================================================================ %
            \Xsubreffigure{photolithography-positive-pcb:g}%
            &%
            Etch copper layer as described in \Xrefappendix{general-chemistry:pcb-etching}.%
            \\%
            \midrule%
            % ================================================================ %
            \Xsubreffigure{photolithography-positive-pcb:h}%
            &%
            Dissolve remaining resist using \gls{acetone} to expose the patterned metal layer.
            Remove surface residue with \gls{ipa}.%
            \\%
            \bottomrule%
        % ==================================================================== %
        \end{tabu}%
    % ======================================================================== %
    \end{table}%
% ============================================================================ %
\end{document}%