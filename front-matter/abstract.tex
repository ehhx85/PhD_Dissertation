% ./front-matter/abstract.tex
% ============================================================================ %
\documentclass[../main.tex]{subfiles}%
\begin{document}%
% ============================================================================ %
    \chapter*{Abstract}%
    \label{ch:abstract}%    
    % ======================================================================== %
    This work demonstrates the neutron sensitivity of single crystal \glsname{lise} (\glschemical{lise} or \glstext{lise} \AbstractPlainText{lithium indium diselenide}).
    The study aimed to design and characterize a neutron imaging system capable of achieving spatial resolution less than \SI{50}{\micro\meter} \AbstractPlainText{micrometer}, operating as a first of its kind direct conversion semiconductor neutron detector.
    Early detection experiments utilized \glsname{lise6}, enriched to 95\% in \isotope[6]{Li} \AbstractPlainText{lithium-6}, following the experimental investigation of enriched chalcopyrites for semiconductor detection.
    In this work, \gls{lise} interchangeably refers to its isotopically enriched complement (\glschemical{lise6}  or \glstext{lise6} \AbstractPlainText{lithium-6 indium diselenide}).
    The primary detection mechanism follows the \glstext{li6-neutron-alpha} \AbstractPlainText{lithium-6, neutron, alpha, hydrogen-3} reaction, with a Q-value of \SI{4.78}{\mega\electronvolt}.
    The proof-of-concept detector consisted of a single \gls{lise} crystal patterned with thin film gold contacts on opposite surfaces.
    After showing a semiconductor response to both \glspl{alpha-particle} and mixed neutron spectrum, the technology was extended to a $4\times 4$ pixel detector using square pixels of \SI{500}{\micro\meter} size and \SI{550}{\micro\meter} pitch.
    Using the \gls{super-sampling} technique, this system successfully resolved features of \SI{300}{\micro\meter}, roughly half the pixel pitch, in a \gls{neutron-cold} beam.
    Concurrently in the study, higher optical quality \gls{lise} sensors demonstrated a scintillation response to neutron exposure.
    An array of scintillating \gls{lise} sensors achieved a resolution of \SI{34}{\micro\meter}, calculated via \gls{mtf}, and were used to reconstruct a \gls{ct-neutron} of a small biological sample.
    Bolstering these results, a semiconducting \gls{lise} sensor was patterned with the \SI{55}{\micro\meter} pitch pattern, derived from the $256\times 256$ channel \glstext{timepix}.
    The \gls{lisepix} completed the groundwork for the detector as a \gls{high-res} neutron imager, surpassing the design goal with a published spatial resolution limit of \SI{34}{\micro\meter} (\gls{fwhm} of \SI{111}{\micro\meter}) for \gls{lise}.
    This project has demonstrated the first application of direct conversion semiconductors for neutron detection and imaging, while qualifying a viable neutron detection material for \gls{solid-state} devices.
    The \gls{lisepix} imaging technology offers a low-cost, low-power, compact neutron detection platform comparable to \gls{soa} neutron imaging technologies.
% ============================================================================ %
\end{document}%