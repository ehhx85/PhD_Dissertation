% ./appendices/thin-film-processing/bump-bonding.tex
% ============================================================================ %
\documentclass[../../main.tex]{subfiles}
\begin{document}%
% ============================================================================ %
    \section{Bump Bonding}%    
    \label{app:thin-film-processing:bump-bonding}%
    % ======================================================================== %
    As the channel density increases in a pixelated imaging device, the previously described wire bonding techniques are no longer viable.
    Flip chip bump bonding offers a method of directly coupling the pixelated sensor to an identically pixelated readout \gls{asic}.    
    Instead of bonding micron-thick wires from the detector pixels to the readout bonding pad, the pixels are electrically coupled to the associated \gls{asic} pixel via bump bonds \cite{Alimonti_2006,Alimonti_2013}.
    The bump ball diameters for this technology range from \SIrange{20}{40}{\micro\meter}, resulting in a bond height of \SIrange{10}{15}{\micro\meter} \cite{Adam_1999}.
    The applicable pixel pitches span from \SIrange{20}{50}{\micro\meter}, suitable for the \SI{55}{\micro\meter} \gls{timepix} system, common in \gls{high-res} radiation detection.
    The process has been illustrated below (\Xreffigure*{bump-bond-process}) courtesy \citeauthor*{book:Rossi_2006} (\citeyear*{book:Rossi_2006}), describing the original processing methodology detailed by \citeauthor*{Broennimann_2006} (\citeyear*{Broennimann_2006}) \cite{book:Rossi_2006, Broennimann_2006}.
    The flip chip bump bond facility used to process the \gls{lisepix} system applied the larger indium bump bonds to the \gls{timepix}, slightly altering the following process to accommodate \gls{lise} processing temperature limits.
    Step (\Xsubreffigure{bump-bond-process:a}) illustrates the chip coated with a capping layer such as \ce{SiO_2}, to insulate the underlying \gls{asic} circuitry.
    This capping layer exposes only the bonding pad for each pixel, precisely covering the edges of the contact to minimize current leakage.
    In Step (\Xsubreffigure{bump-bond-process:b}), the chip is coated in photoresist, and a thin base layer (\ce{Au}) is sputtered into the bonding well, creating a cradle for the bump bond up to the capping layer.
    Step (\Xsubreffigure{bump-bond-process:c}) follows a similar process as Step (\Xsubreffigure{bump-bond-process:b}), depositing a roughly \SI{5}{\micro\meter} thick layer of indium, with a \SI{50}{\micro\meter} diameter, to form the bump bond (described for a \SI{100}{\micro\meter} pixel pitch design) \cite{book:Rossi_2006}.
    After wet lift off, Step (\Xsubreffigure{bump-bond-process:d}) shows the \gls{asic} prepared with the bonding metals.
    % ======================================================================== %
    \par%
    % ======================================================================== %
    A similar sequence, following Steps (\Xsubreffigure{bump-bond-process:a}{\--}\Xsubreffigure{bump-bond-process:d}), is applied to the \gls{lise} sensor using a smaller thickness of \SI{2}{\micro\meter} and diameter of \SI{20}{\micro\meter} for the indium deposit.
    Step (\Xsubreffigure{bump-bond-process:e}) shows the effects of the \SI{180}{\celsius} reflow process, providing the activation energy for the indium to reform into a spherical bead. 
    The reflow is only applied to the \gls{asic} side and not the thinner indium on the sensor side.
    This prevents thermal shock in the \gls{lise} crystal along with lithium off-gassing which can negatively effective the device sensitivity.
    In Step (\Xsubreffigure{bump-bond-process:f}) the bumped \gls{asic} chip and the sensor are brought into alignment.
    As the sensor may be opaque and has metallized contacts, this process is performed blind, relying the on precision positioning stage to approximate the alignment.
    During the bonding process, Step (\Xsubreffigure{bump-bond-process:g}), the two substrates are compressed, roughly \SI{1.0}{\gram} per bump, and reflowed again, this time at a lower temperature of \SI{100}{\celsius}.
    The indium coat on the sensor side helps the bump bond to wet the contact, self-aligning each pixel to position the chip.
    Step (\Xsubreffigure{bump-bond-process:h}) shows the final bump bond, with the indium creating a spherical interconnect between the two contacts. 
% ============================================================================ %
\end{document}%