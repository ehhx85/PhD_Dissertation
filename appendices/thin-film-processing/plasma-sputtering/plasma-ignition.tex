% ./appendix/thin-film-processing/plasma-sputtering/plasma-ignition.tex
% ============================================================================ %
\documentclass[../../../main.tex]{subfiles}
\begin{document}%
% ============================================================================ %
    \subsubsection{Plasma Ignition}%
    \label{app:thin-film-processing:plasma-sputtering:plasma-ignition}%
    % ======================================================================== %
    At room temperature and pressure, the sputtering gas will not interact with enough energy to produce ionized particles, or sputter the target material.
    Thus, the plasma ignition process begins with vacuuming down the process chamber, reducing the chamber pressure along with the \gls{mfp} of internal ions and charged particles.
    First, the mechanical roughing pump is used to bring the chamber to medium vacuum, evacuating the atmospheric air and contamination gases from the system.
    A light, inert gas such as nitrogen may be injected into the system to help back fill the chamber, purging other contaminants.
    % ======================================================================== %
    \par%    
    % ======================================================================== %
    Once the pressure reaches between \SIrange{10}{100}{\milli\torr}, the turbo pump spools up, reducing the pressure to high vacuum levels so the plasma may be ignited.
    The system is pumped to around \SI{1}{\micro\torr} to ensure deadsorption of moisture from the internal surfaces.
    Once the chamber has been thoroughly purged, the sputtering gas (\ce{Ar}) is introduced, increasing the pressure to around \SI{1}{\milli\torr}.
    The selected target gun is switched on at a low power (\SIrange{10}{25}{\watt}) and begins the plasma ignition process.
    In the plasma, free electrons have a higher mobility than the more massive gas atoms, so they attain high kinetic energies in the presence of an applied field.
    Electrons will continue to bombard the cations, losing little energy in elastic collisions, until an inelastic collision causes the ionization of the molecule.
    After the electrons have been accelerated to high enough potentials, they ionize the process gas, stripping off electrons.
    A shutter sits on top of the chimney surrounding the target, and along with the use of magnetrons, confines the charged particles and sputtered atoms near the surface of the gun.
    The glow is caused by photon emission during the recombination of a cation and electron, and will be characteristic of the process gas.
    As the power is gradually increased, the density of the charge carriers in the chimney increases, consequently drawing more current through the plasma.
    Eventually, the ions present in the chimney will avalanche and create enough secondary ionization events to self stabilize the plasma. 
    After the plasma has been successfully ignited, the shutter is opened and process pressure may be lowered to sub-ignition values.
    % ======================================================================== %
    \par%
    % ======================================================================== %
    As the plasma carries a current, the initial startup barriers in the physical system have been overcome and lowering pressure helps to increase the sputtered atom \gls{mfp}.
    The low pressure ensures the sputtered atom reaches the substrate without interacting with a process gas ion.
    Maintaining a \gls{high-voltage} bias is sufficient to continue the production of ionized species for sputtering.
    The bias voltage and pressure are regulated based on the sputtering target material for optimum sputtering rates and deposited film quality.
% ============================================================================ %
\end{document}%