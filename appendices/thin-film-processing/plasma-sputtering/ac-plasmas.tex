% ./appendix/thin-film-processing/plasma-sputtering/ac-plasmas.tex
% ============================================================================ %
\documentclass[../../../main.tex]{subfiles}
\begin{document}%
% ============================================================================ %
    \subsubsection{Alternating Current (Radio Frequency) Plasmas}%
    \label{app:thin-film-processing:plasma-sputtering:ac-plasmas}%
    % ======================================================================== %
    To overcome the limitations of using \gls{dc} plasma sputtering, \gls{ac} bias permits sputtering high resistivity materials.
    Typical \gls{rf} plasma processing follows the \gls{fcc} designated frequency of \SI{13.56}{\mega\hertz}, however systems are designed across a range of \SIrange{5}{30}{\mega\hertz}.
    At higher frequencies, the coupling capacitor may charge and discharge continuously, allowing significant current to flow through the target material.
    Using an \gls{ac} system, insulating materials such as quartz and other oxides may be deposited, along with precious metals such as gold or platinum.
    At lower frequencies, below \SI{1}{\mega\hertz}, the \gls{ac} system performs similar to the \gls{dc} conditions, except the cathode and anode are cyclically swapped.
    When frequencies are increased above the \SI{1}{\mega\hertz} point, the electrons gain enough energy to ionize sputtering gas atoms, sustaining the plasma. 
    % ======================================================================== %
    \par% 
    % ======================================================================== %
    The rapidly oscillating electrons trapped between the electrodes will continue to ionize atoms not requiring the secondary source of electrons from the cathode.
    Furthermore, at frequencies above \SI{100}{\kilo\hertz}, the lighter electrons will acquire a larger fraction of the applied power, resulting in less energetic positive ions.
    Reducing the energy of the plasma gas ions lowers the energy imparted to the sputtered particles along with their impact damage on the sensor surface.
    Energetic sputtered atoms will create a peened topography on the surface, leaving the thin film in a state of compressive stress.
    At low energies, the sputtered ions will coalesce with the thin film and create a more homogeneous topography in tensile stress.
    Using the \gls{ac} sputtering method assists adhesion while simultaneously protecting the surface of delicate radiation detectors.
% ============================================================================ %
\end{document}%