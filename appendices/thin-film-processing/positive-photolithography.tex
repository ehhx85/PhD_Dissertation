% ./appendices/thin-film-processing/positive-photolithography.tex
% ============================================================================ %
\documentclass[../../main.tex]{subfiles}
\begin{document}%
% ============================================================================ %
    \section{Positive Photolithography}%    
    \label{app:thin-film-processing:positive-photolithography}%
    % ======================================================================== %
    Photolithography using \gls{pr1-1000a} finds use in masking substrates for etching, both aqueous and ion based.
    \Gls{pr1-1000a} is a photopolymer which softens, becoming soluble in developer, when exposed to photon emissions.
    This particular process will be described as it applies to \glspl{pcb} for copper trace wet etching.
    The photolithography chemicals for the \gls{pr1-1000a} process are listed in \Xreftable*{pattern-process-pos-chemicals}.
    The positive photolithography process, \Xreffigure*{photolithography-positive-pcb}, shows the sequential stages used to fabricate general \glspl{pcb}.
    % ======================================================================== %
    \par%
    % ======================================================================== %
    Step (\Xsubreffigure{photolithography-positive-pcb:a}) begins with a clean \gls{pcb} substrate of desired size to accommodate the circuit design.
    The board should be large enough to minimize the signal traces proximity to the edge of the \gls{pcb}, generally using a trim ring to outline the board.
    Most commercial grade \gls{pcb} substrates come pre-coated with a protective layer to inhibit surface oxidation and ensure nominal cladding thickness.
    This protective layer is removed using a scouring pad and \gls{acetone} to scrub the surface, dissolving the plastic and priming the metal surface to bond with the photoresist.
	A final rinse with \gls{ipa} then \gls{methanol} leaves a dry, clean surface, free of streaking and contaminants.
    % ======================================================================== %
    \par%
    % ======================================================================== %
    Step (\Xsubreffigure{photolithography-positive-pcb:b}) involves spin coating the \gls{pcb} with \gls{pr1-1000a}.
    The process also uses the \gls{spin-coater}, however disposable pipettes deliver the larger volume of resist required to coat the entire \gls{pcb} surface.
    The spin coating recipe also varies with the size of the board, requiring slower speeds for larger boards to adequately spread the resist across the surface.
    The overall spin cycle follows a similar profile as with the negative process recipe.
    Once coated, the \gls{pcb} is heated to \SI{120}{\celsius} for \SI{120}{\second} on the hot plate.
    % ======================================================================== %
    \par%
    % ======================================================================== %
    Step (\Xsubreffigure{photolithography-positive-pcb:c}) follows a more traditional sequence using the \gls{ma6}.
    The photomask is installed and the \gls{pcb} centered in the chuck.
    The alignment process benefits from the larger scale pattern, typically having fiducial markers at each corner to guide the edge.
	Furthermore, the photomasks for \gls{pcb} fabrication typically only have a single design across the quartz plate.
    % ======================================================================== %
    \par%
    % ======================================================================== %
    Step (\Xsubreffigure{photolithography-positive-pcb:d}) exposes the photoresist to the same \SI{365}{\nano\meter} \gls{uv} radiation.
    Here, the hard contact setting is used, with only a \SI{4}{\second} exposure time. 
    The shorter time minimizes the sidewall angle and undercut, producing the thickest mask possible to protect the surface during wet etching.
    % ======================================================================== %
    \par%
    % ======================================================================== %
    Step (\Xsubreffigure{photolithography-positive-pcb:e}) shows the softened photoresist after exposure.
    Because the pattern is generally much larger, the features are visible by the naked eye.
    The pattern should be inspected near the edges to ensure the resist was fully spread across the surface, and the pattern transferred to all regions where metal should be covered.
    If the \gls{pcb} is pre-drilled for through hole components, the area around the holes should also be reviewed.
    The metallization near holes is crucial for component electrical conductivity and establishing an appropriate surface for soldering.  
    % ======================================================================== %
    \par%
    % ======================================================================== %
    Step (\Xsubreffigure{photolithography-positive-pcb:f}) exposes the bare copper with the desired trace pattern protected by the insoluble resist.
    The \gls{pr1-1000a} is less sensitive to over development so the \gls{rd6} resist remover is liberally applied to the surface until the pattern is clearly developed.
    All excess resist must be removed, otherwise spots of metal may remain across the board, shorting traces or leading to signal breakdown.
    After development, the \gls{pcb} is rinsed with \gls{di-water} and dried with compressed nitrogen.
    An optional anneal may be conducted on the hot plate at \SI{145}{\degree} for \SI{5}{\minute} to further dry the resist and enhance adhesion for wet etching.
    % ======================================================================== %
    \par%
    % ======================================================================== %
    Step (\Xsubreffigure{photolithography-positive-pcb:g}) shows the \gls{pcb} etching process detailed in \Xrefappendix{general-chemistry:pcb-etching}.
    For more exotic \glsplural{pcb} that use other cladding materials, the process will be similar, using an etchant that reacts with the specific cladding metal and not the photoresist.
    After the etching process is complete, the \gls{pcb} will still have a thin layer of resist covering the copper traces.
    This layer is dissolved using \gls{acetone}, followed by a clean with \gls{ipa} and \gls{methanol} to remove and surface films and remaining contamination.
    % ======================================================================== %
    \par%
    % ======================================================================== %
    Step (\Xsubreffigure{photolithography-positive-pcb:h}) illustrates the final, one-sided, \gls{pcb}.
    Once the \gls{pcb} has be cleaned and dried, the pattern is inspected under a low power \gls{lm} to determine if any shorts or breaks exist in the trace pattern.
    As the bare copper layer will begin to oxidize, the \gls{pcb} components should be immediately soldered to the board.
    Furthermore, a protective insulating coating, such as Konform, should be applied to the surface to encapsulate the bare metal.
    The steps conducted in the positive photolithography process have been summarized in \Xreftable*{pattern-process-positive}.
% ============================================================================ %
\end{document}%