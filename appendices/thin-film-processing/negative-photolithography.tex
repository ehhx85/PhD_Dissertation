% ./appendices/thin-film-processing/negative-photolithography.tex
% ============================================================================ %
\documentclass[../../main.tex]{subfiles}
\begin{document}%
% ============================================================================ %
    \section{Negative Photolithography}%
    \label{app:thin-film-processing:negative-photolithography}%
    % ======================================================================== %
    Photolithography with \gls{nr9-1000py} is used to create thin film metal contacts on semiconductor grade radiation sensors.
    \Gls{nr9-1000py} describes a photopolymer which hardens, becoming insoluble to developing chemicals, after exposure to a prescribed spectra of light.
    The hardened resist creates a film coating used to selective apply thin film material coatings to a semiconductor surface.
    In particular, the method is used to fabricate the pixelated \gls{lise} imagers.
    The photolithography chemicals for the \gls{nr9-1000py} process are listed in \Xreftable*{pattern-process-neg-chemicals}.
    The negative photolithography process, \Xreffigure*{photolithography-negative-sensor}, shows the sequential stages used to fabricate the pixel detectors.
    % ======================================================================== %
    \par%
    % ======================================================================== %
    Step (\Xsubreffigure{photolithography-negative-sensor:a}) begins with a clean stainless steel planchet, serving as the wafer.
    The thin metal wafer is used in place of a standard silicon wafer to adapt to the spin coater and mask aligner chuck.
    A small hole is drilled in the center of the planchet to allow solvent to reach the \gls{apiezon-wax} underneath the sensor.
    The planchet is lightly polished and cleaned with solvents to remove residual \gls{apiezon-wax} and material contamination between solvents.
    % ======================================================================== %
    \par%    
    % ======================================================================== %
    Step (\Xsubreffigure{photolithography-negative-sensor:b}) shows the planchet pre-coated with \gls{apiezon-wax}.
    The hole in the plancet is cover with a piece of \gls{kapton} on the back side.
    The \gls{apiezon-wax} is shaved down and small flakes are placed in the center of the planchet.
    On a hot plate, the \gls{apiezon-wax} is melted down at \SI{90}{\celsius}, using drops of \gls{toluene} to dilute the \gls{apiezon-wax} and disperse it to the edges of the planchet.
    % ======================================================================== %
    \par%    
    % ======================================================================== %
    Step (\Xsubreffigure{photolithography-negative-sensor:c}) illustrates the \gls{lise} sensor after bonding to the \gls{apiezon-wax} coated planchet.
    Bringing the temperature up to \SI{100}{\degree}, the sensor should be allowed to sink into the \gls{apiezon-wax} so that the top surface is the only exposed area.
    The bonding process serves to protect the sensor's back surface, which may already be metallized, as well as the sidewalls from further metal deposition.
    If the sidewalls are coated in metal, there is potential for voltage breakdown, allowing charge to travel along the outside of the sensor instead of through the sensor bulk.
    This step also serves as a dehydration bake, drying the surface of the sensor and removing water that may reduce metal adhesion.
    Once bonded, the wafer should be placed on the workbench to slowly cool back to room temperature, preventing thermal shock in the sensor.
    % ======================================================================== %
    \par%    
    % ======================================================================== %
    Step (\Xsubreffigure{photolithography-negative-sensor:d}) follows the application of \gls{nr9-1000py}.
    The \gls{nr9-1000py} is stored in a chilled refrigerator to prolong the chemical life, protecting the photosensitive compound.
    For dispensing, a small quantity is stored at room temperature in a tinted dropper bottle, lowering the viscosity for spin coating.
    The resist is applied via eye dropper to the surface of the \gls{lise} sensor, dispensing roughly \SI{50}{\micro\liter} of resist per drop.
    Immediately after applying the photoresist, the layer is spread thin by the \gls{spin-coater} to a nominal \SI{1}{\micro\meter} thickness as described in \Xreftable*{spin-coater-recipe}.
    Following spin coating, the wafer is removed using wafer tweezers and set on the hot plate for a \SI{60}{\second} pre-bake at \SI{150}{\celsius}.
    The pre-bake helps to dry the resist, eliminating any residual moisture and allowing a slight anneal of the resist to homogenize thickness.
    The wafer is set to cool on the table to slowly remove heat from the sample.
    % ======================================================================== %
    \par%    
    % ======================================================================== %
    Step (\Xsubreffigure{photolithography-negative-sensor:e}) shows the pattern transfer process, conducted in the \gls{ma6}.
    The specific device mask is loaded into the contact printer and the wafer is aligned, positioning the pattern over the sensor area.
    This is the step where the use of a substrate wafer is crucial.
    Typically, the contact printer is used with a standard \SIrange{3}{4}{\inch} silicon wafer.
    The entire mask pattern is transferred to the wafer, with extraneous features being removed along with the \gls{apiezon-wax}.
    % ======================================================================== %
    \par%    
    % ======================================================================== %
    Step (\Xsubreffigure{photolithography-negative-sensor:f}) shows the resist exposure process.
    The photoresist on the crystal is exposed for 15 seconds using the I-line (365 nm) \gls{uv} radiation from the pressurized mercury vapor lamp. 
    The \gls{uv} light chemically reacts with the photoresist, causing the exposed regions to harden through cross-linking of the polymeric bonds.
    % ======================================================================== %
    \par%    
    % ======================================================================== %
    Step (\Xsubreffigure{photolithography-negative-sensor:g}) captures the post-exposure bake, assisting the polymer cross-linking by applying \SI{100}{\celsius} via hot plate for \SI{60}{\second}.
    This bake helps to remove the nitrogen component of the photo-compound which would be otherwise detrimental in the sputtering process due to off gassing.
    The bake also facilitates the breakdown of unexposed regions for pattern development.
    % ======================================================================== %
    \par%    
    % ======================================================================== %
    Step (\Xsubreffigure{photolithography-negative-sensor:h}) follows the pattern development using \gls{rd6} for \SIrange{8}{10}{\second}.
    The developer dissolves the soluble, unexposed regions, carrying away the polymer and leaving an exposed region mimicking the desired pattern.
    Depending on the development time, some undercutting will occur where the developed photoresist leaves a slight negative grade sidewall profile.
    This effectively creates a small (\SIrange{1}{5}{\micro\meter}) gap between the open channels and the edge of the resist.
    Without this gap, lift-off can become either difficult or impossible.
    % ======================================================================== %
    \par%    
    % ======================================================================== %
    Step (\Xsubreffigure{photolithography-negative-sensor:i}) shows the metallized sensor.
    Once the sample has been patterned, the crystal is inspected using an optical microscope.
    This is the first point the pattern can be inspected without exposing the resist using the white light from the microscope.
    After pattern inspection, the substrate wafer is loaded on the \gls{sputterer} chuck and vacuumed down to \SI{1.0}{\micro\torr} or less.
    Reaching high-vacuum first helps to evacuate residual nitrogen along with moisture that has been adsorbed to the various surfaces.
    One advantage of the \gls{apiezon-wax} is its ability to getter impurities in a high vacuum system.
    After the system is evacuated, the pressure is adjusted to roughly \SI{4.0}{\milli\torr} for plasma sputtering.
    A high-purity metallic target is the source of atoms for the uniform thin film deposited on the photoresist and exposed regions of the crystal.
    % ======================================================================== %
    \par%    
    % ======================================================================== %
    Step (\Xsubreffigure{photolithography-negative-sensor:j}) depicts the lift-off stage conducted in a solvent bath.
    A glass petri dish containing \gls{rr41} is heated on the hot plate to \SI{80}{\celsius}.
    Before placing the metallized wafer in the petri dish, scratches are made along the non-sensor areas using the ceramic tweezers, exposing the \gls{apiezon-wax} underneath.    
    The scratches allow the resist remover to penetrate underneath the metal thin film, attacking the resist beneath.
    Depending on the pattern complexity and coverage area, the development times will differ.
    Introducing small amounts of \gls{acetone} into the solution will speed the dissolution process, however, too much \gls{acetone} can cause the loose metal film to re-adhere to the surface, shorting contacts.
    Once the excess metal has been removed, exposing the final pattern, \gls{toluene} should be added to the solution to begin dissolving the \gls{apiezon-wax}.
    As the petri dish holds a limited volume and the solvents are highly volatile, the solution will continuously evaporate and additional toluene must be introduced.
    Typically, the solution will saturate, turning opaque black, requiring a fresh petri dish of pure toluene.
    Repeating this process will eliminate the \gls{apiezon-wax}, now in solution, along with lifted metal that my contaminate the sensor surface.
    % ======================================================================== %
    \par%    
    % ======================================================================== %
    Step (\Xsubreffigure{photolithography-negative-sensor:k}) shows the finalized sensor, free from the planchet.
    The metallized crystal should be removed from the toluene and rinsed in \gls{ipa} then \gls{methanol} to remove residual contaminants, providing a final clean before air drying with nitrogen.    
    All devices require inspection using the optical microscope before integration with other electronics to characterize pixel size and metal film quality.
    If the sensor has only been metallized on one side, these steps are repeated with the opposite side of the sensor exposed to the patterning process.
    % ======================================================================== %
    \par%    
    % ======================================================================== %
    The steps conducted in the negative photolithography process have been summarized in \Xreftable*{pattern-process-negative}.
    As a preliminary validation test, a silicon wafer is patterned with aluminum metal, shown in \Xreffigure*{silicon-wafer-photolithography-test}.
    Because the systems are designed for silicon wafers, the process parameters may be dialed in without the complications of the planchet carrier or fragile crystalline sensor.
    The test wafer also serves to characterize the quality of new photomasks, ensuring the design and pattern scaling meets the requirements of the detector.
    After the process sequence has been reviewed, the semiconductor grade sensors are patterned on both sides, taking great care to minimize stress on the sensor as well as material waste.
% ============================================================================ %
\end{document}%