% ./appendices/thin-film-processing/wire-bonding.tex
% ============================================================================ %
\documentclass[../../main.tex]{subfiles}
\begin{document}%
% ============================================================================ %
    \section{Wire Bonding}%    
    \label{app:thin-film-processing:wire-bonding}%
    % ======================================================================== %
    An analog to soldering wires on \glsplural{pcb}, wire bonding describes a process of electronically coupling micron-scale electronics. 
    The \gls{kns} \gls{wirebonder} implements a wedge bond technique to connect \SIrange{0.5}{3.0}{\mil} diameter wire between device contacts.
    The standard process for wedge bonding follows the sequence described by \citeauthor*{Fischer_2013} in \Xreffigure*{wedge-bond-process} \cite{Fischer_2013}.
    In Step (\Xsubreffigure{wedge-bond-process:a}), the wire is threaded through the bonding wedge tip.
    Care is taken when passing the wire through the wedge orifice to avoid bending the wire, introducing kinks that reduce the mechanical stability of the wire bond.
    The wire clamp closes on the bonding wire to ensure the ultra-light wire remains in position and does not retract on the spool.
    Step (\Xsubreffigure{wedge-bond-process:b}) makes the first bond on the sensor, by depressing the wire bonder head into the contact surface with a specified compressive force ranging from \SIrange{10}{160}{\gram}.
    The established settings for the \gls{kns} \gls{wirebonder} are presented in \Xreftable*{wire-bonder-settings}. 
    The bonding head utilizes a \SI{60}{\kilo\hertz} \gls{usonic} transducer to vibrate the wedge, creating the metallic weld between the bond wire and contact pad.
    In Step (\Xsubreffigure{wedge-bond-process:c}), the wire clamp opens and the wedge bonding tool is lifted to a prescribed height while moving in the direction of the next bond.
    Step (\Xsubreffigure{wedge-bond-process:d}) shows the wedge tool descending on the second bond, smoothly arching the wire, creating a permanent bend to suspend the wire over other electronics.
    Again, the wedge is compressed into the bonding pad with \gls{usonic} agitation to establish the second bond.
    Finally in Step (\Xsubreffigure{wedge-bond-process:e}), the wire is pinched by the wire clamp and the wedge tool jerks backwards, breaking the wire and leaving the finished bond.
    % ======================================================================== %
    \par%
    % ======================================================================== %
    All bonds fabricated in this research utilized \SI{99.99}{\percent} \ce{Au} wire of \SI{1.0}{\mil} diameter bonded with a \SI{45}{\degree} wedge.
    The heated bonding stage was also set to a temperature of \SI{100}{\celsius} to promote the \ce{Au} wire to \ce{Au} bonding pad adhesion.
    Common practice finds \gls{asic} devices housed in a chip carrier.
    This type of packaging bonds an \gls{asic} substrate to the center well of a multi-channel readout chip, with wire bonds extending from the top of the \gls{asic} to the readout pads, as shown in \Xreffigure{wire-bonding:a}.
    Each channel is insulated, connecting to a pin extending from the back of the chip, allowing the chip carrier to mate with a matched socket.
    Chip carriers may be directly attached, or mounted via chip socket to a \gls{pcb} containing peripheral electronics and connections.
    This configuration offers a small form factor, with a replaceable package in the event of \gls{asic} failure.
    For radiation detectors, this type of packaging permits the testing of multiple sensors without having to alter the electrical connections to readout electronics.
    % ======================================================================== %
    \par%
    % ======================================================================== %
    Alternatively, the sensor may be directly mated to a device specific \gls{pcb} as seen in \Xreffigure*{wire-bonding}.
    Wire bonding pads are included on the surface of the \gls{pcb}, routed directly to through hole soldering vias.
    The traces from the sensor \gls{pcb} are connected with a smaller jumper wire to the input traces of the \gls{preamp} module.
    This configuration offers a nearly direct connection between the radiation sensor and readout electronics.
% ============================================================================ %
\end{document}%