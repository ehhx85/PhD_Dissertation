% ./appendices/thin-film-processing/introduction.tex
% ============================================================================ %
\documentclass[../../main.tex]{subfiles}%
\begin{document}%
% ============================================================================ %
    \Xsection%    
    % ======================================================================== %
    The science of thin film processing opened new avenues in the miniaturization of modern electronics \cite{book:Ohring_2001}.
    Semiconductor electronics depend on precision fabrication processes, ranging from the millimeter down to the nanometer scale.
    To achieve the quality required for such devices, processing equipment must be operated in a clean room, protecting sensitive surfaces from dust, oils and particulate contamination.
    In this controlled environment, even the spectrum of lighting is manipulated based on operational requirements, as shown in \Xreffigure*{clean-room}.
    The raw \gls{lise} crystal becomes a radiation sensor through photolithography and subsequent plasma sputtering.
    The delicate metallization process adheres high purity, thin film contacts to the sensor, without disrupting the material composition or crystalline structure.
    Packaging the sensor using wire-bonding or flip chip bump bonding creates fine pitch electrical signal channels, collecting data from each pixel.
    The resulting semiconductor device maintains radiation and electronic performance required for a semiconductor neutron detector.
% ============================================================================ %
\end{document}%