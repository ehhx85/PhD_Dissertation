% ./appendices/image-processing/preprocessing.tex
% ============================================================================ %
\documentclass[../../main.tex]{subfiles}
\begin{document}%
% ============================================================================ %
    \section{Neutron Image Pre-Processing}%
    \label{app:image-processing:preprocessing}%
    % ======================================================================== %
    Typically, neutron images start with 3 sets of data: the dark field, the open beam or white field, and the object image.
    Each data set is composed of multiple frames, captured using identical settings, to create an average image.
    % ======================================================================== %
    \subsubsection{Digital Imaging}%
    % ======================================================================== %
    The pre-processing sequence is used to accumulate the individual, raw acquisition files into a compiled image, representative of the data set.
    This processing may be completed using a range of image analysis tools, however the size and number of images are best accommodated by a well planned processing algorithm.
    A standard tool in the digital image community, \gls{imagej} features a wide range of prepackaged functions and algorithms.
    For larger jobs, or multiple experimental setups, the practical approach incorporates matrix based computing environments such as \gls{matlab} or open source scripting languages such as \gls{python}.
    The specific \gls{python} code to process the \gls{lisepix} binary data files has been included in \Xrefsupplement*{timepix-image-preprocessor}.
    % ======================================================================== %
    \par%
    % ======================================================================== %
    Using the programmatic software approach can render a set of final images from hundreds or thousands of input frames without the need for manual user intervention.
    This strategy helps to minimize user input, code authoring redundancy and user error in large data set analysis.
    A digital image is defined by an \Xmath{M\times N} array of intensity values, given in \Xrefequation{image-matrix-function} \cite{book:Gonzalez_2007}.
    % ======================================================================== %
    \Xequationfile{image-matrix-function}%
    % ======================================================================== %
    The functional representation \Xvariable{f(x,y)} describes the intensity at each pixel location given by the indices \Xvariable{(x,y)}[].
    The traditional matrix form may also be used and serves as an equivalent representation.
    Using the definition \Xvariable{a_{ij}=f(x=i,y=j)=f(i,j)} the matrix takes the form given in \Xrefequation{image-matrix-traditional} \cite{book:Gonzalez_2007}.
    % ======================================================================== %
    \Xequationfile{image-matrix-traditional}%
    % ======================================================================== %
    \subsubsection{Dark Field}%
    % ======================================================================== %
    The dark field image captures the system response, fully powered and biased, without the radiation source or beam.
    This measurement will capture the electronic baseline, defined by the leakage current, also known as the dark current, along with any dead (non-responsive) or hot (fully saturated) channels.
    The dark field also serves as the initial operational check, ensuring the cabling, and readout electronics have been properly installed at the experiment site.
    If the device was damage, a cable is shorted, or the operational temperatures of the system are outside the normal parameters, the dark field measurement will show a non-radiation based response, indicating system malfunction.
    % ======================================================================== %
    \subsubsection{Open Beam}%
    % ======================================================================== %
    The open beam image, also referred to as the white field, shows the characteristic response of the imaging system to a given radiation field.
    The name white field describes the image produced from capturing an unobstructed radiation source, returning a value approaching the saturation limit for all functioning pixels.
    Depending on the intensity of the radiation source, the acquisition is tailored to collect data that spans a significant portion of the dynamic system intensity range without over saturating the sensor.
    The open beam image is a convolution of the non-uniformities in the radiation beam and each sensor channel. 
    As a neutron beam is moderated, redirected and filtered to a desired energy range, periodic banding will occur perpendicular to the beam.
    Furthermore, the output in a given channel to a detection event may vary across channels as a result of minor inhomogeneities during manufacturing.
    The subsequent acquisitions for the object image should ideally be repeated using identical operational settings, counting time, and averaging frames.
    % ======================================================================== %
    \subsubsection{Object Image}%
    % ======================================================================== %
    The object image is the last data set to be collected.
    The imaging target object should be a radiation absorber, removing a significant portion of the radiation flux to create a discernible change from the open beam response (\gls{latin:ie} contrast).
    While the object may be any article under investigation, it helps if the material has a uniform composition, or posses constituents with significantly varying levels of attenuation to enhance image contrast.
    % ======================================================================== %
    \subsubsection{Process}%
    % ======================================================================== %
    The first pre-processing step averages each frame in a sequence into a single raw image file.
    For the most basic imaging experiment with a fixed acquisition time (or temporally normalized frames), this process reduces \Xvariable[]{K_{exp}=K_{DF}+K_{OB}+K_{I}}[] frames into 3 mean images corresponding to the dark field, open beam, and object image.
    Each mean image \Xvariable{\bar{f}(x,y)} is created following \Xrefequation{image-matrix-average} from \Xvariable[]{K}[] frames in the set.
    % ======================================================================== %
    \Xequationfile{image-matrix-average}%
    % ======================================================================== %    
    The dark field typically requires only a few frames \Xvariable{K_{DF}} to capture the radiation independent electronic signal.
    The number of open beam frames \Xvariable{K_{OB}} will depend on the radiation source strength as well as the sensitivity of the detection system to the given radiation spectrum.
    As the experimental variable, the object image will generally require the largest number of frames \Xvariable{K_{I}} to ensure sufficient contrast between the open beam and attenuated signal from the object.
    % ======================================================================== %
    \par%
    % ======================================================================== %
    After averaging, the dark field is subtracted from both the open beam and the object images.
    This effectively reduces the image baseline intensity to zero, facilitating the normalization process.
    In most cases, the dark field will respond as a uniform gray value, however, dead and hot pixels will be resolved by this subtractive procedure.
    The baseline corrected object image is divided by the baseline corrected open beam image to produce the normalized experimental image.
    The resultant image is scaled between zero and one, with zero indicating non-beam areas and one corresponding to full beam view.
    By normalizing the object image to the open beam, any inconsistencies in the neutron flux should be removed from the final image.
% ============================================================================ %
\end{document}%