% ./appendices/image-processing/modulation-transfer-function/analytical-methods/region-of-interest.tex
% ============================================================================ %
\documentclass[../../../../main.tex]{subfiles}
\begin{document}%
% ============================================================================ %
    \subsubsection{Region of Interest}%
    \label{app:image-processing:modulation-transfer-function:analytical-methods:region-of-interest}%
    % ======================================================================== %
    The user specified \gls{roi} captures an area that roughly splits the edge profile between open beam and attenuated sections.
    To preserve the fidelity of the calculation, this region should be exclusive of image artifacts or sensor dead zones (\Xreffigure*{mtf-roi}).
    The \gls{roi} is cropped to the specified area (\Xmath{m\times n}), reducing the data size and accelerating the calculation processing time. 
    The smaller image area (\Xreffigure*{mtf-cropped}) more clearly depicts the edge profile angle.
    Increasing the data index, perpendicular to the edge profile, benefits the analytical process without contaminating the data.
    As a smoothing algorithm, interpolation prevents artificial edge sharpening as seen in the \Xmath{10\times} gain image (\Xmath{m\times n^{*}}) shown in \Xreffigure{mtf-cropped:c} and \Xreffigure{mtf-cropped:d}.
    The increased sampling rate benefits the subsequent alignment and differentiation steps.
    Furthermore, the extra data points will not change the overall shape of the final \gls{mtf} curve but will provide more points to accurately estimate the \SI{10}{\percent} threshold.
% ============================================================================ %
\end{document}%