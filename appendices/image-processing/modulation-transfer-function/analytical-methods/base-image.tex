% ./appendices/image-processing/modulation-transfer-function/analytical-methods/base-image.tex
% ============================================================================ %
\documentclass[../../../../main.tex]{subfiles}%
\begin{document}%
% ============================================================================ %
    \subsubsection{Base Image}%
    \label{app:image-processing:modulation-transfer-function:analytical-methods:base-image}%
    % ======================================================================== %
    The base image served as the input matrix \Xvariable{(M\times N)} for the \gls{mtf} calculator after pre-processing, as described in \Xrefappendix{image-processing:preprocessing}.
    Along with the pixel size \Xvariable{S_{pix}} and selected region of interest, the calculation process is automated to generate reliable, repeatable results.
    The first processing step comprises plotting the raw intensity data in both pixel and distance axes to ensure proper pre-processing.
    The pixel size is used to scale the input data (\Xreffigure*{mtf-raw}), providing a usable output value from the \gls{mtf} curve.
    If the edge profile is aligned parallel to the horizontal pixel rows, the calculated resolution value would be lower limited by the pixel size.
    For this reason, a slight angle is introduced during the experiment to create an edge profile spanning multiple pixel rows.
% ============================================================================ %
\end{document}%