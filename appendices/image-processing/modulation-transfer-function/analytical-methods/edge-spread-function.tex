% ./appendices/image-processing/modulation-transfer-function/analytical-methods/edge-spread-function.tex
% ============================================================================ %
\documentclass[../../../../main.tex]{subfiles}
\begin{document}%
% ============================================================================ %
    \subsubsection{Edge Spread Function}%
    \label{app:image-processing:modulation-transfer-function:analytical-methods:edge-spread-function}%
    % ======================================================================== %
    After interpolation, the region of interest is flattened by plotting the edge profile, perpendicular to the edge axis, as a function of distance.
    Shown in \Xreffigure*{mtf-esf-raw}, the edge profile demonstrates varying intensity in both white field and dark field regions.
    Furthermore, the edge profile has a distinguishable width due to the edge angle imparted by the experimental setup.
    Some vertical banding can be seen as an artifact of the finite interpolation gain.
    For each interpolated edge profile \Xvariable{i} the edge angle \Xvariable{\theta} can be removed by offsetting the distance vector \Xvariable{d_{i}} using the calculated value from the Hough transform and the total distance along the edge axis \Xvariable{x_{i}}[] (\Xrefequation{edge-alignment}).
    % ======================================================================== %
    \Xequationfile{edge-alignment}%
    % ======================================================================== %
    The resulting edge profile, \Xreffigure*{mtf-esf-aligned}, shows a narrower edge transition and no longer has clear banding from the interpolation process.
    The aligned \gls{esf} still exhibits varying intensity in the dark field and more prominently in the white field regions, due to varying pixel performance and inconsistencies in the open beam.
    The distance vectors no longer correspond to the original binning after the alignment process.
    Thus, the \gls{esf} curves are rebinned using the sampling rate in the raw \gls{esf} profile, generating the averaged \gls{esf} profile shown in \Xreffigure*{mtf-esf-averaged}.
    The averaged \gls{esf} exhibits a significantly smoother white field as well as the dark field region.
    The edge transition region also follows a smoother slope, benefiting the differentiation process.     
    After calculating the \gls{lsf}, the edge inflection point was determined and the centered \gls{esf} plotted in \Xreffigure*{mtf-esf-centered}.
% ============================================================================ %
\end{document}%