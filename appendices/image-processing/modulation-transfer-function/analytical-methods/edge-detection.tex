% ./appendices/image-processing/modulation-transfer-function/analytical-methods/edge-detection.tex
% ============================================================================ %
\documentclass[../../../../main.tex]{subfiles}
\begin{document}%
% ============================================================================ %
    \subsubsection{Edge Detection}%
    \label{app:image-processing:modulation-transfer-function:analytical-methods:edge-detection}%
    % ======================================================================== %
    The first step in the edge detection process applies the Canny edge detector to the cropped \gls{roi} \cite{Canny_1986}, generating the binary image shown in \Xreffigure*{mtf-canny}.
    The Canny edge detection algorithm is automated as a \gls{matlab} function, but may be generalized to \num{4} individual steps common among software packages.
    Step (1) reduces the image noise by preprocessing with a Gaussian filter, typically \num{5x5} in size. 
    In step (2), a Sobel kernel is applied along both directions to evaluate the intensity gradients and associated angle for each pixel \cite{book:Shih_2010}.
    Step (3), suppresses all non-maximum pixels, effectively determining if a pixel falls along the edge based on its gradient direction and magnitude compared to neighboring pixels.  
    The final step (4) applies thresholding hysteresis to extend pixels that fall along an edge, effectively creating a binary image of only the pixels forming edge lines.
    \par%
    The Hough transform has been implemented in a vast range of image processing methods to detect lines or edges in image data \cite{Duda_1972,Hart_2009}.
    To evaluate colinear points in a sample image, the location of each Cartesian point can be represented by the Hesse normal form \Xrefequation{hesse-form}.
    % ======================================================================== %
    \Xequationfile{hesse-form}%
    % ======================================================================== %
    The length \Xvariable{\rho_{i}} of the line from the origin to the perpendicular intersection at point position \Xvariable{(x,y)} depends on the angle \Xvariable{\theta_{i}} between the line and the \Xmath{x}-axis.
    The image is sampled across the range \Xvariable{\SI{-90}{\degree}\leq\theta_{i}<\SI{90}{\degree}} and non-zero entries are calculated and tallied, shown in \Xreffigure*{mtf-hough}.
    The Hough transform function returns a simple data set in this instance, because the input is a binary, few line image.
    The transform data is evaluated to find the data peak, indicating the primary edge angle used in the alignment process.
% ============================================================================ %
\end{document}%