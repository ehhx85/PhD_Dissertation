% ./appendices/image-processing/modulation-transfer-function/mtf-calculator.tex
% ============================================================================ %
\documentclass[../../../main.tex]{subfiles}
\begin{document}%
% ============================================================================ %
    \subsection{MTF Calculator}%
    \label{app:image-processing:modulation-transfer-function:mtf-calculator}%
    % ======================================================================== %
    The \gls{mtf} calculator program (see \Xrefsupplement*{mtf-calculator}) comprises a series of \gls{matlab} scripts to process an image file, evaluating the edge resolution parameter.
    The main script file is run from the root directory.
    Function source files are stored in the ``/functions" subdirectory.
    The raw images, saved in \gls{tiff} file format, are stored in the ``/inputs" subdirectory.
    The individual raw images are pre-processed in an external image software such as \gls{imagej}.
    All figure files generated from the script are saved in the ``/outputs" subdirectory, in \gls{png} format.
    % ======================================================================== %
    \subsubsection{main.m}%
    % ======================================================================== %
    This script serves as the user defined input file and the active run.
    The calculator requires 3 input variables: the file name to select the image, the device pixel size for scaling the output plots, and the region of interest to crop the raw image.%
    % ======================================================================== %
    \Xcodefile[mtf-calculator]{main.m}[Matlab]%
    % ======================================================================== %
    \subsubsection{calculate\_mtf.m}%
    % ======================================================================== %
    This script is the primary function controlling the calculation.
    The manipulated data exists in the scope of this function, with subfunctions being called to perform specific operations on the data.
    After processing the data, all figures are sequentially generated at the end of the function.
    The individual subfunctions have been included in \Xrefsupplement{mtf-calculator}, with their methods detailed in the previous \Xrefappendix{image-processing:modulation-transfer-function:analytical-methods}.%
    % ======================================================================== %
    \Xcodefile[mtf-calculator]{calculate_mtf.m}[Matlab]%
% ============================================================================ %
\end{document}%