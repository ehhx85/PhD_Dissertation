% ./appendices/image-processing/mosaic-stitching.tex
% ============================================================================ %
\documentclass[../../main.tex]{subfiles}%
\begin{document}%
% ============================================================================ %
    \section{Mosaic Stitching}%
    \label{app:image-processing:mosaic-stitching}%
    % ======================================================================== %
    When prototyping a new crystalline imaging material, the availability of sensor samples may be limited by growth experimentation.
    With each trial, the electronic quality of the single crystals improves, along with stochiometric refinement.
    However, in early sensor design, the available imaging area is limited, representing only a fraction of the boule cross-sectional area.
    This section details the mosaic stitching methodology applied to increase the effective area for smaller sensors, using multiple frames as seen in \Xreffigure*{power-t-mosaic-stitching-layout}.
    As a scintillator, the \gls{lise} sensor was only able to capture a portion of the \gls{power-t} imaging target.
    Using \Xvariable{K=3} different object positions, the full image is visible across the sensor active area, with a degree of overlap.
    It should also be noted that each of the \Xvariable[]{K}[] images are already cleaned and averaged from the raw acquisition, as described in \Xrefappendix{image-processing:preprocessing}.
    First, \Xrefequation{mosaic-stitch-matrix-size} determines the expanded matrix of size \Xvariable{M^{*}\times N^{*}} to span the increased image dimensions introduced by the largest image position shifts \Xvariable{\Delta x}[] and \Xvariable[]{\Delta y}[].
    % ======================================================================== %
    \Xequationfile{mosaic-stitch-matrix-size}%
    % ======================================================================== %
    In this example, the top left image sees a vertical shift \Xvariable{\Delta y>0} the bottom center sees a horizontal shift \Xvariable{\Delta x>0} and the top right requires both \Xvariable{\Delta x>0}[] and \Xvariable[]{\Delta y>0}[].
    The expanded image \Xvariable{f^{*}_{i}(x,y)} is preallocated with a zero matrix to boost processing speed and ensure any unmapped values do not affect the final result. 
    These image shifts are applied in \Xrefequation{mosaic-stitch-shifted-function} where the pixel map incorporates a pixel sized shift across the image.
    % ======================================================================== %
    \Xequationfile{mosaic-stitch-shifted-function}%
    % ======================================================================== %
    Because the stitching process is based on an averaging function, the number of images used to average a pixel will range span \Xvariable[]{[1,K]}[].
    While the images are being remapped to the expanded matrix \Xvariable{f^{*}_{i}(x,y)} an equivalent sized binary matrix \Xvariable{g^{*}_{i}(x,y)} is created to tally the locations containing active pixel data, as in \Xrefequation{mosaic-stitch-shifted-tally}.
    % ======================================================================== %
    \Xequationfile{mosaic-stitch-shifted-tally}%
    % ======================================================================== %
    The finalized stitched mosaic image \Xvariable{h(x,y)} is formed from dividing the summed expanded image \Xvariable{F^{*}(x,y)} by the summed counting matrix \Xvariable{G^{*}(x,y)} as seen in \Xrefequation{mosaic-stitch-composite}.
    % ======================================================================== %
    \Xequationfile{mosaic-stitch-composite}%
    % ======================================================================== %
    This process is visually demonstrated in \Xreffigure*{power-t-mosaic-stitching-comparison}.
    The end result is an image spanning an area larger than the original sensor, while maintaining feature contrast and spatial resolution across overlapping regions.
    It should also be noted the edges of sensor area were cropped to produce a perfect rectangular area, before stitching.
    Leaving the as-cut sensor outline created artifacts due to inconsistent averaging across those pixels.
    These edge pixels have varying response due to multiple factors: large surface roughness at the edge interface, scattering at sharp crystal edges, inconsistent material thickness, light escaping from the side of the sensor, etc. 
    The cropping was conducted iteratively, removing only enough rows and columns to eliminate the sporadic averaging.
% ============================================================================ %
\end{document}%