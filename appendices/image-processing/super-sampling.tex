% ./appendices/image-processing/super-sampling.tex
% ============================================================================ %
\documentclass[../../main.tex]{subfiles}
\begin{document}%
% ============================================================================ %
    \section{Super-Sampling}%
    \label{app:image-processing:super-sampling}%
    % ======================================================================== %
    Radiographic image \gls{super-sampling} outlines a technique to provide enhanced image resolution in a few channel pixel detector, or a pixel pitch limited device.
    In the early \gls{lise} semiconductor systems, the device functioned as a counter (single channel) or a 16-channel pixel detector.
    With such few channels, any decrease in pixel size to enhance resolution would result in a subsequent decrease in pixel active area.
    % ======================================================================== %
    \par%
    % ======================================================================== %
    To overcome this limitation, larger pixels of pitch \Xvariable{P_{x} \times P_{y}} were used to maximize coverage area and conform to the dimensional requirements for wirebonding.
    In the experimental setup, the imaging target was translated in both directions perpendicular to the beam axis, sweeping across the sensor active area.
    By sub-stepping the target at intervals smaller than the pixel pitch \Xvariable{P_{u} \times P_{v}} each pixel captures a averaged response corresponding to the region of the target in front of the pixel.
    Plotting the response for each channel, shown in \Xreffigure*{lise-16ch-mosaic}, demonstrates the ability of a single physical pixel to resolve an image using precise target manipulation.
    For a single acquisition position, the sensor will capture data for the physical \Xmath{M_{0}\times N_{0}} pixels, producing the base image \Xvariable{f(x,y)} given by \Xrefequation{super-sampling-single-acquisition}.
    % ======================================================================== %
    \Xequationfile{super-sampling-single-acquisition}%
    % ======================================================================== %
    The matrix is also shown as a series of indexed pixels, each channel \Xvariable{f_{i}} assigned an integer value spanning \Xmath{[1,M_{0}N_{0}]} in \Xrefequation{super-sampling-pixel-index}.
    % ======================================================================== %
    \Xequationfile{super-sampling-pixel-index}%
    % ======================================================================== %
    As previously mentioned, the few channel image suffers from poor resolution as a result of maximizing the pixel pitch.
    Instead, the intensity value \Xvariable{f_{i}} from each physical pixel will be recorded as a single virtual pixel \Xvariable{f_{i}(u,v)} of size \Xvariable{P_{u}\times P_{v}} spanning the \Xmath{M\times N} sub-steps, shown in \Xrefequation{super-sampling-single-channel}.
    % ======================================================================== %
    \Xequationfile{super-sampling-single-channel}%
    % ======================================================================== %
    Each virtual image will be offset \Xvariable{\Delta u}[] and \Xvariable[]{\Delta v} along the respective direction for the stitching process, shown in \Xrefequation{super-sampling-matrix-spacing:a}.
    % ======================================================================== %
    \Xequationfile{super-sampling-matrix-spacing}%
    % ======================================================================== %
    If the fractions \Xvariable{P_{x}/P_{u}}[] or \Xvariable[]{P_{y}/P_{v}} do produce integer quotients, a constant value \Xvariable{C} is introduced in \Xrefequation{super-sampling-matrix-spacing:b}, further sub-dividing each virtual pixel into sub-pixels of size \Xvariable{P_{u}/C \times P_{v}/C}[]. 
    The expanded matrix size \Xvariable{M^{*}\times N^{*}} takes a similar form as \Xrefequation{mosaic-stitch-matrix-size}, shown here as \Xrefequation{super-sampling-matrix-size:a} and \Xrefequation{super-sampling-matrix-size:b}, depending on the sub-division of the virtual pixels.
    % ======================================================================== %
    \Xequationfile{super-sampling-matrix-size}%
    % ======================================================================== %
    The virtual images from each physical pixel are mapped to the larger composite image space.
    The remapping image function is given by \Xvariable{f^{*}_{i}(u,v)} as seen in \Xrefequation{super-sampling-shifted-function}.
    % ======================================================================== %
    \Xequationfile{super-sampling-shifted-function}%
    % ======================================================================== %
    Using this image function, the final composite image is produced following \Xrefequation{mosaic-stitch-shifted-tally} and \Xrefequation{mosaic-stitch-composite}.
    The response from each channel noticeably varies, attributed to inconsistencies in electronic coupling at the contact-wirebond interface.
    If the wirebond disrupts the thin film metal contact on the sensor surface, subsequently the electrical field through that cross-section of the sensor will also be distorted.
    This affects both the charge transport properties as well as the signal formation on the contact surface for the affected pixel.
    The composite image may be formed from all, or a subset of available channels to enhance the final image, as shown in \Xreffigure*{lise-16ch-mosaic-composite}.
    Numbering the channels from left to right and top to bottom, channels 7, 10, and 12 were omitted to generate \Xreffigure{lise-16ch-mosaic-composite:b}.
    This comparison shows the minor improvement obtained by eliminating poor performing channels.
    The lower left region more clearly resolves the last couple of line pairs, however a pixel sized blank space exists having removed the edge pixel.
    Both images show nearly identical right halves, resolving the same line pair resolution based on visual inspection.
% ============================================================================ %
\end{document}%