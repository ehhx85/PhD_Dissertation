% ./appendices/general-chemistry/sensor-etching.tex
% ============================================================================ %
\documentclass[../../main.tex]{subfiles}%
\begin{document}%
% ============================================================================ %
    \section{Sensor Etching}%
    \label{app:general-chemistry:sensor-etching}%
    % ======================================================================== %
    The raw sensor is cut and shaped from the larger, cylindrical crystal boule after growth.
    The individual radiation sensor is extracted from the bulk using a diamond wire saw, with wire diameters ranging from \SIrange{100}{500}{\micro\meter}.
    While the saw gives excellent shape control, creating cuboid sensors from the cylindrical boule, the edges are rough and leave the crystal surface in a stressed state.
    This creates defect sites which create problems for metal adhesion, as well as offer recombination centers for charge trapping, negatively effecting the electrical performance.
    To return the crystal surface to a pristine, optically smooth state, the sensor is first treated with a mechanical polish and subsequent chemical etch.
    % ======================================================================== %
    \par%
    % ======================================================================== %
    Depending on the sensor chemical composition, aqueous etching may be accomplished using a variety of solutions including: \ce{HF}, \ce{HCl}, \ce{NH_4OH}, and \ce{Br_2}. 
    Halogen molecules, in particular the \ce{Br_2}, have been used for a range of materials including I-III-VI semiconductors \cite{Causier_2011,Shen_2016}.
    It was shown for \ce{CdZnTe} radiation detectors, a \SI{5.0}{\percent} \gls{brometh} solution successfully removed the surface oxides introduced during sawing and handling \cite{Rouse_2002}.
    In removing the oxides with the acidic etchant, the surface was left \ce{Te}-rich.
    The \SI{20}{\angstrom} \ce{Te}-rich surface was removed \gls{latin:in-situ} via \ce{Ar} sputtering before metalization, revealing a stoichiometric surface.
    % ======================================================================== %
    \par%
    % ======================================================================== %
    After opting to use the \gls{brometh} etchant, establishing a material specific procedure was required.
    Typically, bromine concentrations of \SIrange{1.0}{5.0}{\percent} by volume, are used for primary etching, followed by weaker \SI{0.05}{\percent} wet etch or chemical polish \cite{Aspnes_1981}.
    The use of dilute etchant with low abrasion polishing pads protects the pad material while offering a finer polish with minimal residue.
    For softer materials, such as \gls{lise}, the chemical polish is removed from the sequence, using a milder wet etching series.
    The required chemicals for the general etching process are listed in \Xreftable*{sensor-etch-chemicals}.
    Because the \gls{brometh} etch is frequently applied in semiconductor processing, the fundamental chemical reactions and imposed safety concerns are often overlooked.
    To better understand the etching mechanism, the following reactions have been listed (\Xrefequationrange{bromine-methanol-reaction-1}{bromine-methanol-reaction-4}) as discussed by \citeauthor*{Bowman_1990} \cite{Bowman_1990}.
    % ======================================================================== %
    \Xequationfile{bromine-methanol-reaction-1}%
    \Xequationfile{bromine-methanol-reaction-2}%
    \Xequationfile{bromine-methanol-reaction-3}%
    \Xequationfile{bromine-methanol-reaction-4}%
    % ======================================================================== %
    The study investigated the reaction dynamics of concentrated \gls{brometh} solutions ranging from \SIrange{10.0}{25.0}{\percent}. 
    While all of these reactions may not be present in the dilute \SI{1.0}{\percent} concentration solution, they should be anticipated.
    Any contaminants, an inherent consequence of impurities produced by polishing the sensor material, may present themselves as inhibitors or catalysts, unpredictably altering the reaction rate.
    The general sensor etching process has been summarized in \Xreftable*{sensor-etch-process}.
    The etching sequence utilizes 3 separate Pyrex beakers to effectively etch the sensor while avoiding surface staining.
    While the first beaker holds the primary etchant (deep red color), the second beaker will become contaminated with bromine and dissolved sensor material, turning a light yellow color.
    The second beaker will halt the etching process, diluting the remaining bromine adsorbed on the sensor surface.
    The third beaker serves as a final rinse removing any remaining surface contamination, permitting safe handling of the sensor.
    % ======================================================================== %
    \par%
    % ======================================================================== %
    Because the \gls{lise} sensors are relatively soft, the polishing and etching process should be implemented gradually.
    It was discovered from experimentation, excessive polishing would create a bowed surface, non-ideal for bump bonding.
    Consecutively cycling between polishing and etching helped to approach an ideal surface for contact patterning.
% ============================================================================ %
\end{document}%