% ./appendices/general-chemistry/pcb-etching.tex
% ============================================================================ %
\documentclass[../../main.tex]{subfiles}
\begin{document}%
% ============================================================================ %
    \section{PCB Etching}%
    \label{app:general-chemistry:pcb-etching}%
    % ======================================================================== %
    \Glspl{pcb} are essential components used in all semiconducting detection systems, and more generally computer based data acquisition electronics. 
    A simple, one-sided \gls{pcb} consists of an insulating material, such as \gls{fr4} fiber laminates, clad with a thin layer of conducting material. 
    Common applications utilize copper cladding deposited by electroplating or laminating, with film thicknesses ranging from \SIrange{0.5}{2.0}{oz\per ft\raiseto{2}} (\SIrange{18}{70}{\micro\meter}).
    The copper etching process outlined in this section will follow successful patterning with the photolithography techniques described in \Xrefappendix{thin-film-processing:positive-photolithography}.
    The etching process can be highly exothermic and should be conducted in a fume hood with the sash closed or at wrist level as shown in \Xreffigure*{etching-station}.
    The correct \gls{ppe} should be used at all times and can be found in the \gls{sds} for the chemicals listed in \Xreftable*{pcb-etch-chemicals}.
    The etching solution created in this method can be recovered and reused if fabricating a large number of \glspl{pcb}.
    During prototyping, boards were either etched by themselves or as a pair and the solution was safely disposed as acid waste.
    % ======================================================================== %
    \par%
    % ======================================================================== %
    An industry standard, cupric chloride etching provides a straightforward method for fabricating \glspl{pcb} in the laboratory environment.
    Cupric chloride offers the following benefits over the alternative, ferric chloride: it does not contaminate the solution with iron, it creates a transparent solution for monitoring etch rate, the solution can be regenerated, and etch rate can be adjusted relatively easy.
    While there are only four elements in this solution (\ce{H}, \ce{O}, \ce{Cl}, \ce{Cu}), the etch rate can be dramatically altered by the cupric chloride and cuprous chloride concentrations.
    % ======================================================================== %
    \par%
    % ======================================================================== %
    The metallic copper is dissolved by aqueous copper(II) via the \gls{redox} reaction, yielding copper(I) (\Xrefequation{copper-etch-1}).
    % ======================================================================== %
    \Xequationfile{copper-etch-1}%
    % ======================================================================== %
    In a hydrochloric acid solution, cupric chloride will oxidize metallic copper (\Xrefequation{copper-etch-2}), dissolving the cladding into solution as cuprous chloride.
    % ======================================================================== %
    \Xequationfile{copper-etch-2}%
    % ======================================================================== %
    The etchant recharge mechanism relies on the oxidation of \ce{Cu1+} back to the usable \ce{Cu2+} form.
    Illustrated in the partial reaction (\Xrefequation{copper-etch-3}), this depends on free \ce{Cl^{-}} ion concentration in the solution.
    % ======================================================================== %
    \Xequationfile{copper-etch-3}%
    % ======================================================================== %
    To initially charge the solution with cupric chloride, concentrated hydrogen peroxide is introduced to a piece of pure copper foil.
    Under normal conditions, hydrogen peroxide will decompose on its own (\Xrefequation{copper-etch-4}) releasing oxygen gas.
    % ======================================================================== %
    \begin{subequations}%
        % ==================================================================== %
        \Xequationfile{copper-etch-4}%
        % ==================================================================== %
        The presence of free \ce{Cl^{-}} will catalyze the chain decomposition of \ce{H2O2} (\Xrefequation{copper-etch-5}) while leaving the \ce{Cl^{-}} in free solution, accelerating the release of oxygen (\Xrefequation{copper-etch-6}).
        % ==================================================================== %
        \Xequationfile{copper-etch-5}%
        \Xequationfile{copper-etch-6}%
        % ==================================================================== %
    \end{subequations}%
    % ======================================================================== %
    Introducing oxygen to the solution will oxidize metallic copper to cupric oxide (\Xrefequation{copper-etch-7}).
    % ======================================================================== %
    \Xequationfile{copper-etch-7}%
    % ======================================================================== %
    Cupric oxide, in the presence of \ce{HCl}, will yield cupric chloride (\Xrefequation{copper-etch-8}).
    % ======================================================================== %
    \Xequationfile{copper-etch-8}%
    % ======================================================================== %
    Experimenting with the concentrated (\SI{>30}{\percent}) hydrogen peroxide, it was possible to readily pre-charge the solution with cupric chloride ions.
    The initial piece of copper foil was sacrificially added to the solution with drops of concentrated \ce{H2O2}. 
    Initially, oxygen gas quickly releases from \ce{H2O2} breakdown, accelerated by the free \ce{HCl}.
    The oxygen will serve as the initial oxidizing agent, removing copper ions from the foil surface and converting them to the desirable cupric chloride.
    As the free oxygen is consumed during copper oxidation, the freshly generated cupric chloride in the solution will take over as the primary etching agent.
    Adding copper strips, and promoting dissolution with concentrated \ce{H2O2} charges the etching solution over a period of minutes, instead of days as seen with dilute \ce{H2O2}.
    Given the amount of copper on the single sided \gls{preamp} \gls{pcb}, the board successfully etched, in a matter of minutes as shown in  \Xreffigure*{etching-process-over-time}.
    Immediately after submersing the \gls{pcb} in the etching solution, the acid begins to attack the raw metal (\Xreffigure{etching-process-over-time:a}.).
    Small gas bubbles form along favorable nucleation sites, such as crevices formed at the edges of the resist and the exposed copper.
    After the first minute, broad regions of copper along the periphery have begun to dissolve as seen in \Xreffigure{etching-process-over-time:b}.
    The outer trim ring begins to separate from the surrounding metal after about two and a half minutes, shown in \Xreffigure{etching-process-over-time:c}.
    Approaching the five minute mark (\Xreffigure{etching-process-over-time:d}), the only remaining copper are the resist coated circuit traces and labels.
    The copper etching process, as tested, has been summarized in \Xreftable*{pcb-etch-process}.
    The technique has been shown to successfully prototype custom circuit designs, in-house, using the positive photolithography process in the \gls{mprf}. 
% ============================================================================ %
\end{document}%